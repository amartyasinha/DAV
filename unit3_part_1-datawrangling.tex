\documentclass[11pt]{article}

    \usepackage[breakable]{tcolorbox}
    \usepackage{parskip} % Stop auto-indenting (to mimic markdown behaviour)
    

    % Basic figure setup, for now with no caption control since it's done
    % automatically by Pandoc (which extracts ![](path) syntax from Markdown).
    \usepackage{graphicx}
    % Maintain compatibility with old templates. Remove in nbconvert 6.0
    \let\Oldincludegraphics\includegraphics
    % Ensure that by default, figures have no caption (until we provide a
    % proper Figure object with a Caption API and a way to capture that
    % in the conversion process - todo).
    \usepackage{caption}
    \DeclareCaptionFormat{nocaption}{}
    \captionsetup{format=nocaption,aboveskip=0pt,belowskip=0pt}

    \usepackage{float}
    \floatplacement{figure}{H} % forces figures to be placed at the correct location
    \usepackage{xcolor} % Allow colors to be defined
    \usepackage{enumerate} % Needed for markdown enumerations to work
    \usepackage{geometry} % Used to adjust the document margins
    \usepackage{amsmath} % Equations
    \usepackage{amssymb} % Equations
    \usepackage{textcomp} % defines textquotesingle
    % Hack from http://tex.stackexchange.com/a/47451/13684:
    \AtBeginDocument{%
        \def\PYZsq{\textquotesingle}% Upright quotes in Pygmentized code
    }
    \usepackage{upquote} % Upright quotes for verbatim code
    \usepackage{eurosym} % defines \euro

    \usepackage{iftex}
    \ifPDFTeX
        \usepackage[T1]{fontenc}
        \IfFileExists{alphabeta.sty}{
              \usepackage{alphabeta}
          }{
              \usepackage[mathletters]{ucs}
              \usepackage[utf8x]{inputenc}
          }
    \else
        \usepackage{fontspec}
        \usepackage{unicode-math}
    \fi

    \usepackage{fancyvrb} % verbatim replacement that allows latex
    \usepackage{grffile} % extends the file name processing of package graphics
                         % to support a larger range
    \makeatletter % fix for old versions of grffile with XeLaTeX
    \@ifpackagelater{grffile}{2019/11/01}
    {
      % Do nothing on new versions
    }
    {
      \def\Gread@@xetex#1{%
        \IfFileExists{"\Gin@base".bb}%
        {\Gread@eps{\Gin@base.bb}}%
        {\Gread@@xetex@aux#1}%
      }
    }
    \makeatother
    \usepackage[Export]{adjustbox} % Used to constrain images to a maximum size
    \adjustboxset{max size={0.9\linewidth}{0.9\paperheight}}

    % The hyperref package gives us a pdf with properly built
    % internal navigation ('pdf bookmarks' for the table of contents,
    % internal cross-reference links, web links for URLs, etc.)
    \usepackage{hyperref}
    % The default LaTeX title has an obnoxious amount of whitespace. By default,
    % titling removes some of it. It also provides customization options.
    \usepackage{titling}
    \usepackage{longtable} % longtable support required by pandoc >1.10
    \usepackage{booktabs}  % table support for pandoc > 1.12.2
    \usepackage{array}     % table support for pandoc >= 2.11.3
    \usepackage{calc}      % table minipage width calculation for pandoc >= 2.11.1
    \usepackage[inline]{enumitem} % IRkernel/repr support (it uses the enumerate* environment)
    \usepackage[normalem]{ulem} % ulem is needed to support strikethroughs (\sout)
                                % normalem makes italics be italics, not underlines
    \usepackage{mathrsfs}
    

    
    % Colors for the hyperref package
    \definecolor{urlcolor}{rgb}{0,.145,.698}
    \definecolor{linkcolor}{rgb}{.71,0.21,0.01}
    \definecolor{citecolor}{rgb}{.12,.54,.11}

    % ANSI colors
    \definecolor{ansi-black}{HTML}{3E424D}
    \definecolor{ansi-black-intense}{HTML}{282C36}
    \definecolor{ansi-red}{HTML}{E75C58}
    \definecolor{ansi-red-intense}{HTML}{B22B31}
    \definecolor{ansi-green}{HTML}{00A250}
    \definecolor{ansi-green-intense}{HTML}{007427}
    \definecolor{ansi-yellow}{HTML}{DDB62B}
    \definecolor{ansi-yellow-intense}{HTML}{B27D12}
    \definecolor{ansi-blue}{HTML}{208FFB}
    \definecolor{ansi-blue-intense}{HTML}{0065CA}
    \definecolor{ansi-magenta}{HTML}{D160C4}
    \definecolor{ansi-magenta-intense}{HTML}{A03196}
    \definecolor{ansi-cyan}{HTML}{60C6C8}
    \definecolor{ansi-cyan-intense}{HTML}{258F8F}
    \definecolor{ansi-white}{HTML}{C5C1B4}
    \definecolor{ansi-white-intense}{HTML}{A1A6B2}
    \definecolor{ansi-default-inverse-fg}{HTML}{FFFFFF}
    \definecolor{ansi-default-inverse-bg}{HTML}{000000}

    % common color for the border for error outputs.
    \definecolor{outerrorbackground}{HTML}{FFDFDF}

    % commands and environments needed by pandoc snippets
    % extracted from the output of `pandoc -s`
    \providecommand{\tightlist}{%
      \setlength{\itemsep}{0pt}\setlength{\parskip}{0pt}}
    \DefineVerbatimEnvironment{Highlighting}{Verbatim}{commandchars=\\\{\}}
    % Add ',fontsize=\small' for more characters per line
    \newenvironment{Shaded}{}{}
    \newcommand{\KeywordTok}[1]{\textcolor[rgb]{0.00,0.44,0.13}{\textbf{{#1}}}}
    \newcommand{\DataTypeTok}[1]{\textcolor[rgb]{0.56,0.13,0.00}{{#1}}}
    \newcommand{\DecValTok}[1]{\textcolor[rgb]{0.25,0.63,0.44}{{#1}}}
    \newcommand{\BaseNTok}[1]{\textcolor[rgb]{0.25,0.63,0.44}{{#1}}}
    \newcommand{\FloatTok}[1]{\textcolor[rgb]{0.25,0.63,0.44}{{#1}}}
    \newcommand{\CharTok}[1]{\textcolor[rgb]{0.25,0.44,0.63}{{#1}}}
    \newcommand{\StringTok}[1]{\textcolor[rgb]{0.25,0.44,0.63}{{#1}}}
    \newcommand{\CommentTok}[1]{\textcolor[rgb]{0.38,0.63,0.69}{\textit{{#1}}}}
    \newcommand{\OtherTok}[1]{\textcolor[rgb]{0.00,0.44,0.13}{{#1}}}
    \newcommand{\AlertTok}[1]{\textcolor[rgb]{1.00,0.00,0.00}{\textbf{{#1}}}}
    \newcommand{\FunctionTok}[1]{\textcolor[rgb]{0.02,0.16,0.49}{{#1}}}
    \newcommand{\RegionMarkerTok}[1]{{#1}}
    \newcommand{\ErrorTok}[1]{\textcolor[rgb]{1.00,0.00,0.00}{\textbf{{#1}}}}
    \newcommand{\NormalTok}[1]{{#1}}

    % Additional commands for more recent versions of Pandoc
    \newcommand{\ConstantTok}[1]{\textcolor[rgb]{0.53,0.00,0.00}{{#1}}}
    \newcommand{\SpecialCharTok}[1]{\textcolor[rgb]{0.25,0.44,0.63}{{#1}}}
    \newcommand{\VerbatimStringTok}[1]{\textcolor[rgb]{0.25,0.44,0.63}{{#1}}}
    \newcommand{\SpecialStringTok}[1]{\textcolor[rgb]{0.73,0.40,0.53}{{#1}}}
    \newcommand{\ImportTok}[1]{{#1}}
    \newcommand{\DocumentationTok}[1]{\textcolor[rgb]{0.73,0.13,0.13}{\textit{{#1}}}}
    \newcommand{\AnnotationTok}[1]{\textcolor[rgb]{0.38,0.63,0.69}{\textbf{\textit{{#1}}}}}
    \newcommand{\CommentVarTok}[1]{\textcolor[rgb]{0.38,0.63,0.69}{\textbf{\textit{{#1}}}}}
    \newcommand{\VariableTok}[1]{\textcolor[rgb]{0.10,0.09,0.49}{{#1}}}
    \newcommand{\ControlFlowTok}[1]{\textcolor[rgb]{0.00,0.44,0.13}{\textbf{{#1}}}}
    \newcommand{\OperatorTok}[1]{\textcolor[rgb]{0.40,0.40,0.40}{{#1}}}
    \newcommand{\BuiltInTok}[1]{{#1}}
    \newcommand{\ExtensionTok}[1]{{#1}}
    \newcommand{\PreprocessorTok}[1]{\textcolor[rgb]{0.74,0.48,0.00}{{#1}}}
    \newcommand{\AttributeTok}[1]{\textcolor[rgb]{0.49,0.56,0.16}{{#1}}}
    \newcommand{\InformationTok}[1]{\textcolor[rgb]{0.38,0.63,0.69}{\textbf{\textit{{#1}}}}}
    \newcommand{\WarningTok}[1]{\textcolor[rgb]{0.38,0.63,0.69}{\textbf{\textit{{#1}}}}}


    % Define a nice break command that doesn't care if a line doesn't already
    % exist.
    \def\br{\hspace*{\fill} \\* }
    % Math Jax compatibility definitions
    \def\gt{>}
    \def\lt{<}
    \let\Oldtex\TeX
    \let\Oldlatex\LaTeX
    \renewcommand{\TeX}{\textrm{\Oldtex}}
    \renewcommand{\LaTeX}{\textrm{\Oldlatex}}
    % Document parameters
    % Document title
    \title{unit3\_part\_1-datawrangling}
    
    
    
    
    
% Pygments definitions
\makeatletter
\def\PY@reset{\let\PY@it=\relax \let\PY@bf=\relax%
    \let\PY@ul=\relax \let\PY@tc=\relax%
    \let\PY@bc=\relax \let\PY@ff=\relax}
\def\PY@tok#1{\csname PY@tok@#1\endcsname}
\def\PY@toks#1+{\ifx\relax#1\empty\else%
    \PY@tok{#1}\expandafter\PY@toks\fi}
\def\PY@do#1{\PY@bc{\PY@tc{\PY@ul{%
    \PY@it{\PY@bf{\PY@ff{#1}}}}}}}
\def\PY#1#2{\PY@reset\PY@toks#1+\relax+\PY@do{#2}}

\@namedef{PY@tok@w}{\def\PY@tc##1{\textcolor[rgb]{0.73,0.73,0.73}{##1}}}
\@namedef{PY@tok@c}{\let\PY@it=\textit\def\PY@tc##1{\textcolor[rgb]{0.24,0.48,0.48}{##1}}}
\@namedef{PY@tok@cp}{\def\PY@tc##1{\textcolor[rgb]{0.61,0.40,0.00}{##1}}}
\@namedef{PY@tok@k}{\let\PY@bf=\textbf\def\PY@tc##1{\textcolor[rgb]{0.00,0.50,0.00}{##1}}}
\@namedef{PY@tok@kp}{\def\PY@tc##1{\textcolor[rgb]{0.00,0.50,0.00}{##1}}}
\@namedef{PY@tok@kt}{\def\PY@tc##1{\textcolor[rgb]{0.69,0.00,0.25}{##1}}}
\@namedef{PY@tok@o}{\def\PY@tc##1{\textcolor[rgb]{0.40,0.40,0.40}{##1}}}
\@namedef{PY@tok@ow}{\let\PY@bf=\textbf\def\PY@tc##1{\textcolor[rgb]{0.67,0.13,1.00}{##1}}}
\@namedef{PY@tok@nb}{\def\PY@tc##1{\textcolor[rgb]{0.00,0.50,0.00}{##1}}}
\@namedef{PY@tok@nf}{\def\PY@tc##1{\textcolor[rgb]{0.00,0.00,1.00}{##1}}}
\@namedef{PY@tok@nc}{\let\PY@bf=\textbf\def\PY@tc##1{\textcolor[rgb]{0.00,0.00,1.00}{##1}}}
\@namedef{PY@tok@nn}{\let\PY@bf=\textbf\def\PY@tc##1{\textcolor[rgb]{0.00,0.00,1.00}{##1}}}
\@namedef{PY@tok@ne}{\let\PY@bf=\textbf\def\PY@tc##1{\textcolor[rgb]{0.80,0.25,0.22}{##1}}}
\@namedef{PY@tok@nv}{\def\PY@tc##1{\textcolor[rgb]{0.10,0.09,0.49}{##1}}}
\@namedef{PY@tok@no}{\def\PY@tc##1{\textcolor[rgb]{0.53,0.00,0.00}{##1}}}
\@namedef{PY@tok@nl}{\def\PY@tc##1{\textcolor[rgb]{0.46,0.46,0.00}{##1}}}
\@namedef{PY@tok@ni}{\let\PY@bf=\textbf\def\PY@tc##1{\textcolor[rgb]{0.44,0.44,0.44}{##1}}}
\@namedef{PY@tok@na}{\def\PY@tc##1{\textcolor[rgb]{0.41,0.47,0.13}{##1}}}
\@namedef{PY@tok@nt}{\let\PY@bf=\textbf\def\PY@tc##1{\textcolor[rgb]{0.00,0.50,0.00}{##1}}}
\@namedef{PY@tok@nd}{\def\PY@tc##1{\textcolor[rgb]{0.67,0.13,1.00}{##1}}}
\@namedef{PY@tok@s}{\def\PY@tc##1{\textcolor[rgb]{0.73,0.13,0.13}{##1}}}
\@namedef{PY@tok@sd}{\let\PY@it=\textit\def\PY@tc##1{\textcolor[rgb]{0.73,0.13,0.13}{##1}}}
\@namedef{PY@tok@si}{\let\PY@bf=\textbf\def\PY@tc##1{\textcolor[rgb]{0.64,0.35,0.47}{##1}}}
\@namedef{PY@tok@se}{\let\PY@bf=\textbf\def\PY@tc##1{\textcolor[rgb]{0.67,0.36,0.12}{##1}}}
\@namedef{PY@tok@sr}{\def\PY@tc##1{\textcolor[rgb]{0.64,0.35,0.47}{##1}}}
\@namedef{PY@tok@ss}{\def\PY@tc##1{\textcolor[rgb]{0.10,0.09,0.49}{##1}}}
\@namedef{PY@tok@sx}{\def\PY@tc##1{\textcolor[rgb]{0.00,0.50,0.00}{##1}}}
\@namedef{PY@tok@m}{\def\PY@tc##1{\textcolor[rgb]{0.40,0.40,0.40}{##1}}}
\@namedef{PY@tok@gh}{\let\PY@bf=\textbf\def\PY@tc##1{\textcolor[rgb]{0.00,0.00,0.50}{##1}}}
\@namedef{PY@tok@gu}{\let\PY@bf=\textbf\def\PY@tc##1{\textcolor[rgb]{0.50,0.00,0.50}{##1}}}
\@namedef{PY@tok@gd}{\def\PY@tc##1{\textcolor[rgb]{0.63,0.00,0.00}{##1}}}
\@namedef{PY@tok@gi}{\def\PY@tc##1{\textcolor[rgb]{0.00,0.52,0.00}{##1}}}
\@namedef{PY@tok@gr}{\def\PY@tc##1{\textcolor[rgb]{0.89,0.00,0.00}{##1}}}
\@namedef{PY@tok@ge}{\let\PY@it=\textit}
\@namedef{PY@tok@gs}{\let\PY@bf=\textbf}
\@namedef{PY@tok@gp}{\let\PY@bf=\textbf\def\PY@tc##1{\textcolor[rgb]{0.00,0.00,0.50}{##1}}}
\@namedef{PY@tok@go}{\def\PY@tc##1{\textcolor[rgb]{0.44,0.44,0.44}{##1}}}
\@namedef{PY@tok@gt}{\def\PY@tc##1{\textcolor[rgb]{0.00,0.27,0.87}{##1}}}
\@namedef{PY@tok@err}{\def\PY@bc##1{{\setlength{\fboxsep}{\string -\fboxrule}\fcolorbox[rgb]{1.00,0.00,0.00}{1,1,1}{\strut ##1}}}}
\@namedef{PY@tok@kc}{\let\PY@bf=\textbf\def\PY@tc##1{\textcolor[rgb]{0.00,0.50,0.00}{##1}}}
\@namedef{PY@tok@kd}{\let\PY@bf=\textbf\def\PY@tc##1{\textcolor[rgb]{0.00,0.50,0.00}{##1}}}
\@namedef{PY@tok@kn}{\let\PY@bf=\textbf\def\PY@tc##1{\textcolor[rgb]{0.00,0.50,0.00}{##1}}}
\@namedef{PY@tok@kr}{\let\PY@bf=\textbf\def\PY@tc##1{\textcolor[rgb]{0.00,0.50,0.00}{##1}}}
\@namedef{PY@tok@bp}{\def\PY@tc##1{\textcolor[rgb]{0.00,0.50,0.00}{##1}}}
\@namedef{PY@tok@fm}{\def\PY@tc##1{\textcolor[rgb]{0.00,0.00,1.00}{##1}}}
\@namedef{PY@tok@vc}{\def\PY@tc##1{\textcolor[rgb]{0.10,0.09,0.49}{##1}}}
\@namedef{PY@tok@vg}{\def\PY@tc##1{\textcolor[rgb]{0.10,0.09,0.49}{##1}}}
\@namedef{PY@tok@vi}{\def\PY@tc##1{\textcolor[rgb]{0.10,0.09,0.49}{##1}}}
\@namedef{PY@tok@vm}{\def\PY@tc##1{\textcolor[rgb]{0.10,0.09,0.49}{##1}}}
\@namedef{PY@tok@sa}{\def\PY@tc##1{\textcolor[rgb]{0.73,0.13,0.13}{##1}}}
\@namedef{PY@tok@sb}{\def\PY@tc##1{\textcolor[rgb]{0.73,0.13,0.13}{##1}}}
\@namedef{PY@tok@sc}{\def\PY@tc##1{\textcolor[rgb]{0.73,0.13,0.13}{##1}}}
\@namedef{PY@tok@dl}{\def\PY@tc##1{\textcolor[rgb]{0.73,0.13,0.13}{##1}}}
\@namedef{PY@tok@s2}{\def\PY@tc##1{\textcolor[rgb]{0.73,0.13,0.13}{##1}}}
\@namedef{PY@tok@sh}{\def\PY@tc##1{\textcolor[rgb]{0.73,0.13,0.13}{##1}}}
\@namedef{PY@tok@s1}{\def\PY@tc##1{\textcolor[rgb]{0.73,0.13,0.13}{##1}}}
\@namedef{PY@tok@mb}{\def\PY@tc##1{\textcolor[rgb]{0.40,0.40,0.40}{##1}}}
\@namedef{PY@tok@mf}{\def\PY@tc##1{\textcolor[rgb]{0.40,0.40,0.40}{##1}}}
\@namedef{PY@tok@mh}{\def\PY@tc##1{\textcolor[rgb]{0.40,0.40,0.40}{##1}}}
\@namedef{PY@tok@mi}{\def\PY@tc##1{\textcolor[rgb]{0.40,0.40,0.40}{##1}}}
\@namedef{PY@tok@il}{\def\PY@tc##1{\textcolor[rgb]{0.40,0.40,0.40}{##1}}}
\@namedef{PY@tok@mo}{\def\PY@tc##1{\textcolor[rgb]{0.40,0.40,0.40}{##1}}}
\@namedef{PY@tok@ch}{\let\PY@it=\textit\def\PY@tc##1{\textcolor[rgb]{0.24,0.48,0.48}{##1}}}
\@namedef{PY@tok@cm}{\let\PY@it=\textit\def\PY@tc##1{\textcolor[rgb]{0.24,0.48,0.48}{##1}}}
\@namedef{PY@tok@cpf}{\let\PY@it=\textit\def\PY@tc##1{\textcolor[rgb]{0.24,0.48,0.48}{##1}}}
\@namedef{PY@tok@c1}{\let\PY@it=\textit\def\PY@tc##1{\textcolor[rgb]{0.24,0.48,0.48}{##1}}}
\@namedef{PY@tok@cs}{\let\PY@it=\textit\def\PY@tc##1{\textcolor[rgb]{0.24,0.48,0.48}{##1}}}

\def\PYZbs{\char`\\}
\def\PYZus{\char`\_}
\def\PYZob{\char`\{}
\def\PYZcb{\char`\}}
\def\PYZca{\char`\^}
\def\PYZam{\char`\&}
\def\PYZlt{\char`\<}
\def\PYZgt{\char`\>}
\def\PYZsh{\char`\#}
\def\PYZpc{\char`\%}
\def\PYZdl{\char`\$}
\def\PYZhy{\char`\-}
\def\PYZsq{\char`\'}
\def\PYZdq{\char`\"}
\def\PYZti{\char`\~}
% for compatibility with earlier versions
\def\PYZat{@}
\def\PYZlb{[}
\def\PYZrb{]}
\makeatother


    % For linebreaks inside Verbatim environment from package fancyvrb.
    \makeatletter
        \newbox\Wrappedcontinuationbox
        \newbox\Wrappedvisiblespacebox
        \newcommand*\Wrappedvisiblespace {\textcolor{red}{\textvisiblespace}}
        \newcommand*\Wrappedcontinuationsymbol {\textcolor{red}{\llap{\tiny$\m@th\hookrightarrow$}}}
        \newcommand*\Wrappedcontinuationindent {3ex }
        \newcommand*\Wrappedafterbreak {\kern\Wrappedcontinuationindent\copy\Wrappedcontinuationbox}
        % Take advantage of the already applied Pygments mark-up to insert
        % potential linebreaks for TeX processing.
        %        {, <, #, %, $, ' and ": go to next line.
        %        _, }, ^, &, >, - and ~: stay at end of broken line.
        % Use of \textquotesingle for straight quote.
        \newcommand*\Wrappedbreaksatspecials {%
            \def\PYGZus{\discretionary{\char`\_}{\Wrappedafterbreak}{\char`\_}}%
            \def\PYGZob{\discretionary{}{\Wrappedafterbreak\char`\{}{\char`\{}}%
            \def\PYGZcb{\discretionary{\char`\}}{\Wrappedafterbreak}{\char`\}}}%
            \def\PYGZca{\discretionary{\char`\^}{\Wrappedafterbreak}{\char`\^}}%
            \def\PYGZam{\discretionary{\char`\&}{\Wrappedafterbreak}{\char`\&}}%
            \def\PYGZlt{\discretionary{}{\Wrappedafterbreak\char`\<}{\char`\<}}%
            \def\PYGZgt{\discretionary{\char`\>}{\Wrappedafterbreak}{\char`\>}}%
            \def\PYGZsh{\discretionary{}{\Wrappedafterbreak\char`\#}{\char`\#}}%
            \def\PYGZpc{\discretionary{}{\Wrappedafterbreak\char`\%}{\char`\%}}%
            \def\PYGZdl{\discretionary{}{\Wrappedafterbreak\char`\$}{\char`\$}}%
            \def\PYGZhy{\discretionary{\char`\-}{\Wrappedafterbreak}{\char`\-}}%
            \def\PYGZsq{\discretionary{}{\Wrappedafterbreak\textquotesingle}{\textquotesingle}}%
            \def\PYGZdq{\discretionary{}{\Wrappedafterbreak\char`\"}{\char`\"}}%
            \def\PYGZti{\discretionary{\char`\~}{\Wrappedafterbreak}{\char`\~}}%
        }
        % Some characters . , ; ? ! / are not pygmentized.
        % This macro makes them "active" and they will insert potential linebreaks
        \newcommand*\Wrappedbreaksatpunct {%
            \lccode`\~`\.\lowercase{\def~}{\discretionary{\hbox{\char`\.}}{\Wrappedafterbreak}{\hbox{\char`\.}}}%
            \lccode`\~`\,\lowercase{\def~}{\discretionary{\hbox{\char`\,}}{\Wrappedafterbreak}{\hbox{\char`\,}}}%
            \lccode`\~`\;\lowercase{\def~}{\discretionary{\hbox{\char`\;}}{\Wrappedafterbreak}{\hbox{\char`\;}}}%
            \lccode`\~`\:\lowercase{\def~}{\discretionary{\hbox{\char`\:}}{\Wrappedafterbreak}{\hbox{\char`\:}}}%
            \lccode`\~`\?\lowercase{\def~}{\discretionary{\hbox{\char`\?}}{\Wrappedafterbreak}{\hbox{\char`\?}}}%
            \lccode`\~`\!\lowercase{\def~}{\discretionary{\hbox{\char`\!}}{\Wrappedafterbreak}{\hbox{\char`\!}}}%
            \lccode`\~`\/\lowercase{\def~}{\discretionary{\hbox{\char`\/}}{\Wrappedafterbreak}{\hbox{\char`\/}}}%
            \catcode`\.\active
            \catcode`\,\active
            \catcode`\;\active
            \catcode`\:\active
            \catcode`\?\active
            \catcode`\!\active
            \catcode`\/\active
            \lccode`\~`\~
        }
    \makeatother

    \let\OriginalVerbatim=\Verbatim
    \makeatletter
    \renewcommand{\Verbatim}[1][1]{%
        %\parskip\z@skip
        \sbox\Wrappedcontinuationbox {\Wrappedcontinuationsymbol}%
        \sbox\Wrappedvisiblespacebox {\FV@SetupFont\Wrappedvisiblespace}%
        \def\FancyVerbFormatLine ##1{\hsize\linewidth
            \vtop{\raggedright\hyphenpenalty\z@\exhyphenpenalty\z@
                \doublehyphendemerits\z@\finalhyphendemerits\z@
                \strut ##1\strut}%
        }%
        % If the linebreak is at a space, the latter will be displayed as visible
        % space at end of first line, and a continuation symbol starts next line.
        % Stretch/shrink are however usually zero for typewriter font.
        \def\FV@Space {%
            \nobreak\hskip\z@ plus\fontdimen3\font minus\fontdimen4\font
            \discretionary{\copy\Wrappedvisiblespacebox}{\Wrappedafterbreak}
            {\kern\fontdimen2\font}%
        }%

        % Allow breaks at special characters using \PYG... macros.
        \Wrappedbreaksatspecials
        % Breaks at punctuation characters . , ; ? ! and / need catcode=\active
        \OriginalVerbatim[#1,codes*=\Wrappedbreaksatpunct]%
    }
    \makeatother

    % Exact colors from NB
    \definecolor{incolor}{HTML}{303F9F}
    \definecolor{outcolor}{HTML}{D84315}
    \definecolor{cellborder}{HTML}{CFCFCF}
    \definecolor{cellbackground}{HTML}{F7F7F7}

    % prompt
    \makeatletter
    \newcommand{\boxspacing}{\kern\kvtcb@left@rule\kern\kvtcb@boxsep}
    \makeatother
    \newcommand{\prompt}[4]{
        {\ttfamily\llap{{\color{#2}[#3]:\hspace{3pt}#4}}\vspace{-\baselineskip}}
    }
    

    
    % Prevent overflowing lines due to hard-to-break entities
    \sloppy
    % Setup hyperref package
    \hypersetup{
      breaklinks=true,  % so long urls are correctly broken across lines
      colorlinks=true,
      urlcolor=urlcolor,
      linkcolor=linkcolor,
      citecolor=citecolor,
      }
    % Slightly bigger margins than the latex defaults
    
    \geometry{verbose,tmargin=1in,bmargin=1in,lmargin=1in,rmargin=1in}
    
    

\begin{document}
    
    \maketitle
    
    

    
    \hypertarget{data-wrangling-i}{%
\section{data wrangling-I}\label{data-wrangling-i}}

    \begin{tcolorbox}[breakable, size=fbox, boxrule=1pt, pad at break*=1mm,colback=cellbackground, colframe=cellborder]
\prompt{In}{incolor}{1}{\boxspacing}
\begin{Verbatim}[commandchars=\\\{\}]
\PY{k+kn}{import} \PY{n+nn}{pandas} \PY{k}{as} \PY{n+nn}{pd} 
\PY{k+kn}{import} \PY{n+nn}{numpy} \PY{k}{as} \PY{n+nn}{np}
\end{Verbatim}
\end{tcolorbox}

    \#to upload files in content folder

    \begin{tcolorbox}[breakable, size=fbox, boxrule=1pt, pad at break*=1mm,colback=cellbackground, colframe=cellborder]
\prompt{In}{incolor}{57}{\boxspacing}
\begin{Verbatim}[commandchars=\\\{\}]
\PY{k+kn}{from} \PY{n+nn}{google}\PY{n+nn}{.}\PY{n+nn}{colab} \PY{k+kn}{import} \PY{n}{files}
\PY{n}{uploaded} \PY{o}{=} \PY{n}{files}\PY{o}{.}\PY{n}{upload}\PY{p}{(}\PY{p}{)}
\end{Verbatim}
\end{tcolorbox}

    
    \begin{Verbatim}[commandchars=\\\{\}]
<IPython.core.display.HTML object>
    \end{Verbatim}

    
    \begin{Verbatim}[commandchars=\\\{\}]
Saving subjects-count-course.xlsx to subjects-count-course (1).xlsx
    \end{Verbatim}

    \hypertarget{reading-a-xls-having-details-regarding-number-of-coredsesec-subjects}{%
\section{Reading a xls having details regarding number of core/dse/sec
subjects}\label{reading-a-xls-having-details-regarding-number-of-coredsesec-subjects}}

    \begin{tcolorbox}[breakable, size=fbox, boxrule=1pt, pad at break*=1mm,colback=cellbackground, colframe=cellborder]
\prompt{In}{incolor}{2}{\boxspacing}
\begin{Verbatim}[commandchars=\\\{\}]
\PY{n}{f1} \PY{o}{=} \PY{n}{pd}\PY{o}{.}\PY{n}{ExcelFile}\PY{p}{(}\PY{l+s+s1}{\PYZsq{}}\PY{l+s+s1}{D:}\PY{l+s+s1}{\PYZbs{}}\PY{l+s+s1}{cs(h)Vsem Data analysis and visulaization 2021}\PY{l+s+s1}{\PYZbs{}}\PY{l+s+s1}{programs}\PY{l+s+s1}{\PYZbs{}}\PY{l+s+s1}{pandas}\PY{l+s+s1}{\PYZbs{}}\PY{l+s+s1}{subjects\PYZhy{}count\PYZhy{}course.xlsx}\PY{l+s+s1}{\PYZsq{}}\PY{p}{)}
\end{Verbatim}
\end{tcolorbox}

    \begin{tcolorbox}[breakable, size=fbox, boxrule=1pt, pad at break*=1mm,colback=cellbackground, colframe=cellborder]
\prompt{In}{incolor}{3}{\boxspacing}
\begin{Verbatim}[commandchars=\\\{\}]
\PY{n}{DF}\PY{o}{=}\PY{n}{pd}\PY{o}{.}\PY{n}{read\PYZus{}excel}\PY{p}{(}\PY{n}{f1}\PY{p}{,}\PY{n}{sheet\PYZus{}name}\PY{o}{=}\PY{l+m+mi}{0}\PY{p}{)}
\end{Verbatim}
\end{tcolorbox}

    \begin{tcolorbox}[breakable, size=fbox, boxrule=1pt, pad at break*=1mm,colback=cellbackground, colframe=cellborder]
\prompt{In}{incolor}{5}{\boxspacing}
\begin{Verbatim}[commandchars=\\\{\}]
\PY{n}{DF}
\end{Verbatim}
\end{tcolorbox}

            \begin{tcolorbox}[breakable, size=fbox, boxrule=.5pt, pad at break*=1mm, opacityfill=0]
\prompt{Out}{outcolor}{5}{\boxspacing}
\begin{Verbatim}[commandchars=\\\{\}]
    Course Semester  Corepaper  SEC  DSE
0     PSCS        I          3  NIL  NIL
1     PSCS       II          3  NIL  NIL
2     PSCS      III          4    1  NIL
3     PSCS       IV          4    1  NIL
4     PSCS        V          4    1    1
5     PSCS       VI          4    1    1
6   CSHons        I          4    2  NIL
7   CSHons       II          4    2  NIL
8   CSHons      III          4    2  NIL
9   CSHons       IV          4    2  NIL
10  CSHons        V          4  NIL    2
11  CSHons       VI          4  NIL    2
12    Bcom        I          4    2  NIL
13    Bcom       II          4    2  NIL
14    Bcom      III          4    3    1
15    Bcom       IV          4    3  NIL
16  lifesc        I          5  NIL    1
17  lifesc       II          5    0    1
\end{Verbatim}
\end{tcolorbox}
        
    \hypertarget{hierarchical-indexing}{%
\section{hierarchical indexing}\label{hierarchical-indexing}}

    \hypertarget{setting-one-column-as-row-indexsingle-level-index}{%
\subsection{setting one column as row index:single level
index}\label{setting-one-column-as-row-indexsingle-level-index}}

    \begin{tcolorbox}[breakable, size=fbox, boxrule=1pt, pad at break*=1mm,colback=cellbackground, colframe=cellborder]
\prompt{In}{incolor}{4}{\boxspacing}
\begin{Verbatim}[commandchars=\\\{\}]
\PY{n}{DF1}\PY{o}{=}\PY{n}{DF}\PY{o}{.}\PY{n}{set\PYZus{}index}\PY{p}{(}\PY{n}{DF}\PY{o}{.}\PY{n}{columns}\PY{p}{[}\PY{l+m+mi}{0}\PY{p}{]}\PY{p}{)}
\end{Verbatim}
\end{tcolorbox}

    \begin{tcolorbox}[breakable, size=fbox, boxrule=1pt, pad at break*=1mm,colback=cellbackground, colframe=cellborder]
\prompt{In}{incolor}{7}{\boxspacing}
\begin{Verbatim}[commandchars=\\\{\}]
\PY{n}{DF1}\PY{o}{.}\PY{n}{index}
\end{Verbatim}
\end{tcolorbox}

            \begin{tcolorbox}[breakable, size=fbox, boxrule=.5pt, pad at break*=1mm, opacityfill=0]
\prompt{Out}{outcolor}{7}{\boxspacing}
\begin{Verbatim}[commandchars=\\\{\}]
Index(['PSCS', 'PSCS', 'PSCS', 'PSCS', 'PSCS', 'PSCS', 'CSHons', 'CSHons',
       'CSHons', 'CSHons', 'CSHons', 'CSHons', 'Bcom', 'Bcom', 'Bcom', 'Bcom',
       'lifesc', 'lifesc'],
      dtype='object', name='Course')
\end{Verbatim}
\end{tcolorbox}
        
    \begin{tcolorbox}[breakable, size=fbox, boxrule=1pt, pad at break*=1mm,colback=cellbackground, colframe=cellborder]
\prompt{In}{incolor}{5}{\boxspacing}
\begin{Verbatim}[commandchars=\\\{\}]
\PY{n}{DF1}
\end{Verbatim}
\end{tcolorbox}

            \begin{tcolorbox}[breakable, size=fbox, boxrule=.5pt, pad at break*=1mm, opacityfill=0]
\prompt{Out}{outcolor}{5}{\boxspacing}
\begin{Verbatim}[commandchars=\\\{\}]
       Semester  Corepaper  SEC  DSE
Course
PSCS          I          3  NIL  NIL
PSCS         II          3  NIL  NIL
PSCS        III          4    1  NIL
PSCS         IV          4    1  NIL
PSCS          V          4    1    1
PSCS         VI          4    1    1
CSHons        I          4    2  NIL
CSHons       II          4    2  NIL
CSHons      III          4    2  NIL
CSHons       IV          4    2  NIL
CSHons        V          4  NIL    2
CSHons       VI          4  NIL    2
Bcom          I          4    2  NIL
Bcom         II          4    2  NIL
Bcom        III          4    3    1
Bcom         IV          4    3  NIL
lifesc        I          5  NIL    1
lifesc       II          5    0    1
\end{Verbatim}
\end{tcolorbox}
        
    \hypertarget{setting-two-columns-as-row-indexhierarchical-level-index}{%
\subsection{setting two columns as row index:hierarchical level
index}\label{setting-two-columns-as-row-indexhierarchical-level-index}}

    \begin{tcolorbox}[breakable, size=fbox, boxrule=1pt, pad at break*=1mm,colback=cellbackground, colframe=cellborder]
\prompt{In}{incolor}{6}{\boxspacing}
\begin{Verbatim}[commandchars=\\\{\}]
\PY{n}{DF2}\PY{o}{=}\PY{n}{DF}\PY{o}{.}\PY{n}{set\PYZus{}index}\PY{p}{(}\PY{n}{keys}\PY{o}{=}\PY{p}{[}\PY{n}{DF}\PY{o}{.}\PY{n}{columns}\PY{p}{[}\PY{l+m+mi}{0}\PY{p}{]}\PY{p}{,}\PY{n}{DF}\PY{o}{.}\PY{n}{columns}\PY{p}{[}\PY{l+m+mi}{1}\PY{p}{]}\PY{p}{]}\PY{p}{)}
\end{Verbatim}
\end{tcolorbox}

    

    \begin{tcolorbox}[breakable, size=fbox, boxrule=1pt, pad at break*=1mm,colback=cellbackground, colframe=cellborder]
\prompt{In}{incolor}{9}{\boxspacing}
\begin{Verbatim}[commandchars=\\\{\}]
\PY{n}{I}\PY{o}{=}\PY{n}{DF2}\PY{o}{.}\PY{n}{index}
\end{Verbatim}
\end{tcolorbox}

    \begin{tcolorbox}[breakable, size=fbox, boxrule=1pt, pad at break*=1mm,colback=cellbackground, colframe=cellborder]
\prompt{In}{incolor}{12}{\boxspacing}
\begin{Verbatim}[commandchars=\\\{\}]
\PY{n}{I}\PY{o}{.}\PY{n}{levels}
\end{Verbatim}
\end{tcolorbox}

            \begin{tcolorbox}[breakable, size=fbox, boxrule=.5pt, pad at break*=1mm, opacityfill=0]
\prompt{Out}{outcolor}{12}{\boxspacing}
\begin{Verbatim}[commandchars=\\\{\}]
FrozenList([['Bcom', 'CSHons', 'PSCS', 'lifesc'], ['I', 'II', 'III', 'IV', 'V',
'VI']])
\end{Verbatim}
\end{tcolorbox}
        
    \begin{tcolorbox}[breakable, size=fbox, boxrule=1pt, pad at break*=1mm,colback=cellbackground, colframe=cellborder]
\prompt{In}{incolor}{13}{\boxspacing}
\begin{Verbatim}[commandchars=\\\{\}]
\PY{n}{I}\PY{o}{.}\PY{n}{codes}
\end{Verbatim}
\end{tcolorbox}

            \begin{tcolorbox}[breakable, size=fbox, boxrule=.5pt, pad at break*=1mm, opacityfill=0]
\prompt{Out}{outcolor}{13}{\boxspacing}
\begin{Verbatim}[commandchars=\\\{\}]
FrozenList([[2, 2, 2, 2, 2, 2, 1, 1, 1, 1, 1, 1, 0, 0, 0, 0, 3, 3], [0, 1, 2, 3,
4, 5, 0, 1, 2, 3, 4, 5, 0, 1, 2, 3, 0, 1]])
\end{Verbatim}
\end{tcolorbox}
        
    \begin{tcolorbox}[breakable, size=fbox, boxrule=1pt, pad at break*=1mm,colback=cellbackground, colframe=cellborder]
\prompt{In}{incolor}{10}{\boxspacing}
\begin{Verbatim}[commandchars=\\\{\}]
\PY{n}{DF2}
\end{Verbatim}
\end{tcolorbox}

            \begin{tcolorbox}[breakable, size=fbox, boxrule=.5pt, pad at break*=1mm, opacityfill=0]
\prompt{Out}{outcolor}{10}{\boxspacing}
\begin{Verbatim}[commandchars=\\\{\}]
                 Corepaper  SEC  DSE
Course Semester
PSCS   I                 3  NIL  NIL
       II                3  NIL  NIL
       III               4    1  NIL
       IV                4    1  NIL
       V                 4    1    1
       VI                4    1    1
CSHons I                 4    2  NIL
       II                4    2  NIL
       III               4    2  NIL
       IV                4    2  NIL
       V                 4  NIL    2
       VI                4  NIL    2
Bcom   I                 4    2  NIL
       II                4    2  NIL
       III               4    3    1
       IV                4    3  NIL
lifesc I                 5  NIL    1
       II                5    0    1
\end{Verbatim}
\end{tcolorbox}
        
    \begin{tcolorbox}[breakable, size=fbox, boxrule=1pt, pad at break*=1mm,colback=cellbackground, colframe=cellborder]
\prompt{In}{incolor}{14}{\boxspacing}
\begin{Verbatim}[commandchars=\\\{\}]
\PY{n}{DF2}\PY{o}{.}\PY{n}{index}\PY{o}{.}\PY{n}{names}
\end{Verbatim}
\end{tcolorbox}

            \begin{tcolorbox}[breakable, size=fbox, boxrule=.5pt, pad at break*=1mm, opacityfill=0]
\prompt{Out}{outcolor}{14}{\boxspacing}
\begin{Verbatim}[commandchars=\\\{\}]
FrozenList(['Course', 'Semester'])
\end{Verbatim}
\end{tcolorbox}
        
    \hypertarget{accessing-values-using-multiindex-using-tuple}{%
\subsection{Accessing values using multiindex using
tuple()}\label{accessing-values-using-multiindex-using-tuple}}

    \begin{tcolorbox}[breakable, size=fbox, boxrule=1pt, pad at break*=1mm,colback=cellbackground, colframe=cellborder]
\prompt{In}{incolor}{15}{\boxspacing}
\begin{Verbatim}[commandchars=\\\{\}]
\PY{n}{DF2}\PY{o}{.}\PY{n}{loc}\PY{p}{[}\PY{p}{(}\PY{l+s+s1}{\PYZsq{}}\PY{l+s+s1}{Bcom}\PY{l+s+s1}{\PYZsq{}}\PY{p}{,}\PY{l+s+s1}{\PYZsq{}}\PY{l+s+s1}{I}\PY{l+s+s1}{\PYZsq{}}\PY{p}{)}\PY{p}{]}
\end{Verbatim}
\end{tcolorbox}

            \begin{tcolorbox}[breakable, size=fbox, boxrule=.5pt, pad at break*=1mm, opacityfill=0]
\prompt{Out}{outcolor}{15}{\boxspacing}
\begin{Verbatim}[commandchars=\\\{\}]
Corepaper      4
SEC            2
DSE          NIL
Name: (Bcom, I), dtype: object
\end{Verbatim}
\end{tcolorbox}
        
    \begin{tcolorbox}[breakable, size=fbox, boxrule=1pt, pad at break*=1mm,colback=cellbackground, colframe=cellborder]
\prompt{In}{incolor}{48}{\boxspacing}
\begin{Verbatim}[commandchars=\\\{\}]
\PY{n}{DF2}
\end{Verbatim}
\end{tcolorbox}

            \begin{tcolorbox}[breakable, size=fbox, boxrule=.5pt, pad at break*=1mm, opacityfill=0]
\prompt{Out}{outcolor}{48}{\boxspacing}
\begin{Verbatim}[commandchars=\\\{\}]
                 corepaper  SEC  DSE
course Semester
PSCS   I                 3  NIL  NIL
       II                3  NIL  NIL
       III               4    1  NIL
       IV                4    1  NIL
       V                 4    1    1
       VI                4    1    1
CSHons I                 4    2  NIL
       II                4    2  NIL
       III               4    2  NIL
       IV                4    2  NIL
       V                 4  NIL    2
       VI                4  NIL    2
Bcom   I                 4    2  NIL
       II                4    2  NIL
       III               4    3    1
       IV                4    3  NIL
lifesc I                 5  NIL    1
       II                5    0    1
\end{Verbatim}
\end{tcolorbox}
        
    \begin{tcolorbox}[breakable, size=fbox, boxrule=1pt, pad at break*=1mm,colback=cellbackground, colframe=cellborder]
\prompt{In}{incolor}{16}{\boxspacing}
\begin{Verbatim}[commandchars=\\\{\}]
\PY{n}{DF2}\PY{o}{.}\PY{n}{shape}
\end{Verbatim}
\end{tcolorbox}

            \begin{tcolorbox}[breakable, size=fbox, boxrule=.5pt, pad at break*=1mm, opacityfill=0]
\prompt{Out}{outcolor}{16}{\boxspacing}
\begin{Verbatim}[commandchars=\\\{\}]
(18, 3)
\end{Verbatim}
\end{tcolorbox}
        
    \hypertarget{accessing-rows-of-a-particular-course-index}{%
\subsection{accessing rows of a particular course
index}\label{accessing-rows-of-a-particular-course-index}}

    \begin{tcolorbox}[breakable, size=fbox, boxrule=1pt, pad at break*=1mm,colback=cellbackground, colframe=cellborder]
\prompt{In}{incolor}{17}{\boxspacing}
\begin{Verbatim}[commandchars=\\\{\}]
\PY{n}{DF2}\PY{o}{.}\PY{n}{loc}\PY{p}{[}\PY{l+s+s1}{\PYZsq{}}\PY{l+s+s1}{Bcom}\PY{l+s+s1}{\PYZsq{}}\PY{p}{]}
\end{Verbatim}
\end{tcolorbox}

            \begin{tcolorbox}[breakable, size=fbox, boxrule=.5pt, pad at break*=1mm, opacityfill=0]
\prompt{Out}{outcolor}{17}{\boxspacing}
\begin{Verbatim}[commandchars=\\\{\}]
          Corepaper SEC  DSE
Semester
I                 4   2  NIL
II                4   2  NIL
III               4   3    1
IV                4   3  NIL
\end{Verbatim}
\end{tcolorbox}
        
    \begin{tcolorbox}[breakable, size=fbox, boxrule=1pt, pad at break*=1mm,colback=cellbackground, colframe=cellborder]
\prompt{In}{incolor}{18}{\boxspacing}
\begin{Verbatim}[commandchars=\\\{\}]
\PY{n}{DF2}\PY{o}{.}\PY{n}{loc}\PY{p}{[}\PY{p}{[}\PY{l+s+s1}{\PYZsq{}}\PY{l+s+s1}{Bcom}\PY{l+s+s1}{\PYZsq{}}\PY{p}{,}\PY{l+s+s1}{\PYZsq{}}\PY{l+s+s1}{PSCS}\PY{l+s+s1}{\PYZsq{}}\PY{p}{]}\PY{p}{,}\PY{p}{:}\PY{p}{]}
\end{Verbatim}
\end{tcolorbox}

            \begin{tcolorbox}[breakable, size=fbox, boxrule=.5pt, pad at break*=1mm, opacityfill=0]
\prompt{Out}{outcolor}{18}{\boxspacing}
\begin{Verbatim}[commandchars=\\\{\}]
                 Corepaper  SEC  DSE
Course Semester
Bcom   I                 4    2  NIL
       II                4    2  NIL
       III               4    3    1
       IV                4    3  NIL
PSCS   I                 3  NIL  NIL
       II                3  NIL  NIL
       III               4    1  NIL
       IV                4    1  NIL
       V                 4    1    1
       VI                4    1    1
\end{Verbatim}
\end{tcolorbox}
        
    \begin{tcolorbox}[breakable, size=fbox, boxrule=1pt, pad at break*=1mm,colback=cellbackground, colframe=cellborder]
\prompt{In}{incolor}{19}{\boxspacing}
\begin{Verbatim}[commandchars=\\\{\}]
\PY{n}{DF2}\PY{o}{.}\PY{n}{loc}\PY{p}{[}\PY{p}{(}\PY{l+s+s1}{\PYZsq{}}\PY{l+s+s1}{Bcom}\PY{l+s+s1}{\PYZsq{}}\PY{p}{,}\PY{l+s+s1}{\PYZsq{}}\PY{l+s+s1}{I}\PY{l+s+s1}{\PYZsq{}}\PY{p}{)}\PY{p}{,}\PY{p}{:}\PY{p}{]}
\end{Verbatim}
\end{tcolorbox}

            \begin{tcolorbox}[breakable, size=fbox, boxrule=.5pt, pad at break*=1mm, opacityfill=0]
\prompt{Out}{outcolor}{19}{\boxspacing}
\begin{Verbatim}[commandchars=\\\{\}]
Corepaper      4
SEC            2
DSE          NIL
Name: (Bcom, I), dtype: object
\end{Verbatim}
\end{tcolorbox}
        
    \hypertarget{find-the-output}{%
\subsection{find the output?}\label{find-the-output}}

    \begin{tcolorbox}[breakable, size=fbox, boxrule=1pt, pad at break*=1mm,colback=cellbackground, colframe=cellborder]
\prompt{In}{incolor}{20}{\boxspacing}
\begin{Verbatim}[commandchars=\\\{\}]
\PY{n}{DF2}\PY{o}{.}\PY{n}{loc}\PY{p}{[}\PY{p}{(}\PY{l+s+s1}{\PYZsq{}}\PY{l+s+s1}{Bcom}\PY{l+s+s1}{\PYZsq{}}\PY{p}{,}\PY{p}{[}\PY{l+s+s1}{\PYZsq{}}\PY{l+s+s1}{I}\PY{l+s+s1}{\PYZsq{}}\PY{p}{,}\PY{l+s+s1}{\PYZsq{}}\PY{l+s+s1}{II}\PY{l+s+s1}{\PYZsq{}}\PY{p}{]}\PY{p}{)}\PY{p}{,}\PY{p}{:}\PY{p}{]}
\end{Verbatim}
\end{tcolorbox}

            \begin{tcolorbox}[breakable, size=fbox, boxrule=.5pt, pad at break*=1mm, opacityfill=0]
\prompt{Out}{outcolor}{20}{\boxspacing}
\begin{Verbatim}[commandchars=\\\{\}]
                 Corepaper SEC  DSE
Course Semester
Bcom   I                 4   2  NIL
       II                4   2  NIL
\end{Verbatim}
\end{tcolorbox}
        
    \begin{tcolorbox}[breakable, size=fbox, boxrule=1pt, pad at break*=1mm,colback=cellbackground, colframe=cellborder]
\prompt{In}{incolor}{21}{\boxspacing}
\begin{Verbatim}[commandchars=\\\{\}]
\PY{n+nb}{type}\PY{p}{(}\PY{n}{I}\PY{p}{)}
\end{Verbatim}
\end{tcolorbox}

            \begin{tcolorbox}[breakable, size=fbox, boxrule=.5pt, pad at break*=1mm, opacityfill=0]
\prompt{Out}{outcolor}{21}{\boxspacing}
\begin{Verbatim}[commandchars=\\\{\}]
pandas.core.indexes.multi.MultiIndex
\end{Verbatim}
\end{tcolorbox}
        
    \begin{tcolorbox}[breakable, size=fbox, boxrule=1pt, pad at break*=1mm,colback=cellbackground, colframe=cellborder]
\prompt{In}{incolor}{23}{\boxspacing}
\begin{Verbatim}[commandchars=\\\{\}]
\PY{n}{I}\PY{o}{.}\PY{n}{levels}\PY{p}{[}\PY{l+m+mi}{0}\PY{p}{]}\PY{o}{.}\PY{n}{names}
\end{Verbatim}
\end{tcolorbox}

            \begin{tcolorbox}[breakable, size=fbox, boxrule=.5pt, pad at break*=1mm, opacityfill=0]
\prompt{Out}{outcolor}{23}{\boxspacing}
\begin{Verbatim}[commandchars=\\\{\}]
FrozenList(['Course'])
\end{Verbatim}
\end{tcolorbox}
        
    \begin{tcolorbox}[breakable, size=fbox, boxrule=1pt, pad at break*=1mm,colback=cellbackground, colframe=cellborder]
\prompt{In}{incolor}{56}{\boxspacing}
\begin{Verbatim}[commandchars=\\\{\}]
\PY{n}{DF2}
\end{Verbatim}
\end{tcolorbox}

            \begin{tcolorbox}[breakable, size=fbox, boxrule=.5pt, pad at break*=1mm, opacityfill=0]
\prompt{Out}{outcolor}{56}{\boxspacing}
\begin{Verbatim}[commandchars=\\\{\}]
                 corepaper  SEC  DSE
course Semester
PSCS   I                 3  NIL  NIL
       II                3  NIL  NIL
       III               4    1  NIL
       IV                4    1  NIL
       V                 4    1    1
       VI                4    1    1
CSHons I                 4    2  NIL
       II                4    2  NIL
       III               4    2  NIL
       IV                4    2  NIL
       V                 4  NIL    2
       VI                4  NIL    2
Bcom   I                 4    2  NIL
       II                4    2  NIL
       III               4    3    1
       IV                4    3  NIL
lifesc I                 5  NIL    1
       II                5    0    1
\end{Verbatim}
\end{tcolorbox}
        
    \begin{tcolorbox}[breakable, size=fbox, boxrule=1pt, pad at break*=1mm,colback=cellbackground, colframe=cellborder]
\prompt{In}{incolor}{24}{\boxspacing}
\begin{Verbatim}[commandchars=\\\{\}]
\PY{n}{DF2}\PY{o}{.}\PY{n}{values}
\end{Verbatim}
\end{tcolorbox}

            \begin{tcolorbox}[breakable, size=fbox, boxrule=.5pt, pad at break*=1mm, opacityfill=0]
\prompt{Out}{outcolor}{24}{\boxspacing}
\begin{Verbatim}[commandchars=\\\{\}]
array([[3, 'NIL', 'NIL'],
       [3, 'NIL', 'NIL'],
       [4, 1, 'NIL'],
       [4, 1, 'NIL'],
       [4, 1, 1],
       [4, 1, 1],
       [4, 2, 'NIL'],
       [4, 2, 'NIL'],
       [4, 2, 'NIL'],
       [4, 2, 'NIL'],
       [4, 'NIL', 2],
       [4, 'NIL', 2],
       [4, 2, 'NIL'],
       [4, 2, 'NIL'],
       [4, 3, 1],
       [4, 3, 'NIL'],
       [5, 'NIL', 1],
       [5, 0, 1]], dtype=object)
\end{Verbatim}
\end{tcolorbox}
        
    \begin{tcolorbox}[breakable, size=fbox, boxrule=1pt, pad at break*=1mm,colback=cellbackground, colframe=cellborder]
\prompt{In}{incolor}{25}{\boxspacing}
\begin{Verbatim}[commandchars=\\\{\}]
\PY{n}{DF2}\PY{o}{.}\PY{n}{index}\PY{p}{[}\PY{l+m+mi}{0}\PY{p}{]}
\end{Verbatim}
\end{tcolorbox}

            \begin{tcolorbox}[breakable, size=fbox, boxrule=.5pt, pad at break*=1mm, opacityfill=0]
\prompt{Out}{outcolor}{25}{\boxspacing}
\begin{Verbatim}[commandchars=\\\{\}]
('PSCS', 'I')
\end{Verbatim}
\end{tcolorbox}
        
    \hypertarget{dropping-a-particular-level-in-hierarcchy}{%
\section{dropping a particular level in
Hierarcchy}\label{dropping-a-particular-level-in-hierarcchy}}

    \begin{tcolorbox}[breakable, size=fbox, boxrule=1pt, pad at break*=1mm,colback=cellbackground, colframe=cellborder]
\prompt{In}{incolor}{27}{\boxspacing}
\begin{Verbatim}[commandchars=\\\{\}]
\PY{n}{DF2}\PY{o}{.}\PY{n}{droplevel}\PY{p}{(}\PY{l+m+mi}{1}\PY{p}{)} 
\end{Verbatim}
\end{tcolorbox}

            \begin{tcolorbox}[breakable, size=fbox, boxrule=.5pt, pad at break*=1mm, opacityfill=0]
\prompt{Out}{outcolor}{27}{\boxspacing}
\begin{Verbatim}[commandchars=\\\{\}]
        Corepaper  SEC  DSE
Course
PSCS            3  NIL  NIL
PSCS            3  NIL  NIL
PSCS            4    1  NIL
PSCS            4    1  NIL
PSCS            4    1    1
PSCS            4    1    1
CSHons          4    2  NIL
CSHons          4    2  NIL
CSHons          4    2  NIL
CSHons          4    2  NIL
CSHons          4  NIL    2
CSHons          4  NIL    2
Bcom            4    2  NIL
Bcom            4    2  NIL
Bcom            4    3    1
Bcom            4    3  NIL
lifesc          5  NIL    1
lifesc          5    0    1
\end{Verbatim}
\end{tcolorbox}
        
    \hypertarget{unstacking-inner-level}{%
\section{unstacking inner level}\label{unstacking-inner-level}}

    \begin{tcolorbox}[breakable, size=fbox, boxrule=1pt, pad at break*=1mm,colback=cellbackground, colframe=cellborder]
\prompt{In}{incolor}{28}{\boxspacing}
\begin{Verbatim}[commandchars=\\\{\}]
\PY{n}{DF2}\PY{o}{.}\PY{n}{unstack}\PY{p}{(}\PY{p}{)}
\end{Verbatim}
\end{tcolorbox}

            \begin{tcolorbox}[breakable, size=fbox, boxrule=.5pt, pad at break*=1mm, opacityfill=0]
\prompt{Out}{outcolor}{28}{\boxspacing}
\begin{Verbatim}[commandchars=\\\{\}]
         Corepaper                           SEC                           \textbackslash{}
Semester         I   II  III   IV    V   VI    I   II  III   IV    V   VI
Course
Bcom           4.0  4.0  4.0  4.0  NaN  NaN    2    2    3    3  NaN  NaN
CSHons         4.0  4.0  4.0  4.0  4.0  4.0    2    2    2    2  NIL  NIL
PSCS           3.0  3.0  4.0  4.0  4.0  4.0  NIL  NIL    1    1    1    1
lifesc         5.0  5.0  NaN  NaN  NaN  NaN  NIL    0  NaN  NaN  NaN  NaN

          DSE
Semester    I   II  III   IV    V   VI
Course
Bcom      NIL  NIL    1  NIL  NaN  NaN
CSHons    NIL  NIL  NIL  NIL    2    2
PSCS      NIL  NIL  NIL  NIL    1    1
lifesc      1    1  NaN  NaN  NaN  NaN
\end{Verbatim}
\end{tcolorbox}
        
    \begin{tcolorbox}[breakable, size=fbox, boxrule=1pt, pad at break*=1mm,colback=cellbackground, colframe=cellborder]
\prompt{In}{incolor}{29}{\boxspacing}
\begin{Verbatim}[commandchars=\\\{\}]
\PY{n}{DF2}\PY{o}{.}\PY{n}{unstack}\PY{p}{(}\PY{p}{)}\PY{o}{.}\PY{n}{stack}\PY{p}{(}\PY{p}{)}
\end{Verbatim}
\end{tcolorbox}

            \begin{tcolorbox}[breakable, size=fbox, boxrule=.5pt, pad at break*=1mm, opacityfill=0]
\prompt{Out}{outcolor}{29}{\boxspacing}
\begin{Verbatim}[commandchars=\\\{\}]
                 Corepaper  SEC  DSE
Course Semester
Bcom   I               4.0    2  NIL
       II              4.0    2  NIL
       III             4.0    3    1
       IV              4.0    3  NIL
CSHons I               4.0    2  NIL
       II              4.0    2  NIL
       III             4.0    2  NIL
       IV              4.0    2  NIL
       V               4.0  NIL    2
       VI              4.0  NIL    2
PSCS   I               3.0  NIL  NIL
       II              3.0  NIL  NIL
       III             4.0    1  NIL
       IV              4.0    1  NIL
       V               4.0    1    1
       VI              4.0    1    1
lifesc I               5.0  NIL    1
       II              5.0    0    1
\end{Verbatim}
\end{tcolorbox}
        
    \hypertarget{making-two-levels-in-columns}{%
\section{making two levels in
columns}\label{making-two-levels-in-columns}}

    \begin{tcolorbox}[breakable, size=fbox, boxrule=1pt, pad at break*=1mm,colback=cellbackground, colframe=cellborder]
\prompt{In}{incolor}{30}{\boxspacing}
\begin{Verbatim}[commandchars=\\\{\}]
\PY{n}{DF2}\PY{o}{.}\PY{n}{columns}\PY{o}{.}\PY{n}{values}
\end{Verbatim}
\end{tcolorbox}

            \begin{tcolorbox}[breakable, size=fbox, boxrule=.5pt, pad at break*=1mm, opacityfill=0]
\prompt{Out}{outcolor}{30}{\boxspacing}
\begin{Verbatim}[commandchars=\\\{\}]
array(['Corepaper', 'SEC', 'DSE'], dtype=object)
\end{Verbatim}
\end{tcolorbox}
        
    \begin{tcolorbox}[breakable, size=fbox, boxrule=1pt, pad at break*=1mm,colback=cellbackground, colframe=cellborder]
\prompt{In}{incolor}{31}{\boxspacing}
\begin{Verbatim}[commandchars=\\\{\}]
\PY{n}{newcol} \PY{o}{=} \PY{n+nb}{zip}\PY{p}{(}\PY{n}{DF2}\PY{o}{.}\PY{n}{columns}\PY{o}{.}\PY{n}{values}\PY{p}{,}\PY{p}{[}\PY{l+m+mi}{100}\PY{p}{,}\PY{l+m+mi}{50}\PY{p}{,}\PY{l+m+mi}{100}\PY{p}{]}\PY{p}{)}
\PY{n}{newcol}
\end{Verbatim}
\end{tcolorbox}

            \begin{tcolorbox}[breakable, size=fbox, boxrule=.5pt, pad at break*=1mm, opacityfill=0]
\prompt{Out}{outcolor}{31}{\boxspacing}
\begin{Verbatim}[commandchars=\\\{\}]
<zip at 0x20caa4cdd80>
\end{Verbatim}
\end{tcolorbox}
        
    \begin{tcolorbox}[breakable, size=fbox, boxrule=1pt, pad at break*=1mm,colback=cellbackground, colframe=cellborder]
\prompt{In}{incolor}{32}{\boxspacing}
\begin{Verbatim}[commandchars=\\\{\}]
\PY{n}{DF2}\PY{o}{.}\PY{n}{columns} \PY{o}{=} \PY{n}{pd}\PY{o}{.}\PY{n}{MultiIndex}\PY{o}{.}\PY{n}{from\PYZus{}tuples}\PY{p}{(}\PY{n}{newcol}\PY{p}{,} \PY{n}{names}\PY{o}{=}\PY{p}{[}\PY{l+s+s1}{\PYZsq{}}\PY{l+s+s1}{CL1}\PY{l+s+s1}{\PYZsq{}}\PY{p}{,}\PY{l+s+s1}{\PYZsq{}}\PY{l+s+s1}{CL2}\PY{l+s+s1}{\PYZsq{}}\PY{p}{]}\PY{p}{)}
\end{Verbatim}
\end{tcolorbox}

    \begin{tcolorbox}[breakable, size=fbox, boxrule=1pt, pad at break*=1mm,colback=cellbackground, colframe=cellborder]
\prompt{In}{incolor}{33}{\boxspacing}
\begin{Verbatim}[commandchars=\\\{\}]
\PY{n}{DF2}
\end{Verbatim}
\end{tcolorbox}

            \begin{tcolorbox}[breakable, size=fbox, boxrule=.5pt, pad at break*=1mm, opacityfill=0]
\prompt{Out}{outcolor}{33}{\boxspacing}
\begin{Verbatim}[commandchars=\\\{\}]
CL1             Corepaper  SEC  DSE
CL2                   100  50   100
Course Semester
PSCS   I                3  NIL  NIL
       II               3  NIL  NIL
       III              4    1  NIL
       IV               4    1  NIL
       V                4    1    1
       VI               4    1    1
CSHons I                4    2  NIL
       II               4    2  NIL
       III              4    2  NIL
       IV               4    2  NIL
       V                4  NIL    2
       VI               4  NIL    2
Bcom   I                4    2  NIL
       II               4    2  NIL
       III              4    3    1
       IV               4    3  NIL
lifesc I                5  NIL    1
       II               5    0    1
\end{Verbatim}
\end{tcolorbox}
        
    \begin{tcolorbox}[breakable, size=fbox, boxrule=1pt, pad at break*=1mm,colback=cellbackground, colframe=cellborder]
\prompt{In}{incolor}{34}{\boxspacing}
\begin{Verbatim}[commandchars=\\\{\}]
\PY{n}{DF2}\PY{o}{.}\PY{n}{columns}
\end{Verbatim}
\end{tcolorbox}

            \begin{tcolorbox}[breakable, size=fbox, boxrule=.5pt, pad at break*=1mm, opacityfill=0]
\prompt{Out}{outcolor}{34}{\boxspacing}
\begin{Verbatim}[commandchars=\\\{\}]
MultiIndex([('Corepaper', 100),
            (      'SEC',  50),
            (      'DSE', 100)],
           names=['CL1', 'CL2'])
\end{Verbatim}
\end{tcolorbox}
        
    \hypertarget{retreiving-values-of-rowcolum-index}{%
\subsection{retreiving values of row/colum
index}\label{retreiving-values-of-rowcolum-index}}

    \begin{tcolorbox}[breakable, size=fbox, boxrule=1pt, pad at break*=1mm,colback=cellbackground, colframe=cellborder]
\prompt{In}{incolor}{35}{\boxspacing}
\begin{Verbatim}[commandchars=\\\{\}]
\PY{n}{DF2}\PY{o}{.}\PY{n}{index}\PY{o}{.}\PY{n}{get\PYZus{}level\PYZus{}values}\PY{p}{(}\PY{l+m+mi}{0}\PY{p}{)}
\end{Verbatim}
\end{tcolorbox}

            \begin{tcolorbox}[breakable, size=fbox, boxrule=.5pt, pad at break*=1mm, opacityfill=0]
\prompt{Out}{outcolor}{35}{\boxspacing}
\begin{Verbatim}[commandchars=\\\{\}]
Index(['PSCS', 'PSCS', 'PSCS', 'PSCS', 'PSCS', 'PSCS', 'CSHons', 'CSHons',
       'CSHons', 'CSHons', 'CSHons', 'CSHons', 'Bcom', 'Bcom', 'Bcom', 'Bcom',
       'lifesc', 'lifesc'],
      dtype='object', name='Course')
\end{Verbatim}
\end{tcolorbox}
        
    \begin{tcolorbox}[breakable, size=fbox, boxrule=1pt, pad at break*=1mm,colback=cellbackground, colframe=cellborder]
\prompt{In}{incolor}{37}{\boxspacing}
\begin{Verbatim}[commandchars=\\\{\}]
\PY{n}{DF2}\PY{o}{.}\PY{n}{columns}\PY{o}{.}\PY{n}{get\PYZus{}level\PYZus{}values}\PY{p}{(}\PY{l+m+mi}{0}\PY{p}{)}
\end{Verbatim}
\end{tcolorbox}

            \begin{tcolorbox}[breakable, size=fbox, boxrule=.5pt, pad at break*=1mm, opacityfill=0]
\prompt{Out}{outcolor}{37}{\boxspacing}
\begin{Verbatim}[commandchars=\\\{\}]
Index(['Corepaper', 'SEC', 'DSE'], dtype='object', name='CL1')
\end{Verbatim}
\end{tcolorbox}
        
    \hypertarget{accessing-a-particular-column-label}{%
\subsection{accessing a particular column
label}\label{accessing-a-particular-column-label}}

    \begin{tcolorbox}[breakable, size=fbox, boxrule=1pt, pad at break*=1mm,colback=cellbackground, colframe=cellborder]
\prompt{In}{incolor}{38}{\boxspacing}
\begin{Verbatim}[commandchars=\\\{\}]
\PY{n}{DF2}\PY{o}{.}\PY{n}{columns}\PY{o}{.}\PY{n}{levels}
\end{Verbatim}
\end{tcolorbox}

            \begin{tcolorbox}[breakable, size=fbox, boxrule=.5pt, pad at break*=1mm, opacityfill=0]
\prompt{Out}{outcolor}{38}{\boxspacing}
\begin{Verbatim}[commandchars=\\\{\}]
FrozenList([['Corepaper', 'DSE', 'SEC'], [50, 100]])
\end{Verbatim}
\end{tcolorbox}
        
    \begin{tcolorbox}[breakable, size=fbox, boxrule=1pt, pad at break*=1mm,colback=cellbackground, colframe=cellborder]
\prompt{In}{incolor}{40}{\boxspacing}
\begin{Verbatim}[commandchars=\\\{\}]
\PY{n}{DF2}\PY{o}{.}\PY{n}{columns}\PY{o}{.}\PY{n}{levels}\PY{p}{[}\PY{l+m+mi}{1}\PY{p}{]}\PY{p}{[}\PY{l+m+mi}{1}\PY{p}{]}
\end{Verbatim}
\end{tcolorbox}

            \begin{tcolorbox}[breakable, size=fbox, boxrule=.5pt, pad at break*=1mm, opacityfill=0]
\prompt{Out}{outcolor}{40}{\boxspacing}
\begin{Verbatim}[commandchars=\\\{\}]
100
\end{Verbatim}
\end{tcolorbox}
        
    \begin{tcolorbox}[breakable, size=fbox, boxrule=1pt, pad at break*=1mm,colback=cellbackground, colframe=cellborder]
\prompt{In}{incolor}{44}{\boxspacing}
\begin{Verbatim}[commandchars=\\\{\}]
\PY{n+nb}{type}\PY{p}{(}\PY{n}{DF2}\PY{p}{[}\PY{l+s+s1}{\PYZsq{}}\PY{l+s+s1}{SEC}\PY{l+s+s1}{\PYZsq{}}\PY{p}{]}\PY{p}{)}
\end{Verbatim}
\end{tcolorbox}

            \begin{tcolorbox}[breakable, size=fbox, boxrule=.5pt, pad at break*=1mm, opacityfill=0]
\prompt{Out}{outcolor}{44}{\boxspacing}
\begin{Verbatim}[commandchars=\\\{\}]
pandas.core.frame.DataFrame
\end{Verbatim}
\end{tcolorbox}
        
    \begin{tcolorbox}[breakable, size=fbox, boxrule=1pt, pad at break*=1mm,colback=cellbackground, colframe=cellborder]
\prompt{In}{incolor}{46}{\boxspacing}
\begin{Verbatim}[commandchars=\\\{\}]
\PY{n}{DF2}\PY{p}{[}\PY{l+s+s1}{\PYZsq{}}\PY{l+s+s1}{SEC}\PY{l+s+s1}{\PYZsq{}}\PY{p}{]}
\end{Verbatim}
\end{tcolorbox}

            \begin{tcolorbox}[breakable, size=fbox, boxrule=.5pt, pad at break*=1mm, opacityfill=0]
\prompt{Out}{outcolor}{46}{\boxspacing}
\begin{Verbatim}[commandchars=\\\{\}]
CL2               50
Course Semester
PSCS   I         NIL
       II        NIL
       III         1
       IV          1
       V           1
       VI          1
CSHons I           2
       II          2
       III         2
       IV          2
       V         NIL
       VI        NIL
Bcom   I           2
       II          2
       III         3
       IV          3
lifesc I         NIL
       II          0
\end{Verbatim}
\end{tcolorbox}
        
    \begin{tcolorbox}[breakable, size=fbox, boxrule=1pt, pad at break*=1mm,colback=cellbackground, colframe=cellborder]
\prompt{In}{incolor}{45}{\boxspacing}
\begin{Verbatim}[commandchars=\\\{\}]
\PY{n+nb}{type}\PY{p}{(}\PY{n}{DF2}\PY{p}{[}\PY{l+s+s1}{\PYZsq{}}\PY{l+s+s1}{SEC}\PY{l+s+s1}{\PYZsq{}}\PY{p}{]}\PY{p}{[}\PY{l+m+mi}{50}\PY{p}{]}\PY{p}{)}
\end{Verbatim}
\end{tcolorbox}

            \begin{tcolorbox}[breakable, size=fbox, boxrule=.5pt, pad at break*=1mm, opacityfill=0]
\prompt{Out}{outcolor}{45}{\boxspacing}
\begin{Verbatim}[commandchars=\\\{\}]
pandas.core.series.Series
\end{Verbatim}
\end{tcolorbox}
        
    \hypertarget{dropping-a-column-level}{%
\section{dropping a column level}\label{dropping-a-column-level}}

    \begin{tcolorbox}[breakable, size=fbox, boxrule=1pt, pad at break*=1mm,colback=cellbackground, colframe=cellborder]
\prompt{In}{incolor}{47}{\boxspacing}
\begin{Verbatim}[commandchars=\\\{\}]
\PY{n}{DF2}\PY{o}{.}\PY{n}{droplevel}\PY{p}{(}\PY{l+m+mi}{1}\PY{p}{,}\PY{n}{axis}\PY{o}{=}\PY{l+m+mi}{1}\PY{p}{)}
\end{Verbatim}
\end{tcolorbox}

            \begin{tcolorbox}[breakable, size=fbox, boxrule=.5pt, pad at break*=1mm, opacityfill=0]
\prompt{Out}{outcolor}{47}{\boxspacing}
\begin{Verbatim}[commandchars=\\\{\}]
CL1              Corepaper  SEC  DSE
Course Semester
PSCS   I                 3  NIL  NIL
       II                3  NIL  NIL
       III               4    1  NIL
       IV                4    1  NIL
       V                 4    1    1
       VI                4    1    1
CSHons I                 4    2  NIL
       II                4    2  NIL
       III               4    2  NIL
       IV                4    2  NIL
       V                 4  NIL    2
       VI                4  NIL    2
Bcom   I                 4    2  NIL
       II                4    2  NIL
       III               4    3    1
       IV                4    3  NIL
lifesc I                 5  NIL    1
       II                5    0    1
\end{Verbatim}
\end{tcolorbox}
        
    \hypertarget{reordering-and-sorting-levels}{%
\section{Reordering and Sorting
Levels}\label{reordering-and-sorting-levels}}

    \begin{tcolorbox}[breakable, size=fbox, boxrule=1pt, pad at break*=1mm,colback=cellbackground, colframe=cellborder]
\prompt{In}{incolor}{49}{\boxspacing}
\begin{Verbatim}[commandchars=\\\{\}]
\PY{n}{DF2}\PY{o}{.}\PY{n}{swaplevel}\PY{p}{(}\PY{l+s+s1}{\PYZsq{}}\PY{l+s+s1}{Course}\PY{l+s+s1}{\PYZsq{}}\PY{p}{,} \PY{l+s+s1}{\PYZsq{}}\PY{l+s+s1}{Semester}\PY{l+s+s1}{\PYZsq{}}\PY{p}{)}
\end{Verbatim}
\end{tcolorbox}

            \begin{tcolorbox}[breakable, size=fbox, boxrule=.5pt, pad at break*=1mm, opacityfill=0]
\prompt{Out}{outcolor}{49}{\boxspacing}
\begin{Verbatim}[commandchars=\\\{\}]
CL1             Corepaper  SEC  DSE
CL2                   100  50   100
Semester Course
I        PSCS           3  NIL  NIL
II       PSCS           3  NIL  NIL
III      PSCS           4    1  NIL
IV       PSCS           4    1  NIL
V        PSCS           4    1    1
VI       PSCS           4    1    1
I        CSHons         4    2  NIL
II       CSHons         4    2  NIL
III      CSHons         4    2  NIL
IV       CSHons         4    2  NIL
V        CSHons         4  NIL    2
VI       CSHons         4  NIL    2
I        Bcom           4    2  NIL
II       Bcom           4    2  NIL
III      Bcom           4    3    1
IV       Bcom           4    3  NIL
I        lifesc         5  NIL    1
II       lifesc         5    0    1
\end{Verbatim}
\end{tcolorbox}
        
    \begin{tcolorbox}[breakable, size=fbox, boxrule=1pt, pad at break*=1mm,colback=cellbackground, colframe=cellborder]
\prompt{In}{incolor}{50}{\boxspacing}
\begin{Verbatim}[commandchars=\\\{\}]
\PY{n}{DF2}\PY{o}{.}\PY{n}{sort\PYZus{}index}\PY{p}{(}\PY{n}{level}\PY{o}{=}\PY{l+m+mi}{1}\PY{p}{)}
\end{Verbatim}
\end{tcolorbox}

            \begin{tcolorbox}[breakable, size=fbox, boxrule=.5pt, pad at break*=1mm, opacityfill=0]
\prompt{Out}{outcolor}{50}{\boxspacing}
\begin{Verbatim}[commandchars=\\\{\}]
CL1             Corepaper  SEC  DSE
CL2                   100  50   100
Course Semester
Bcom   I                4    2  NIL
CSHons I                4    2  NIL
PSCS   I                3  NIL  NIL
lifesc I                5  NIL    1
Bcom   II               4    2  NIL
CSHons II               4    2  NIL
PSCS   II               3  NIL  NIL
lifesc II               5    0    1
Bcom   III              4    3    1
CSHons III              4    2  NIL
PSCS   III              4    1  NIL
Bcom   IV               4    3  NIL
CSHons IV               4    2  NIL
PSCS   IV               4    1  NIL
CSHons V                4  NIL    2
PSCS   V                4    1    1
CSHons VI               4  NIL    2
PSCS   VI               4    1    1
\end{Verbatim}
\end{tcolorbox}
        
    \begin{tcolorbox}[breakable, size=fbox, boxrule=1pt, pad at break*=1mm,colback=cellbackground, colframe=cellborder]
\prompt{In}{incolor}{78}{\boxspacing}
\begin{Verbatim}[commandchars=\\\{\}]
\PY{n}{DF2}
\end{Verbatim}
\end{tcolorbox}

            \begin{tcolorbox}[breakable, size=fbox, boxrule=.5pt, pad at break*=1mm, opacityfill=0]
\prompt{Out}{outcolor}{78}{\boxspacing}
\begin{Verbatim}[commandchars=\\\{\}]
CL1             corepaper  SEC  DSE
CL2                   100  100  50
course Semester
PSCS   I                3  NIL  NIL
       II               3  NIL  NIL
       III              4    1  NIL
       IV               4    1  NIL
       V                4    1    1
       VI               4    1    1
CSHons I                4    2  NIL
       II               4    2  NIL
       III              4    2  NIL
       IV               4    2  NIL
       V                4  NIL    2
       VI               4  NIL    2
Bcom   I                4    2  NIL
       II               4    2  NIL
       III              4    3    1
       IV               4    3  NIL
lifesc I                5  NIL    1
       II               5    0    1
\end{Verbatim}
\end{tcolorbox}
        
    \hypertarget{sorting-each-row-on-second-column-level}{%
\section{sorting each row on second column
level}\label{sorting-each-row-on-second-column-level}}

    \begin{tcolorbox}[breakable, size=fbox, boxrule=1pt, pad at break*=1mm,colback=cellbackground, colframe=cellborder]
\prompt{In}{incolor}{51}{\boxspacing}
\begin{Verbatim}[commandchars=\\\{\}]
\PY{n}{DF2}\PY{o}{.}\PY{n}{sort\PYZus{}index}\PY{p}{(}\PY{n}{level}\PY{o}{=}\PY{l+m+mi}{1}\PY{p}{,}\PY{n}{axis}\PY{o}{=}\PY{l+m+mi}{1}\PY{p}{)}
\end{Verbatim}
\end{tcolorbox}

            \begin{tcolorbox}[breakable, size=fbox, boxrule=.5pt, pad at break*=1mm, opacityfill=0]
\prompt{Out}{outcolor}{51}{\boxspacing}
\begin{Verbatim}[commandchars=\\\{\}]
CL1              SEC Corepaper  DSE
CL2              50        100  100
Course Semester
PSCS   I         NIL         3  NIL
       II        NIL         3  NIL
       III         1         4  NIL
       IV          1         4  NIL
       V           1         4    1
       VI          1         4    1
CSHons I           2         4  NIL
       II          2         4  NIL
       III         2         4  NIL
       IV          2         4  NIL
       V         NIL         4    2
       VI        NIL         4    2
Bcom   I           2         4  NIL
       II          2         4  NIL
       III         3         4    1
       IV          3         4  NIL
lifesc I         NIL         5    1
       II          0         5    1
\end{Verbatim}
\end{tcolorbox}
        
    \hypertarget{get-summary-statistics-by-level}{%
\section{Get Summary Statistics by
Level}\label{get-summary-statistics-by-level}}

    \begin{tcolorbox}[breakable, size=fbox, boxrule=1pt, pad at break*=1mm,colback=cellbackground, colframe=cellborder]
\prompt{In}{incolor}{52}{\boxspacing}
\begin{Verbatim}[commandchars=\\\{\}]
\PY{n}{DF2}\PY{o}{=}\PY{n}{DF2}\PY{o}{.}\PY{n}{replace}\PY{p}{(}\PY{p}{\PYZob{}}\PY{l+s+s1}{\PYZsq{}}\PY{l+s+s1}{NIL}\PY{l+s+s1}{\PYZsq{}}\PY{p}{:}\PY{k+kc}{None}\PY{p}{\PYZcb{}}\PY{p}{)}
\end{Verbatim}
\end{tcolorbox}

    \hypertarget{query-is-to-find-total-papers-in-each-course}{%
\subsection{query is to find total papers in each
course}\label{query-is-to-find-total-papers-in-each-course}}

    \begin{tcolorbox}[breakable, size=fbox, boxrule=1pt, pad at break*=1mm,colback=cellbackground, colframe=cellborder]
\prompt{In}{incolor}{53}{\boxspacing}
\begin{Verbatim}[commandchars=\\\{\}]
\PY{n}{DF2}
\end{Verbatim}
\end{tcolorbox}

            \begin{tcolorbox}[breakable, size=fbox, boxrule=.5pt, pad at break*=1mm, opacityfill=0]
\prompt{Out}{outcolor}{53}{\boxspacing}
\begin{Verbatim}[commandchars=\\\{\}]
CL1             Corepaper  SEC  DSE
CL2                   100  50   100
Course Semester
PSCS   I                3  NaN  NaN
       II               3  NaN  NaN
       III              4  1.0  NaN
       IV               4  1.0  NaN
       V                4  1.0  1.0
       VI               4  1.0  1.0
CSHons I                4  2.0  NaN
       II               4  2.0  NaN
       III              4  2.0  NaN
       IV               4  2.0  NaN
       V                4  NaN  2.0
       VI               4  NaN  2.0
Bcom   I                4  2.0  NaN
       II               4  2.0  NaN
       III              4  3.0  1.0
       IV               4  3.0  NaN
lifesc I                5  NaN  1.0
       II               5  0.0  1.0
\end{Verbatim}
\end{tcolorbox}
        
    \begin{tcolorbox}[breakable, size=fbox, boxrule=1pt, pad at break*=1mm,colback=cellbackground, colframe=cellborder]
\prompt{In}{incolor}{56}{\boxspacing}
\begin{Verbatim}[commandchars=\\\{\}]
\PY{n}{DF2}\PY{o}{.}\PY{n}{sum}\PY{p}{(}\PY{n}{level}\PY{o}{=}\PY{l+s+s1}{\PYZsq{}}\PY{l+s+s1}{Course}\PY{l+s+s1}{\PYZsq{}}\PY{p}{,}\PY{n}{skipna}\PY{o}{=}\PY{k+kc}{True}\PY{p}{)}\PY{o}{.}\PY{n}{astype}\PY{p}{(}\PY{n+nb}{int}\PY{p}{)}
\end{Verbatim}
\end{tcolorbox}

            \begin{tcolorbox}[breakable, size=fbox, boxrule=.5pt, pad at break*=1mm, opacityfill=0]
\prompt{Out}{outcolor}{56}{\boxspacing}
\begin{Verbatim}[commandchars=\\\{\}]
CL1    Corepaper SEC DSE
CL2          100 50  100
Course
PSCS          22   4   2
CSHons        24   8   4
Bcom          16  10   1
lifesc        10   0   2
\end{Verbatim}
\end{tcolorbox}
        
    \hypertarget{find-total-papers-of-marks-100-and-50-each}{%
\subsection{find total papers of marks 100 and 50
each}\label{find-total-papers-of-marks-100-and-50-each}}

    \begin{tcolorbox}[breakable, size=fbox, boxrule=1pt, pad at break*=1mm,colback=cellbackground, colframe=cellborder]
\prompt{In}{incolor}{59}{\boxspacing}
\begin{Verbatim}[commandchars=\\\{\}]
\PY{n}{DF2}\PY{o}{.}\PY{n}{sum}\PY{p}{(}\PY{n}{level}\PY{o}{=}\PY{l+m+mi}{0}\PY{p}{,} \PY{n}{axis}\PY{o}{=}\PY{l+m+mi}{1}\PY{p}{,}\PY{n}{skipna}\PY{o}{=}\PY{k+kc}{True}\PY{p}{)}
\end{Verbatim}
\end{tcolorbox}

            \begin{tcolorbox}[breakable, size=fbox, boxrule=.5pt, pad at break*=1mm, opacityfill=0]
\prompt{Out}{outcolor}{59}{\boxspacing}
\begin{Verbatim}[commandchars=\\\{\}]
CL1              Corepaper  SEC  DSE
Course Semester
PSCS   I               3.0  0.0  0.0
       II              3.0  0.0  0.0
       III             4.0  1.0  0.0
       IV              4.0  1.0  0.0
       V               4.0  1.0  1.0
       VI              4.0  1.0  1.0
CSHons I               4.0  2.0  0.0
       II              4.0  2.0  0.0
       III             4.0  2.0  0.0
       IV              4.0  2.0  0.0
       V               4.0  0.0  2.0
       VI              4.0  0.0  2.0
Bcom   I               4.0  2.0  0.0
       II              4.0  2.0  0.0
       III             4.0  3.0  1.0
       IV              4.0  3.0  0.0
lifesc I               5.0  0.0  1.0
       II              5.0  0.0  1.0
\end{Verbatim}
\end{tcolorbox}
        
    \hypertarget{combining-and-merging-dataframes-merge-join-concat}{%
\section{Combining and merging dataframes merge() join()
concat()}\label{combining-and-merging-dataframes-merge-join-concat}}

\begin{itemize}
\tightlist
\item
  creating DF 1 and 2
\end{itemize}

    \begin{tcolorbox}[breakable, size=fbox, boxrule=1pt, pad at break*=1mm,colback=cellbackground, colframe=cellborder]
\prompt{In}{incolor}{7}{\boxspacing}
\begin{Verbatim}[commandchars=\\\{\}]
\PY{n}{df1} \PY{o}{=} \PY{n}{pd}\PY{o}{.}\PY{n}{DataFrame}\PY{p}{(}\PY{p}{\PYZob{}}\PY{l+s+s1}{\PYZsq{}}\PY{l+s+s1}{key}\PY{l+s+s1}{\PYZsq{}}\PY{p}{:} \PY{p}{[}\PY{l+s+s1}{\PYZsq{}}\PY{l+s+s1}{b}\PY{l+s+s1}{\PYZsq{}}\PY{p}{,} \PY{l+s+s1}{\PYZsq{}}\PY{l+s+s1}{b}\PY{l+s+s1}{\PYZsq{}}\PY{p}{,} \PY{l+s+s1}{\PYZsq{}}\PY{l+s+s1}{a}\PY{l+s+s1}{\PYZsq{}}\PY{p}{,} \PY{l+s+s1}{\PYZsq{}}\PY{l+s+s1}{c}\PY{l+s+s1}{\PYZsq{}}\PY{p}{,} \PY{l+s+s1}{\PYZsq{}}\PY{l+s+s1}{a}\PY{l+s+s1}{\PYZsq{}}\PY{p}{,} \PY{l+s+s1}{\PYZsq{}}\PY{l+s+s1}{a}\PY{l+s+s1}{\PYZsq{}}\PY{p}{,} \PY{l+s+s1}{\PYZsq{}}\PY{l+s+s1}{b}\PY{l+s+s1}{\PYZsq{}}\PY{p}{]}\PY{p}{,}
                    \PY{l+s+s1}{\PYZsq{}}\PY{l+s+s1}{data1}\PY{l+s+s1}{\PYZsq{}}\PY{p}{:} \PY{n+nb}{range}\PY{p}{(}\PY{l+m+mi}{7}\PY{p}{)}\PY{p}{\PYZcb{}}\PY{p}{)}
\PY{n}{df2} \PY{o}{=} \PY{n}{pd}\PY{o}{.}\PY{n}{DataFrame}\PY{p}{(}\PY{p}{\PYZob{}}\PY{l+s+s1}{\PYZsq{}}\PY{l+s+s1}{key}\PY{l+s+s1}{\PYZsq{}}\PY{p}{:} \PY{p}{[}\PY{l+s+s1}{\PYZsq{}}\PY{l+s+s1}{a}\PY{l+s+s1}{\PYZsq{}}\PY{p}{,} \PY{l+s+s1}{\PYZsq{}}\PY{l+s+s1}{b}\PY{l+s+s1}{\PYZsq{}}\PY{p}{,} \PY{l+s+s1}{\PYZsq{}}\PY{l+s+s1}{d}\PY{l+s+s1}{\PYZsq{}}\PY{p}{]}\PY{p}{,}
                    \PY{l+s+s1}{\PYZsq{}}\PY{l+s+s1}{data2}\PY{l+s+s1}{\PYZsq{}}\PY{p}{:} \PY{n+nb}{range}\PY{p}{(}\PY{l+m+mi}{3}\PY{p}{)}\PY{p}{\PYZcb{}}\PY{p}{)}
\end{Verbatim}
\end{tcolorbox}

    \begin{tcolorbox}[breakable, size=fbox, boxrule=1pt, pad at break*=1mm,colback=cellbackground, colframe=cellborder]
\prompt{In}{incolor}{8}{\boxspacing}
\begin{Verbatim}[commandchars=\\\{\}]
\PY{n}{df1}
\end{Verbatim}
\end{tcolorbox}

            \begin{tcolorbox}[breakable, size=fbox, boxrule=.5pt, pad at break*=1mm, opacityfill=0]
\prompt{Out}{outcolor}{8}{\boxspacing}
\begin{Verbatim}[commandchars=\\\{\}]
  key  data1
0   b      0
1   b      1
2   a      2
3   c      3
4   a      4
5   a      5
6   b      6
\end{Verbatim}
\end{tcolorbox}
        
    \begin{tcolorbox}[breakable, size=fbox, boxrule=1pt, pad at break*=1mm,colback=cellbackground, colframe=cellborder]
\prompt{In}{incolor}{9}{\boxspacing}
\begin{Verbatim}[commandchars=\\\{\}]
\PY{n}{df2}
\end{Verbatim}
\end{tcolorbox}

            \begin{tcolorbox}[breakable, size=fbox, boxrule=.5pt, pad at break*=1mm, opacityfill=0]
\prompt{Out}{outcolor}{9}{\boxspacing}
\begin{Verbatim}[commandchars=\\\{\}]
  key  data2
0   a      0
1   b      1
2   d      2
\end{Verbatim}
\end{tcolorbox}
        
    \hypertarget{combining-merge-dataframe-or-named-series-objects-with-a-database-style-join.-on-common-attributes}{%
\section{Combining: Merge DataFrame or named Series objects with a
database-style join. on common
attributes}\label{combining-merge-dataframe-or-named-series-objects-with-a-database-style-join.-on-common-attributes}}

how : \{`left', `right', `outer', `inner'\}, default `inner'

    \hypertarget{merge-by-default-inner-joins-on-all-common-columns-with-same-values}{%
\subsection{merge() by default inner joins on all common columns with
same
values}\label{merge-by-default-inner-joins-on-all-common-columns-with-same-values}}

    \begin{tcolorbox}[breakable, size=fbox, boxrule=1pt, pad at break*=1mm,colback=cellbackground, colframe=cellborder]
\prompt{In}{incolor}{13}{\boxspacing}
\begin{Verbatim}[commandchars=\\\{\}]
\PY{n}{pd}\PY{o}{.}\PY{n}{merge}\PY{p}{(}\PY{n}{df1}\PY{p}{,} \PY{n}{df2}\PY{p}{)}
\end{Verbatim}
\end{tcolorbox}

            \begin{tcolorbox}[breakable, size=fbox, boxrule=.5pt, pad at break*=1mm, opacityfill=0]
\prompt{Out}{outcolor}{13}{\boxspacing}
\begin{Verbatim}[commandchars=\\\{\}]
  key  data1  data2
0   b      0      1
1   b      1      1
2   b      6      1
3   a      2      0
4   a      4      0
5   a      5      0
\end{Verbatim}
\end{tcolorbox}
        
    \begin{tcolorbox}[breakable, size=fbox, boxrule=1pt, pad at break*=1mm,colback=cellbackground, colframe=cellborder]
\prompt{In}{incolor}{64}{\boxspacing}
\begin{Verbatim}[commandchars=\\\{\}]
\PY{n}{pd}\PY{o}{.}\PY{n}{merge}\PY{p}{(}\PY{n}{df1}\PY{p}{,} \PY{n}{df2}\PY{p}{)}\PY{o}{.}\PY{n}{sort\PYZus{}values}\PY{p}{(}\PY{l+s+s1}{\PYZsq{}}\PY{l+s+s1}{key}\PY{l+s+s1}{\PYZsq{}}\PY{p}{)}
\end{Verbatim}
\end{tcolorbox}

            \begin{tcolorbox}[breakable, size=fbox, boxrule=.5pt, pad at break*=1mm, opacityfill=0]
\prompt{Out}{outcolor}{64}{\boxspacing}
\begin{Verbatim}[commandchars=\\\{\}]
  key  data1  data2
3   a      2      0
4   a      4      0
5   a      5      0
0   b      0      1
1   b      1      1
2   b      6      1
\end{Verbatim}
\end{tcolorbox}
        
    \begin{tcolorbox}[breakable, size=fbox, boxrule=1pt, pad at break*=1mm,colback=cellbackground, colframe=cellborder]
\prompt{In}{incolor}{65}{\boxspacing}
\begin{Verbatim}[commandchars=\\\{\}]
\PY{n}{pd}\PY{o}{.}\PY{n}{merge}\PY{p}{(}\PY{n}{df1}\PY{p}{,}\PY{n}{df2}\PY{p}{,}\PY{n}{how}\PY{o}{=}\PY{l+s+s1}{\PYZsq{}}\PY{l+s+s1}{outer}\PY{l+s+s1}{\PYZsq{}}\PY{p}{)}
\end{Verbatim}
\end{tcolorbox}

            \begin{tcolorbox}[breakable, size=fbox, boxrule=.5pt, pad at break*=1mm, opacityfill=0]
\prompt{Out}{outcolor}{65}{\boxspacing}
\begin{Verbatim}[commandchars=\\\{\}]
  key  data1  data2
0   b    0.0    1.0
1   b    1.0    1.0
2   b    6.0    1.0
3   a    2.0    0.0
4   a    4.0    0.0
5   a    5.0    0.0
6   c    3.0    NaN
7   d    NaN    2.0
\end{Verbatim}
\end{tcolorbox}
        
    \hypertarget{adding-more-columns-to-dataframe}{%
\subsection{adding more columns to
dataframe}\label{adding-more-columns-to-dataframe}}

    \begin{tcolorbox}[breakable, size=fbox, boxrule=1pt, pad at break*=1mm,colback=cellbackground, colframe=cellborder]
\prompt{In}{incolor}{11}{\boxspacing}
\begin{Verbatim}[commandchars=\\\{\}]
\PY{n}{df1}\PY{p}{[}\PY{l+s+s1}{\PYZsq{}}\PY{l+s+s1}{new}\PY{l+s+s1}{\PYZsq{}}\PY{p}{]}\PY{o}{=}\PY{n}{np}\PY{o}{.}\PY{n}{arange}\PY{p}{(}\PY{n+nb}{len}\PY{p}{(}\PY{n}{df1}\PY{p}{)}\PY{p}{)}
\end{Verbatim}
\end{tcolorbox}

    \begin{tcolorbox}[breakable, size=fbox, boxrule=1pt, pad at break*=1mm,colback=cellbackground, colframe=cellborder]
\prompt{In}{incolor}{12}{\boxspacing}
\begin{Verbatim}[commandchars=\\\{\}]
\PY{n}{df2}\PY{p}{[}\PY{l+s+s1}{\PYZsq{}}\PY{l+s+s1}{new}\PY{l+s+s1}{\PYZsq{}}\PY{p}{]}\PY{o}{=}\PY{n}{np}\PY{o}{.}\PY{n}{arange}\PY{p}{(}\PY{n+nb}{len}\PY{p}{(}\PY{n}{df2}\PY{p}{)}\PY{p}{)}\PY{o}{+}\PY{l+m+mi}{1}
\end{Verbatim}
\end{tcolorbox}

    \begin{tcolorbox}[breakable, size=fbox, boxrule=1pt, pad at break*=1mm,colback=cellbackground, colframe=cellborder]
\prompt{In}{incolor}{6}{\boxspacing}
\begin{Verbatim}[commandchars=\\\{\}]
\PY{n}{df1}
\end{Verbatim}
\end{tcolorbox}

            \begin{tcolorbox}[breakable, size=fbox, boxrule=.5pt, pad at break*=1mm, opacityfill=0]
\prompt{Out}{outcolor}{6}{\boxspacing}
\begin{Verbatim}[commandchars=\\\{\}]
  key  data1  new
0   b      0    0
1   b      1    1
2   a      2    2
3   c      3    3
4   a      4    4
5   a      5    5
6   b      6    6
\end{Verbatim}
\end{tcolorbox}
        
    \begin{tcolorbox}[breakable, size=fbox, boxrule=1pt, pad at break*=1mm,colback=cellbackground, colframe=cellborder]
\prompt{In}{incolor}{8}{\boxspacing}
\begin{Verbatim}[commandchars=\\\{\}]
\PY{n}{df2}
\end{Verbatim}
\end{tcolorbox}

            \begin{tcolorbox}[breakable, size=fbox, boxrule=.5pt, pad at break*=1mm, opacityfill=0]
\prompt{Out}{outcolor}{8}{\boxspacing}
\begin{Verbatim}[commandchars=\\\{\}]
  key  data2  new
0   a      0    1
1   b      1    2
2   d      2    3
\end{Verbatim}
\end{tcolorbox}
        
    \hypertarget{find-the-output}{%
\subsection{Find the output?}\label{find-the-output}}

    \begin{tcolorbox}[breakable, size=fbox, boxrule=1pt, pad at break*=1mm,colback=cellbackground, colframe=cellborder]
\prompt{In}{incolor}{70}{\boxspacing}
\begin{Verbatim}[commandchars=\\\{\}]
\PY{n}{pd}\PY{o}{.}\PY{n}{merge}\PY{p}{(}\PY{n}{df1}\PY{p}{,} \PY{n}{df2}\PY{p}{)}
\end{Verbatim}
\end{tcolorbox}

            \begin{tcolorbox}[breakable, size=fbox, boxrule=.5pt, pad at break*=1mm, opacityfill=0]
\prompt{Out}{outcolor}{70}{\boxspacing}
\begin{Verbatim}[commandchars=\\\{\}]
  key  data1  new  data2
0   b      1    1      1
\end{Verbatim}
\end{tcolorbox}
        
    \hypertarget{by-deafult-joining-is-on-all-common-attributes-on-bot-df.-for-joining-on-the-specified-column-as-using-on-other-common-attributes-are-renamed-as-att_x-to-differentiate}{%
\subsection{By deafult, joining is on all common attributes on bot DF.
for joining on the specified column as using `on', other common
attributes are renamed as att\_x to
differentiate}\label{by-deafult-joining-is-on-all-common-attributes-on-bot-df.-for-joining-on-the-specified-column-as-using-on-other-common-attributes-are-renamed-as-att_x-to-differentiate}}

    \begin{tcolorbox}[breakable, size=fbox, boxrule=1pt, pad at break*=1mm,colback=cellbackground, colframe=cellborder]
\prompt{In}{incolor}{18}{\boxspacing}
\begin{Verbatim}[commandchars=\\\{\}]
\PY{n}{pd}\PY{o}{.}\PY{n}{merge}\PY{p}{(}\PY{n}{df1}\PY{p}{,} \PY{n}{df2}\PY{p}{,}\PY{n}{on}\PY{o}{=}\PY{p}{[}\PY{l+s+s1}{\PYZsq{}}\PY{l+s+s1}{key}\PY{l+s+s1}{\PYZsq{}}\PY{p}{]}\PY{p}{)}
\end{Verbatim}
\end{tcolorbox}

            \begin{tcolorbox}[breakable, size=fbox, boxrule=.5pt, pad at break*=1mm, opacityfill=0]
\prompt{Out}{outcolor}{18}{\boxspacing}
\begin{Verbatim}[commandchars=\\\{\}]
  key  data1  new\_x  data2  new\_y
0   b      0      0      1      1
1   b      1      1      1      1
2   b      6      6      1      1
3   a      2      2      0      0
4   a      4      4      0      0
5   a      5      5      0      0
\end{Verbatim}
\end{tcolorbox}
        
    \begin{tcolorbox}[breakable, size=fbox, boxrule=1pt, pad at break*=1mm,colback=cellbackground, colframe=cellborder]
\prompt{In}{incolor}{72}{\boxspacing}
\begin{Verbatim}[commandchars=\\\{\}]
\PY{n}{pd}\PY{o}{.}\PY{n}{merge}\PY{p}{(}\PY{n}{df1}\PY{p}{,} \PY{n}{df2}\PY{p}{,}\PY{n}{on}\PY{o}{=}\PY{p}{[}\PY{l+s+s1}{\PYZsq{}}\PY{l+s+s1}{key}\PY{l+s+s1}{\PYZsq{}}\PY{p}{,}\PY{l+s+s1}{\PYZsq{}}\PY{l+s+s1}{new}\PY{l+s+s1}{\PYZsq{}}\PY{p}{]}\PY{p}{)}
\end{Verbatim}
\end{tcolorbox}

            \begin{tcolorbox}[breakable, size=fbox, boxrule=.5pt, pad at break*=1mm, opacityfill=0]
\prompt{Out}{outcolor}{72}{\boxspacing}
\begin{Verbatim}[commandchars=\\\{\}]
  key  data1  new  data2
0   b      1    1      1
\end{Verbatim}
\end{tcolorbox}
        
    \hypertarget{joining-on-mentioned-attribute-and-if-any-other-common-attribute-then-suffice-it-with-user-specified-name-as-per-apperance}{%
\subsection{joining on mentioned attribute and if any other common
attribute then suffice it with user-specified name as per
apperance}\label{joining-on-mentioned-attribute-and-if-any-other-common-attribute-then-suffice-it-with-user-specified-name-as-per-apperance}}

    \begin{tcolorbox}[breakable, size=fbox, boxrule=1pt, pad at break*=1mm,colback=cellbackground, colframe=cellborder]
\prompt{In}{incolor}{19}{\boxspacing}
\begin{Verbatim}[commandchars=\\\{\}]
\PY{n}{pd}\PY{o}{.}\PY{n}{merge}\PY{p}{(}\PY{n}{df1}\PY{p}{,} \PY{n}{df2}\PY{p}{,} \PY{n}{on}\PY{o}{=}\PY{l+s+s1}{\PYZsq{}}\PY{l+s+s1}{key}\PY{l+s+s1}{\PYZsq{}}\PY{p}{,} \PY{n}{suffixes}\PY{o}{=}\PY{p}{(}\PY{l+s+s1}{\PYZsq{}}\PY{l+s+s1}{\PYZus{}df1}\PY{l+s+s1}{\PYZsq{}}\PY{p}{,} \PY{l+s+s1}{\PYZsq{}}\PY{l+s+s1}{\PYZus{}df2}\PY{l+s+s1}{\PYZsq{}}\PY{p}{)}\PY{p}{)}
\end{Verbatim}
\end{tcolorbox}

            \begin{tcolorbox}[breakable, size=fbox, boxrule=.5pt, pad at break*=1mm, opacityfill=0]
\prompt{Out}{outcolor}{19}{\boxspacing}
\begin{Verbatim}[commandchars=\\\{\}]
  key  data1  new\_df1  data2  new\_df2
0   b      0        0      1        1
1   b      1        1      1        1
2   b      6        6      1        1
3   a      2        2      0        0
4   a      4        4      0        0
5   a      5        5      0        0
\end{Verbatim}
\end{tcolorbox}
        
    \hypertarget{joining-over-two-different-attributes-in-two-data-frames}{%
\subsection{joining over two different attributes in two data
frames}\label{joining-over-two-different-attributes-in-two-data-frames}}

    \begin{tcolorbox}[breakable, size=fbox, boxrule=1pt, pad at break*=1mm,colback=cellbackground, colframe=cellborder]
\prompt{In}{incolor}{74}{\boxspacing}
\begin{Verbatim}[commandchars=\\\{\}]
\PY{n}{pd}\PY{o}{.}\PY{n}{merge}\PY{p}{(}\PY{n}{df1}\PY{p}{,} \PY{n}{df2}\PY{p}{,} \PY{n}{left\PYZus{}on}\PY{o}{=}\PY{l+s+s1}{\PYZsq{}}\PY{l+s+s1}{data1}\PY{l+s+s1}{\PYZsq{}}\PY{p}{,} \PY{n}{right\PYZus{}on}\PY{o}{=}\PY{l+s+s1}{\PYZsq{}}\PY{l+s+s1}{data2}\PY{l+s+s1}{\PYZsq{}}\PY{p}{)}
\end{Verbatim}
\end{tcolorbox}

            \begin{tcolorbox}[breakable, size=fbox, boxrule=.5pt, pad at break*=1mm, opacityfill=0]
\prompt{Out}{outcolor}{74}{\boxspacing}
\begin{Verbatim}[commandchars=\\\{\}]
  key\_x  data1  new\_x key\_y  data2  new\_y
0     b      0      0     a      0      0
1     b      1      1     b      1      1
2     a      2      2     d      2      2
\end{Verbatim}
\end{tcolorbox}
        
    \begin{tcolorbox}[breakable, size=fbox, boxrule=1pt, pad at break*=1mm,colback=cellbackground, colframe=cellborder]
\prompt{In}{incolor}{75}{\boxspacing}
\begin{Verbatim}[commandchars=\\\{\}]
\PY{n}{df1}
\end{Verbatim}
\end{tcolorbox}

            \begin{tcolorbox}[breakable, size=fbox, boxrule=.5pt, pad at break*=1mm, opacityfill=0]
\prompt{Out}{outcolor}{75}{\boxspacing}
\begin{Verbatim}[commandchars=\\\{\}]
  key  data1  new
0   b      0    0
1   b      1    1
2   a      2    2
3   c      3    3
4   a      4    4
5   a      5    5
6   b      6    6
\end{Verbatim}
\end{tcolorbox}
        
    \begin{tcolorbox}[breakable, size=fbox, boxrule=1pt, pad at break*=1mm,colback=cellbackground, colframe=cellborder]
\prompt{In}{incolor}{87}{\boxspacing}
\begin{Verbatim}[commandchars=\\\{\}]
\PY{n}{df2}
\end{Verbatim}
\end{tcolorbox}

            \begin{tcolorbox}[breakable, size=fbox, boxrule=.5pt, pad at break*=1mm, opacityfill=0]
\prompt{Out}{outcolor}{87}{\boxspacing}
\begin{Verbatim}[commandchars=\\\{\}]
  key  data2  new
0   a      0    0
1   b      1    1
2   d     10    2
\end{Verbatim}
\end{tcolorbox}
        
    \begin{tcolorbox}[breakable, size=fbox, boxrule=1pt, pad at break*=1mm,colback=cellbackground, colframe=cellborder]
\prompt{In}{incolor}{84}{\boxspacing}
\begin{Verbatim}[commandchars=\\\{\}]
\PY{n}{df2}\PY{o}{.}\PY{n}{iloc}\PY{p}{[}\PY{l+m+mi}{2}\PY{p}{,}\PY{l+m+mi}{1}\PY{p}{]}\PY{o}{=}\PY{l+m+mi}{10}
\end{Verbatim}
\end{tcolorbox}

    \hypertarget{merging-on-index}{%
\section{merging on index}\label{merging-on-index}}

(index of one DF and column of another DF)

    \begin{tcolorbox}[breakable, size=fbox, boxrule=1pt, pad at break*=1mm,colback=cellbackground, colframe=cellborder]
\prompt{In}{incolor}{20}{\boxspacing}
\begin{Verbatim}[commandchars=\\\{\}]
\PY{n}{pd}\PY{o}{.}\PY{n}{merge}\PY{p}{(}\PY{n}{df1}\PY{p}{,} \PY{n}{df2}\PY{p}{,} \PY{n}{left\PYZus{}on}\PY{o}{=}\PY{l+s+s1}{\PYZsq{}}\PY{l+s+s1}{data1}\PY{l+s+s1}{\PYZsq{}}\PY{p}{,} \PY{n}{right\PYZus{}index}\PY{o}{=}\PY{k+kc}{True}\PY{p}{)}
\end{Verbatim}
\end{tcolorbox}

            \begin{tcolorbox}[breakable, size=fbox, boxrule=.5pt, pad at break*=1mm, opacityfill=0]
\prompt{Out}{outcolor}{20}{\boxspacing}
\begin{Verbatim}[commandchars=\\\{\}]
  key\_x  data1  new\_x key\_y  data2  new\_y
0     b      0      0     a      0      0
1     b      1      1     b      1      1
2     a      2      2     d      2      2
\end{Verbatim}
\end{tcolorbox}
        
    \begin{tcolorbox}[breakable, size=fbox, boxrule=1pt, pad at break*=1mm,colback=cellbackground, colframe=cellborder]
\prompt{In}{incolor}{77}{\boxspacing}
\begin{Verbatim}[commandchars=\\\{\}]
\PY{n}{DFnew}\PY{o}{=}\PY{n}{pd}\PY{o}{.}\PY{n}{merge}\PY{p}{(}\PY{n}{df1}\PY{p}{,} \PY{n}{df2}\PY{p}{,} \PY{n}{left\PYZus{}on}\PY{o}{=}\PY{l+s+s1}{\PYZsq{}}\PY{l+s+s1}{data1}\PY{l+s+s1}{\PYZsq{}}\PY{p}{,} \PY{n}{right\PYZus{}index}\PY{o}{=}\PY{k+kc}{True}\PY{p}{)}
\end{Verbatim}
\end{tcolorbox}

    \begin{tcolorbox}[breakable, size=fbox, boxrule=1pt, pad at break*=1mm,colback=cellbackground, colframe=cellborder]
\prompt{In}{incolor}{80}{\boxspacing}
\begin{Verbatim}[commandchars=\\\{\}]
\PY{n}{DFnew}
\end{Verbatim}
\end{tcolorbox}

            \begin{tcolorbox}[breakable, size=fbox, boxrule=.5pt, pad at break*=1mm, opacityfill=0]
\prompt{Out}{outcolor}{80}{\boxspacing}
\begin{Verbatim}[commandchars=\\\{\}]
  key\_x  new\_x key\_y  data2  new\_y
0     b      0     a      0      0
1     b      1     b      1      1
2     a      2     d      2      2
\end{Verbatim}
\end{tcolorbox}
        
    \begin{tcolorbox}[breakable, size=fbox, boxrule=1pt, pad at break*=1mm,colback=cellbackground, colframe=cellborder]
\prompt{In}{incolor}{ }{\boxspacing}
\begin{Verbatim}[commandchars=\\\{\}]
\PY{n+nb}{type}\PY{p}{(}\PY{n}{DFnew}\PY{p}{)}
\end{Verbatim}
\end{tcolorbox}

            \begin{tcolorbox}[breakable, size=fbox, boxrule=.5pt, pad at break*=1mm, opacityfill=0]
\prompt{Out}{outcolor}{ }{\boxspacing}
\begin{Verbatim}[commandchars=\\\{\}]
pandas.core.frame.DataFrame
\end{Verbatim}
\end{tcolorbox}
        
    \begin{tcolorbox}[breakable, size=fbox, boxrule=1pt, pad at break*=1mm,colback=cellbackground, colframe=cellborder]
\prompt{In}{incolor}{79}{\boxspacing}
\begin{Verbatim}[commandchars=\\\{\}]
\PY{n}{DFnew}\PY{o}{.}\PY{n}{drop}\PY{p}{(}\PY{l+s+s1}{\PYZsq{}}\PY{l+s+s1}{data1}\PY{l+s+s1}{\PYZsq{}}\PY{p}{,}\PY{n}{axis}\PY{o}{=}\PY{l+m+mi}{1}\PY{p}{,}\PY{n}{inplace}\PY{o}{=}\PY{k+kc}{True}\PY{p}{)}
\end{Verbatim}
\end{tcolorbox}

    \begin{tcolorbox}[breakable, size=fbox, boxrule=1pt, pad at break*=1mm,colback=cellbackground, colframe=cellborder]
\prompt{In}{incolor}{86}{\boxspacing}
\begin{Verbatim}[commandchars=\\\{\}]
\PY{n}{df1}
\end{Verbatim}
\end{tcolorbox}

            \begin{tcolorbox}[breakable, size=fbox, boxrule=.5pt, pad at break*=1mm, opacityfill=0]
\prompt{Out}{outcolor}{86}{\boxspacing}
\begin{Verbatim}[commandchars=\\\{\}]
  key  data1  new
0   b      0    0
1   b      1    1
2   a      2    2
3   c      3    3
4   a      4    4
5   a      5    5
6   b      6    6
\end{Verbatim}
\end{tcolorbox}
        
    \hypertarget{find-the-output}{%
\subsection{find the output?}\label{find-the-output}}

    \begin{tcolorbox}[breakable, size=fbox, boxrule=1pt, pad at break*=1mm,colback=cellbackground, colframe=cellborder]
\prompt{In}{incolor}{21}{\boxspacing}
\begin{Verbatim}[commandchars=\\\{\}]
\PY{n}{pd}\PY{o}{.}\PY{n}{merge}\PY{p}{(}\PY{n}{df1}\PY{p}{,} \PY{n}{df2}\PY{p}{,} \PY{n}{right\PYZus{}on}\PY{o}{=}\PY{l+s+s1}{\PYZsq{}}\PY{l+s+s1}{data2}\PY{l+s+s1}{\PYZsq{}}\PY{p}{,} \PY{n}{left\PYZus{}index}\PY{o}{=}\PY{k+kc}{True}\PY{p}{,} \PY{n}{how}\PY{o}{=}\PY{l+s+s1}{\PYZsq{}}\PY{l+s+s1}{outer}\PY{l+s+s1}{\PYZsq{}}\PY{p}{)}
\end{Verbatim}
\end{tcolorbox}

            \begin{tcolorbox}[breakable, size=fbox, boxrule=.5pt, pad at break*=1mm, opacityfill=0]
\prompt{Out}{outcolor}{21}{\boxspacing}
\begin{Verbatim}[commandchars=\\\{\}]
    key\_x  data1  new\_x key\_y  data2  new\_y
0.0     b      0      0     a      0    0.0
1.0     b      1      1     b      1    1.0
2.0     a      2      2     d      2    2.0
NaN     c      3      3   NaN      3    NaN
NaN     a      4      4   NaN      4    NaN
NaN     a      5      5   NaN      5    NaN
NaN     b      6      6   NaN      6    NaN
\end{Verbatim}
\end{tcolorbox}
        
    \begin{tcolorbox}[breakable, size=fbox, boxrule=1pt, pad at break*=1mm,colback=cellbackground, colframe=cellborder]
\prompt{In}{incolor}{89}{\boxspacing}
\begin{Verbatim}[commandchars=\\\{\}]
\PY{n}{pd}\PY{o}{.}\PY{n}{merge}\PY{p}{(}\PY{n}{df1}\PY{p}{,} \PY{n}{df2}\PY{p}{,} \PY{n}{how}\PY{o}{=}\PY{l+s+s1}{\PYZsq{}}\PY{l+s+s1}{outer}\PY{l+s+s1}{\PYZsq{}}\PY{p}{,} \PY{n}{left\PYZus{}index}\PY{o}{=}\PY{k+kc}{True}\PY{p}{,} \PY{n}{right\PYZus{}index}\PY{o}{=}\PY{k+kc}{True}\PY{p}{)}
\end{Verbatim}
\end{tcolorbox}

            \begin{tcolorbox}[breakable, size=fbox, boxrule=.5pt, pad at break*=1mm, opacityfill=0]
\prompt{Out}{outcolor}{89}{\boxspacing}
\begin{Verbatim}[commandchars=\\\{\}]
  key\_x  data1  new\_x key\_y  data2  new\_y
0     b      0      0     a    0.0    0.0
1     b      1      1     b    1.0    1.0
2     a      2      2     d   10.0    2.0
3     c      3      3   NaN    NaN    NaN
4     a      4      4   NaN    NaN    NaN
5     a      5      5   NaN    NaN    NaN
6     b      6      6   NaN    NaN    NaN
\end{Verbatim}
\end{tcolorbox}
        
    \hypertarget{merging-of-dataframe-with-multilevel-index-and-df-with-single-level-index}{%
\subsection{merging of dataframe with multilevel index and DF with
single level
index}\label{merging-of-dataframe-with-multilevel-index-and-df-with-single-level-index}}

    \begin{tcolorbox}[breakable, size=fbox, boxrule=1pt, pad at break*=1mm,colback=cellbackground, colframe=cellborder]
\prompt{In}{incolor}{90}{\boxspacing}
\begin{Verbatim}[commandchars=\\\{\}]
\PY{n}{DF2}
\end{Verbatim}
\end{tcolorbox}

            \begin{tcolorbox}[breakable, size=fbox, boxrule=.5pt, pad at break*=1mm, opacityfill=0]
\prompt{Out}{outcolor}{90}{\boxspacing}
\begin{Verbatim}[commandchars=\\\{\}]
CL1             Corepaper  SEC  DSE
CL2                   100  50   100
Course Semester
PSCS   I                3  NaN  NaN
       II               3  NaN  NaN
       III              4  1.0  NaN
       IV               4  1.0  NaN
       V                4  1.0  1.0
       VI               4  1.0  1.0
CSHons I                4  2.0  NaN
       II               4  2.0  NaN
       III              4  2.0  NaN
       IV               4  2.0  NaN
       V                4  NaN  2.0
       VI               4  NaN  2.0
Bcom   I                4  2.0  NaN
       II               4  2.0  NaN
       III              4  3.0  1.0
       IV               4  3.0  NaN
lifesc I                5  NaN  1.0
       II               5  0.0  1.0
\end{Verbatim}
\end{tcolorbox}
        
    \begin{tcolorbox}[breakable, size=fbox, boxrule=1pt, pad at break*=1mm,colback=cellbackground, colframe=cellborder]
\prompt{In}{incolor}{95}{\boxspacing}
\begin{Verbatim}[commandchars=\\\{\}]
\PY{n}{DF1}
\end{Verbatim}
\end{tcolorbox}

            \begin{tcolorbox}[breakable, size=fbox, boxrule=.5pt, pad at break*=1mm, opacityfill=0]
\prompt{Out}{outcolor}{95}{\boxspacing}
\begin{Verbatim}[commandchars=\\\{\}]
       Semester  Corepaper  SEC  DSE     new
Course
PSCS          I          3  NIL  NIL    PSCS
PSCS         II          3  NIL  NIL    PSCS
PSCS        III          4    1  NIL    PSCS
PSCS         IV          4    1  NIL    PSCS
PSCS          V          4    1    1    PSCS
PSCS         VI          4    1    1    PSCS
CSHons        I          4    2  NIL  CSHons
CSHons       II          4    2  NIL  CSHons
CSHons      III          4    2  NIL  CSHons
CSHons       IV          4    2  NIL  CSHons
CSHons        V          4  NIL    2  CSHons
CSHons       VI          4  NIL    2  CSHons
Bcom          I          4    2  NIL    Bcom
Bcom         II          4    2  NIL    Bcom
Bcom        III          4    3    1    Bcom
Bcom         IV          4    3  NIL    Bcom
lifesc        I          5  NIL    1  lifesc
lifesc       II          5    0    1  lifesc
\end{Verbatim}
\end{tcolorbox}
        
    \begin{tcolorbox}[breakable, size=fbox, boxrule=1pt, pad at break*=1mm,colback=cellbackground, colframe=cellborder]
\prompt{In}{incolor}{ }{\boxspacing}
\begin{Verbatim}[commandchars=\\\{\}]
\PY{n+nb}{len}\PY{p}{(}\PY{n}{DF1}\PY{p}{)}
\end{Verbatim}
\end{tcolorbox}

            \begin{tcolorbox}[breakable, size=fbox, boxrule=.5pt, pad at break*=1mm, opacityfill=0]
\prompt{Out}{outcolor}{ }{\boxspacing}
\begin{Verbatim}[commandchars=\\\{\}]
18
\end{Verbatim}
\end{tcolorbox}
        
    \begin{tcolorbox}[breakable, size=fbox, boxrule=1pt, pad at break*=1mm,colback=cellbackground, colframe=cellborder]
\prompt{In}{incolor}{92}{\boxspacing}
\begin{Verbatim}[commandchars=\\\{\}]
\PY{n}{DF1}\PY{p}{[}\PY{l+s+s1}{\PYZsq{}}\PY{l+s+s1}{new}\PY{l+s+s1}{\PYZsq{}}\PY{p}{]}\PY{o}{=}\PY{n}{DF1}\PY{o}{.}\PY{n}{index}
\end{Verbatim}
\end{tcolorbox}

    \begin{tcolorbox}[breakable, size=fbox, boxrule=1pt, pad at break*=1mm,colback=cellbackground, colframe=cellborder]
\prompt{In}{incolor}{22}{\boxspacing}
\begin{Verbatim}[commandchars=\\\{\}]
\PY{n}{pd}\PY{o}{.}\PY{n}{merge}\PY{p}{(}\PY{n}{DF1}\PY{p}{,} \PY{n}{DF2}\PY{p}{,} \PY{n}{left\PYZus{}on}\PY{o}{=}\PY{p}{[}\PY{l+s+s1}{\PYZsq{}}\PY{l+s+s1}{new}\PY{l+s+s1}{\PYZsq{}}\PY{p}{,}\PY{l+s+s1}{\PYZsq{}}\PY{l+s+s1}{Semester}\PY{l+s+s1}{\PYZsq{}}\PY{p}{]}\PY{p}{,} \PY{n}{right\PYZus{}index}\PY{o}{=}\PY{k+kc}{True}\PY{p}{)}
\end{Verbatim}
\end{tcolorbox}

    \begin{Verbatim}[commandchars=\\\{\}, frame=single, framerule=2mm, rulecolor=\color{outerrorbackground}]
\textcolor{ansi-red-intense}{\textbf{---------------------------------------------------------------------------}}
\textcolor{ansi-red-intense}{\textbf{KeyError}}                                  Traceback (most recent call last)
\textcolor{ansi-green-intense}{\textbf{C:\textbackslash{}Users\textbackslash{}SHARAN\textasciitilde{}1\textbackslash{}AppData\textbackslash{}Local\textbackslash{}Temp/ipykernel\_30116/3260519049.py}} in \textcolor{ansi-cyan}{<module>}
\textcolor{ansi-green-intense}{\textbf{----> 1}}\textcolor{ansi-yellow-intense}{\textbf{ }}pd\textcolor{ansi-yellow-intense}{\textbf{.}}merge\textcolor{ansi-yellow-intense}{\textbf{(}}DF1\textcolor{ansi-yellow-intense}{\textbf{,}} DF2\textcolor{ansi-yellow-intense}{\textbf{,}} left\_on\textcolor{ansi-yellow-intense}{\textbf{=}}\textcolor{ansi-yellow-intense}{\textbf{[}}\textcolor{ansi-blue-intense}{\textbf{'new'}}\textcolor{ansi-yellow-intense}{\textbf{,}}\textcolor{ansi-blue-intense}{\textbf{'Semester'}}\textcolor{ansi-yellow-intense}{\textbf{]}}\textcolor{ansi-yellow-intense}{\textbf{,}} right\_index\textcolor{ansi-yellow-intense}{\textbf{=}}\textcolor{ansi-green-intense}{\textbf{True}}\textcolor{ansi-yellow-intense}{\textbf{)}}

\textcolor{ansi-green-intense}{\textbf{c:\textbackslash{}Users\textbackslash{}Sharanjit Kaur\textbackslash{}AppData\textbackslash{}Local\textbackslash{}Programs\textbackslash{}Python\textbackslash{}Python39\textbackslash{}lib\textbackslash{}site-packages\textbackslash{}pandas\textbackslash{}core\textbackslash{}reshape\textbackslash{}merge.py}} in \textcolor{ansi-cyan}{merge}\textcolor{ansi-blue-intense}{\textbf{(left, right, how, on, left\_on, right\_on, left\_index, right\_index, sort, suffixes, copy, indicator, validate)}}
\textcolor{ansi-green}{     72}     validate\textcolor{ansi-yellow-intense}{\textbf{=}}\textcolor{ansi-green-intense}{\textbf{None}}\textcolor{ansi-yellow-intense}{\textbf{,}}
\textcolor{ansi-green}{     73} ) -> "DataFrame":
\textcolor{ansi-green-intense}{\textbf{---> 74}}\textcolor{ansi-yellow-intense}{\textbf{     op = \_MergeOperation(
}}\textcolor{ansi-green}{     75}         left\textcolor{ansi-yellow-intense}{\textbf{,}}
\textcolor{ansi-green}{     76}         right\textcolor{ansi-yellow-intense}{\textbf{,}}

\textcolor{ansi-green-intense}{\textbf{c:\textbackslash{}Users\textbackslash{}Sharanjit Kaur\textbackslash{}AppData\textbackslash{}Local\textbackslash{}Programs\textbackslash{}Python\textbackslash{}Python39\textbackslash{}lib\textbackslash{}site-packages\textbackslash{}pandas\textbackslash{}core\textbackslash{}reshape\textbackslash{}merge.py}} in \textcolor{ansi-cyan}{\_\_init\_\_}\textcolor{ansi-blue-intense}{\textbf{(self, left, right, how, on, left\_on, right\_on, axis, left\_index, right\_index, sort, suffixes, copy, indicator, validate)}}
\textcolor{ansi-green}{    666}             self\textcolor{ansi-yellow-intense}{\textbf{.}}right\_join\_keys\textcolor{ansi-yellow-intense}{\textbf{,}}
\textcolor{ansi-green}{    667}             self\textcolor{ansi-yellow-intense}{\textbf{.}}join\_names\textcolor{ansi-yellow-intense}{\textbf{,}}
\textcolor{ansi-green-intense}{\textbf{--> 668}}\textcolor{ansi-yellow-intense}{\textbf{         ) = self.\_get\_merge\_keys()
}}\textcolor{ansi-green}{    669} 
\textcolor{ansi-green}{    670}         \textcolor{ansi-red-intense}{\textbf{\# validate the merge keys dtypes. We may need to coerce}}

\textcolor{ansi-green-intense}{\textbf{c:\textbackslash{}Users\textbackslash{}Sharanjit Kaur\textbackslash{}AppData\textbackslash{}Local\textbackslash{}Programs\textbackslash{}Python\textbackslash{}Python39\textbackslash{}lib\textbackslash{}site-packages\textbackslash{}pandas\textbackslash{}core\textbackslash{}reshape\textbackslash{}merge.py}} in \textcolor{ansi-cyan}{\_get\_merge\_keys}\textcolor{ansi-blue-intense}{\textbf{(self)}}
\textcolor{ansi-green}{   1056}                     join\_names\textcolor{ansi-yellow-intense}{\textbf{.}}append\textcolor{ansi-yellow-intense}{\textbf{(}}\textcolor{ansi-green-intense}{\textbf{None}}\textcolor{ansi-yellow-intense}{\textbf{)}}
\textcolor{ansi-green}{   1057}                 \textcolor{ansi-green-intense}{\textbf{else}}\textcolor{ansi-yellow-intense}{\textbf{:}}
\textcolor{ansi-green-intense}{\textbf{-> 1058}}\textcolor{ansi-yellow-intense}{\textbf{                     }}left\_keys\textcolor{ansi-yellow-intense}{\textbf{.}}append\textcolor{ansi-yellow-intense}{\textbf{(}}left\textcolor{ansi-yellow-intense}{\textbf{.}}\_get\_label\_or\_level\_values\textcolor{ansi-yellow-intense}{\textbf{(}}k\textcolor{ansi-yellow-intense}{\textbf{)}}\textcolor{ansi-yellow-intense}{\textbf{)}}
\textcolor{ansi-green}{   1059}                     join\_names\textcolor{ansi-yellow-intense}{\textbf{.}}append\textcolor{ansi-yellow-intense}{\textbf{(}}k\textcolor{ansi-yellow-intense}{\textbf{)}}
\textcolor{ansi-green}{   1060}             \textcolor{ansi-green-intense}{\textbf{if}} isinstance\textcolor{ansi-yellow-intense}{\textbf{(}}self\textcolor{ansi-yellow-intense}{\textbf{.}}right\textcolor{ansi-yellow-intense}{\textbf{.}}index\textcolor{ansi-yellow-intense}{\textbf{,}} MultiIndex\textcolor{ansi-yellow-intense}{\textbf{)}}\textcolor{ansi-yellow-intense}{\textbf{:}}

\textcolor{ansi-green-intense}{\textbf{c:\textbackslash{}Users\textbackslash{}Sharanjit Kaur\textbackslash{}AppData\textbackslash{}Local\textbackslash{}Programs\textbackslash{}Python\textbackslash{}Python39\textbackslash{}lib\textbackslash{}site-packages\textbackslash{}pandas\textbackslash{}core\textbackslash{}generic.py}} in \textcolor{ansi-cyan}{\_get\_label\_or\_level\_values}\textcolor{ansi-blue-intense}{\textbf{(self, key, axis)}}
\textcolor{ansi-green}{   1682}             values \textcolor{ansi-yellow-intense}{\textbf{=}} self\textcolor{ansi-yellow-intense}{\textbf{.}}axes\textcolor{ansi-yellow-intense}{\textbf{[}}axis\textcolor{ansi-yellow-intense}{\textbf{]}}\textcolor{ansi-yellow-intense}{\textbf{.}}get\_level\_values\textcolor{ansi-yellow-intense}{\textbf{(}}key\textcolor{ansi-yellow-intense}{\textbf{)}}\textcolor{ansi-yellow-intense}{\textbf{.}}\_values
\textcolor{ansi-green}{   1683}         \textcolor{ansi-green-intense}{\textbf{else}}\textcolor{ansi-yellow-intense}{\textbf{:}}
\textcolor{ansi-green-intense}{\textbf{-> 1684}}\textcolor{ansi-yellow-intense}{\textbf{             }}\textcolor{ansi-green-intense}{\textbf{raise}} KeyError\textcolor{ansi-yellow-intense}{\textbf{(}}key\textcolor{ansi-yellow-intense}{\textbf{)}}
\textcolor{ansi-green}{   1685} 
\textcolor{ansi-green}{   1686}         \textcolor{ansi-red-intense}{\textbf{\# Check for duplicates}}

\textcolor{ansi-red-intense}{\textbf{KeyError}}: 'new'
    \end{Verbatim}

    \hypertarget{problem-is-to-merge-df2-on-column-with-that-of-index-of-df1.-so-adding-a-new-col-having-semester-val-iii-and-in-df1-making-semester-as-index}{%
\section{Problem is to merge DF2 on column with that of index of DF1. So
adding a new col having Semester val I,II and in DF1 making Semester as
index}\label{problem-is-to-merge-df2-on-column-with-that-of-index-of-df1.-so-adding-a-new-col-having-semester-val-iii-and-in-df1-making-semester-as-index}}

    \hypertarget{changing-index-of-df1-as-column-and-column-as-semester-index}{%
\subsection{changing index of DF1 as column and column as `semester'
index}\label{changing-index-of-df1-as-column-and-column-as-semester-index}}

    \begin{tcolorbox}[breakable, size=fbox, boxrule=1pt, pad at break*=1mm,colback=cellbackground, colframe=cellborder]
\prompt{In}{incolor}{96}{\boxspacing}
\begin{Verbatim}[commandchars=\\\{\}]
\PY{n}{DF1}\PY{o}{.}\PY{n}{reset\PYZus{}index}\PY{p}{(}\PY{n}{level}\PY{o}{=}\PY{l+m+mi}{0}\PY{p}{,}\PY{n}{inplace}\PY{o}{=}\PY{k+kc}{True}\PY{p}{)}
\end{Verbatim}
\end{tcolorbox}

    \begin{tcolorbox}[breakable, size=fbox, boxrule=1pt, pad at break*=1mm,colback=cellbackground, colframe=cellborder]
\prompt{In}{incolor}{100}{\boxspacing}
\begin{Verbatim}[commandchars=\\\{\}]
\PY{n}{DF2}
\end{Verbatim}
\end{tcolorbox}

            \begin{tcolorbox}[breakable, size=fbox, boxrule=.5pt, pad at break*=1mm, opacityfill=0]
\prompt{Out}{outcolor}{100}{\boxspacing}
\begin{Verbatim}[commandchars=\\\{\}]
CL1             Corepaper  SEC  DSE
CL2                   100  50   100
Course Semester
PSCS   I                3  NaN  NaN
       II               3  NaN  NaN
       III              4  1.0  NaN
       IV               4  1.0  NaN
       V                4  1.0  1.0
       VI               4  1.0  1.0
CSHons I                4  2.0  NaN
       II               4  2.0  NaN
       III              4  2.0  NaN
       IV               4  2.0  NaN
       V                4  NaN  2.0
       VI               4  NaN  2.0
Bcom   I                4  2.0  NaN
       II               4  2.0  NaN
       III              4  3.0  1.0
       IV               4  3.0  NaN
lifesc I                5  NaN  1.0
       II               5  0.0  1.0
\end{Verbatim}
\end{tcolorbox}
        
    \begin{tcolorbox}[breakable, size=fbox, boxrule=1pt, pad at break*=1mm,colback=cellbackground, colframe=cellborder]
\prompt{In}{incolor}{98}{\boxspacing}
\begin{Verbatim}[commandchars=\\\{\}]
\PY{n}{DF1}\PY{o}{.}\PY{n}{set\PYZus{}index}\PY{p}{(}\PY{l+s+s1}{\PYZsq{}}\PY{l+s+s1}{Semester}\PY{l+s+s1}{\PYZsq{}}\PY{p}{,}\PY{n}{inplace}\PY{o}{=}\PY{k+kc}{True}\PY{p}{)}
\end{Verbatim}
\end{tcolorbox}

    \begin{tcolorbox}[breakable, size=fbox, boxrule=1pt, pad at break*=1mm,colback=cellbackground, colframe=cellborder]
\prompt{In}{incolor}{24}{\boxspacing}
\begin{Verbatim}[commandchars=\\\{\}]
\PY{n}{DF2}
\end{Verbatim}
\end{tcolorbox}

            \begin{tcolorbox}[breakable, size=fbox, boxrule=.5pt, pad at break*=1mm, opacityfill=0]
\prompt{Out}{outcolor}{24}{\boxspacing}
\begin{Verbatim}[commandchars=\\\{\}]
                 Corepaper  SEC  DSE
Course Semester
PSCS   I                 3  NIL  NIL
       II                3  NIL  NIL
       III               4    1  NIL
       IV                4    1  NIL
       V                 4    1    1
       VI                4    1    1
CSHons I                 4    2  NIL
       II                4    2  NIL
       III               4    2  NIL
       IV                4    2  NIL
       V                 4  NIL    2
       VI                4  NIL    2
Bcom   I                 4    2  NIL
       II                4    2  NIL
       III               4    3    1
       IV                4    3  NIL
lifesc I                 5  NIL    1
       II                5    0    1
\end{Verbatim}
\end{tcolorbox}
        
    \hypertarget{adding-a-newcol-having-semester-values-by-mapping-any-col-value}{%
\subsection{adding a newcol having Semester values by mapping any col
value}\label{adding-a-newcol-having-semester-values-by-mapping-any-col-value}}

    \begin{tcolorbox}[breakable, size=fbox, boxrule=1pt, pad at break*=1mm,colback=cellbackground, colframe=cellborder]
\prompt{In}{incolor}{31}{\boxspacing}
\begin{Verbatim}[commandchars=\\\{\}]
\PY{n}{dict1}\PY{o}{=}\PY{p}{\PYZob{}}\PY{l+m+mi}{1}\PY{p}{:}\PY{l+s+s1}{\PYZsq{}}\PY{l+s+s1}{I}\PY{l+s+s1}{\PYZsq{}}\PY{p}{,}\PY{l+m+mi}{2}\PY{p}{:}\PY{l+s+s1}{\PYZsq{}}\PY{l+s+s1}{II}\PY{l+s+s1}{\PYZsq{}}\PY{p}{,}\PY{l+m+mi}{3}\PY{p}{:}\PY{l+s+s1}{\PYZsq{}}\PY{l+s+s1}{III}\PY{l+s+s1}{\PYZsq{}}\PY{p}{,}\PY{l+m+mi}{4}\PY{p}{:}\PY{l+s+s1}{\PYZsq{}}\PY{l+s+s1}{IV}\PY{l+s+s1}{\PYZsq{}}\PY{p}{,}\PY{l+s+s1}{\PYZsq{}}\PY{l+s+s1}{NIL}\PY{l+s+s1}{\PYZsq{}}\PY{p}{:}\PY{l+s+s1}{\PYZsq{}}\PY{l+s+s1}{VI}\PY{l+s+s1}{\PYZsq{}}\PY{p}{\PYZcb{}}
\end{Verbatim}
\end{tcolorbox}

    \begin{tcolorbox}[breakable, size=fbox, boxrule=1pt, pad at break*=1mm,colback=cellbackground, colframe=cellborder]
\prompt{In}{incolor}{33}{\boxspacing}
\begin{Verbatim}[commandchars=\\\{\}]
\PY{n}{DFnew}\PY{o}{=}\PY{n}{DF2}\PY{p}{[}\PY{n}{DF2}\PY{p}{[}\PY{l+s+s1}{\PYZsq{}}\PY{l+s+s1}{SEC}\PY{l+s+s1}{\PYZsq{}}\PY{p}{]}\PY{o}{.}\PY{n}{notna}\PY{p}{(}\PY{p}{)}\PY{o}{.}\PY{n}{values}\PY{o}{.}\PY{n}{reshape}\PY{p}{(}\PY{l+m+mi}{18}\PY{p}{,}\PY{l+m+mi}{1}\PY{p}{)}\PY{p}{]}\PY{p}{[}\PY{l+s+s1}{\PYZsq{}}\PY{l+s+s1}{SEC}\PY{l+s+s1}{\PYZsq{}}\PY{p}{]}
\end{Verbatim}
\end{tcolorbox}

    \begin{tcolorbox}[breakable, size=fbox, boxrule=1pt, pad at break*=1mm,colback=cellbackground, colframe=cellborder]
\prompt{In}{incolor}{34}{\boxspacing}
\begin{Verbatim}[commandchars=\\\{\}]
\PY{n}{DFnew}\PY{o}{.}\PY{n}{map}\PY{p}{(}\PY{n}{dict1}\PY{p}{)}
\end{Verbatim}
\end{tcolorbox}

            \begin{tcolorbox}[breakable, size=fbox, boxrule=.5pt, pad at break*=1mm, opacityfill=0]
\prompt{Out}{outcolor}{34}{\boxspacing}
\begin{Verbatim}[commandchars=\\\{\}]
Course  Semester
PSCS    I            VI
        II           VI
        III           I
        IV            I
        V             I
        VI            I
CSHons  I            II
        II           II
        III          II
        IV           II
        V            VI
        VI           VI
Bcom    I            II
        II           II
        III         III
        IV          III
lifesc  I            VI
        II          NaN
Name: SEC, dtype: object
\end{Verbatim}
\end{tcolorbox}
        
    \hypertarget{while-mapping-take-care-of-the-levels}{%
\subsection{while mapping take care of the
levels}\label{while-mapping-take-care-of-the-levels}}

map(), applymap() and apply() methods are used to modify the data
whereas their usage is slightly different. * pandas.Series.map(): in the
data series element * pandas.DataFrame.applymap() on the each element of
data frame * pandas.Series.apply(): used on each element whereas
pandas.DataFrame.apply() is applied on row/column wise

    \begin{tcolorbox}[breakable, size=fbox, boxrule=1pt, pad at break*=1mm,colback=cellbackground, colframe=cellborder]
\prompt{In}{incolor}{35}{\boxspacing}
\begin{Verbatim}[commandchars=\\\{\}]
\PY{n}{dict1}\PY{o}{=}\PY{p}{\PYZob{}}\PY{l+m+mi}{1}\PY{p}{:}\PY{l+s+s1}{\PYZsq{}}\PY{l+s+s1}{I}\PY{l+s+s1}{\PYZsq{}}\PY{p}{,}\PY{l+m+mi}{2}\PY{p}{:}\PY{l+s+s1}{\PYZsq{}}\PY{l+s+s1}{II}\PY{l+s+s1}{\PYZsq{}}\PY{p}{,}\PY{l+m+mi}{3}\PY{p}{:}\PY{l+s+s1}{\PYZsq{}}\PY{l+s+s1}{III}\PY{l+s+s1}{\PYZsq{}}\PY{p}{,}\PY{l+m+mi}{4}\PY{p}{:}\PY{l+s+s1}{\PYZsq{}}\PY{l+s+s1}{IV}\PY{l+s+s1}{\PYZsq{}}\PY{p}{,}\PY{l+s+s1}{\PYZsq{}}\PY{l+s+s1}{NaN}\PY{l+s+s1}{\PYZsq{}}\PY{p}{:}\PY{l+s+s1}{\PYZsq{}}\PY{l+s+s1}{VI}\PY{l+s+s1}{\PYZsq{}}\PY{p}{\PYZcb{}}
\PY{n}{DF2}\PY{p}{[}\PY{l+s+s1}{\PYZsq{}}\PY{l+s+s1}{newcol}\PY{l+s+s1}{\PYZsq{}}\PY{p}{]}\PY{o}{=}\PY{n}{DF2}\PY{p}{[}\PY{l+s+s1}{\PYZsq{}}\PY{l+s+s1}{SEC}\PY{l+s+s1}{\PYZsq{}}\PY{p}{]}\PY{o}{.}\PY{n}{map}\PY{p}{(}\PY{n}{dict1}\PY{p}{)}
\end{Verbatim}
\end{tcolorbox}

    \hypertarget{example-on-map-apply-applymap}{%
\subsection{example on map() apply()
applymap()}\label{example-on-map-apply-applymap}}

    \begin{tcolorbox}[breakable, size=fbox, boxrule=1pt, pad at break*=1mm,colback=cellbackground, colframe=cellborder]
\prompt{In}{incolor}{60}{\boxspacing}
\begin{Verbatim}[commandchars=\\\{\}]
\PY{k+kn}{import} \PY{n+nn}{pandas} \PY{k}{as} \PY{n+nn}{pd}
\PY{k+kn}{import} \PY{n+nn}{numpy} \PY{k}{as} \PY{n+nn}{np}
\PY{n}{data} \PY{o}{=} \PY{p}{[}\PY{p}{(}\PY{l+m+mi}{30}\PY{p}{,}\PY{l+m+mi}{50}\PY{p}{,}\PY{l+m+mi}{70}\PY{p}{,}\PY{l+m+mi}{80}\PY{p}{)}\PY{p}{,} \PY{p}{(}\PY{l+m+mi}{21}\PY{p}{,}\PY{l+m+mi}{41}\PY{p}{,}\PY{l+m+mi}{61}\PY{p}{,}\PY{l+m+mi}{81}\PY{p}{)}\PY{p}{,}\PY{p}{(}\PY{l+m+mi}{5}\PY{p}{,}\PY{l+m+mi}{5}\PY{p}{,}\PY{l+m+mi}{8}\PY{p}{,}\PY{l+m+mi}{9}\PY{p}{)}\PY{p}{]}
\PY{n}{df} \PY{o}{=} \PY{n}{pd}\PY{o}{.}\PY{n}{DataFrame}\PY{p}{(}\PY{n}{data}\PY{p}{,} \PY{n}{columns} \PY{o}{=} \PY{p}{[}\PY{l+s+s1}{\PYZsq{}}\PY{l+s+s1}{f1}\PY{l+s+s1}{\PYZsq{}}\PY{p}{,}\PY{l+s+s1}{\PYZsq{}}\PY{l+s+s1}{f2}\PY{l+s+s1}{\PYZsq{}}\PY{p}{,}\PY{l+s+s1}{\PYZsq{}}\PY{l+s+s1}{f3}\PY{l+s+s1}{\PYZsq{}}\PY{p}{,}\PY{l+s+s1}{\PYZsq{}}\PY{l+s+s1}{f4}\PY{l+s+s1}{\PYZsq{}}\PY{p}{]}\PY{p}{)}
\PY{n+nb}{print}\PY{p}{(}\PY{n}{df}\PY{p}{)}
\end{Verbatim}
\end{tcolorbox}

    \begin{Verbatim}[commandchars=\\\{\}]
   f1  f2  f3  f4
0  30  50  70  80
1  21  41  61  81
2   5   5   8   9
    \end{Verbatim}

    \begin{tcolorbox}[breakable, size=fbox, boxrule=1pt, pad at break*=1mm,colback=cellbackground, colframe=cellborder]
\prompt{In}{incolor}{61}{\boxspacing}
\begin{Verbatim}[commandchars=\\\{\}]
\PY{n}{df}\PY{p}{[}\PY{l+s+s1}{\PYZsq{}}\PY{l+s+s1}{f1}\PY{l+s+s1}{\PYZsq{}}\PY{p}{]}\PY{o}{=}\PY{n}{df}\PY{p}{[}\PY{l+s+s1}{\PYZsq{}}\PY{l+s+s1}{f1}\PY{l+s+s1}{\PYZsq{}}\PY{p}{]}\PY{o}{.}\PY{n}{map}\PY{p}{(}\PY{k}{lambda} \PY{n}{x}\PY{p}{:} \PY{n}{x}\PY{o}{/}\PY{l+m+mi}{100}\PY{p}{)}
\end{Verbatim}
\end{tcolorbox}

    \begin{tcolorbox}[breakable, size=fbox, boxrule=1pt, pad at break*=1mm,colback=cellbackground, colframe=cellborder]
\prompt{In}{incolor}{64}{\boxspacing}
\begin{Verbatim}[commandchars=\\\{\}]
\PY{n}{df}
\end{Verbatim}
\end{tcolorbox}

            \begin{tcolorbox}[breakable, size=fbox, boxrule=.5pt, pad at break*=1mm, opacityfill=0]
\prompt{Out}{outcolor}{64}{\boxspacing}
\begin{Verbatim}[commandchars=\\\{\}]
       f1  f2  f3  f4
0  0.0030  50  70  80
1  0.0021  41  61  81
2  0.0005   5   8   9
\end{Verbatim}
\end{tcolorbox}
        
    \begin{tcolorbox}[breakable, size=fbox, boxrule=1pt, pad at break*=1mm,colback=cellbackground, colframe=cellborder]
\prompt{In}{incolor}{63}{\boxspacing}
\begin{Verbatim}[commandchars=\\\{\}]
\PY{n}{df}\PY{p}{[}\PY{l+s+s1}{\PYZsq{}}\PY{l+s+s1}{f1}\PY{l+s+s1}{\PYZsq{}}\PY{p}{]}\PY{o}{=}\PY{n}{df}\PY{p}{[}\PY{l+s+s1}{\PYZsq{}}\PY{l+s+s1}{f1}\PY{l+s+s1}{\PYZsq{}}\PY{p}{]}\PY{o}{.}\PY{n}{apply}\PY{p}{(}\PY{k}{lambda} \PY{n}{x}\PY{p}{:} \PY{n}{x}\PY{o}{/}\PY{l+m+mi}{100}\PY{p}{)}
\end{Verbatim}
\end{tcolorbox}

    \begin{tcolorbox}[breakable, size=fbox, boxrule=1pt, pad at break*=1mm,colback=cellbackground, colframe=cellborder]
\prompt{In}{incolor}{65}{\boxspacing}
\begin{Verbatim}[commandchars=\\\{\}]
\PY{n}{df3} \PY{o}{=} \PY{n}{df}\PY{o}{.}\PY{n}{apply}\PY{p}{(}\PY{k}{lambda} \PY{n}{x}\PY{p}{:} \PY{n}{x}\PY{o}{/}\PY{l+m+mi}{10}\PY{p}{)}
\PY{n}{df3}
\end{Verbatim}
\end{tcolorbox}

            \begin{tcolorbox}[breakable, size=fbox, boxrule=.5pt, pad at break*=1mm, opacityfill=0]
\prompt{Out}{outcolor}{65}{\boxspacing}
\begin{Verbatim}[commandchars=\\\{\}]
        f1   f2   f3   f4
0  0.00030  5.0  7.0  8.0
1  0.00021  4.1  6.1  8.1
2  0.00005  0.5  0.8  0.9
\end{Verbatim}
\end{tcolorbox}
        
    \begin{tcolorbox}[breakable, size=fbox, boxrule=1pt, pad at break*=1mm,colback=cellbackground, colframe=cellborder]
\prompt{In}{incolor}{66}{\boxspacing}
\begin{Verbatim}[commandchars=\\\{\}]
\PY{n}{df3}\PY{o}{=}\PY{n}{df}\PY{o}{.}\PY{n}{applymap}\PY{p}{(}\PY{k}{lambda} \PY{n}{a}\PY{p}{:} \PY{n+nb}{str}\PY{p}{(}\PY{n}{a}\PY{p}{)}\PY{o}{+}\PY{l+s+s2}{\PYZdq{}}\PY{l+s+s2}{.00}\PY{l+s+s2}{\PYZdq{}}\PY{p}{)}
\PY{n}{df3}
\end{Verbatim}
\end{tcolorbox}

            \begin{tcolorbox}[breakable, size=fbox, boxrule=.5pt, pad at break*=1mm, opacityfill=0]
\prompt{Out}{outcolor}{66}{\boxspacing}
\begin{Verbatim}[commandchars=\\\{\}]
          f1     f2     f3     f4
0   0.003.00  50.00  70.00  80.00
1  0.0021.00  41.00  61.00  81.00
2  0.0005.00   5.00   8.00   9.00
\end{Verbatim}
\end{tcolorbox}
        
    \hypertarget{merging-will-preserve-the-index-level-as-in-left-side}{%
\subsection{merging will preserve the index level as in left
side}\label{merging-will-preserve-the-index-level-as-in-left-side}}

    \begin{tcolorbox}[breakable, size=fbox, boxrule=1pt, pad at break*=1mm,colback=cellbackground, colframe=cellborder]
\prompt{In}{incolor}{ }{\boxspacing}
\begin{Verbatim}[commandchars=\\\{\}]
\PY{n}{pd}\PY{o}{.}\PY{n}{merge}\PY{p}{(}\PY{n}{DF2}\PY{p}{,} \PY{n}{DF1}\PY{p}{,} \PY{n}{left\PYZus{}on}\PY{o}{=}\PY{p}{[}\PY{l+s+s1}{\PYZsq{}}\PY{l+s+s1}{newcol}\PY{l+s+s1}{\PYZsq{}}\PY{p}{]}\PY{p}{,} \PY{n}{right\PYZus{}index}\PY{o}{=}\PY{k+kc}{True}\PY{p}{)}
\end{Verbatim}
\end{tcolorbox}

    \hypertarget{explicitly-using-join-clause-join-columns-of-another-dataframe-either-on-index-or-on-a-key-column.-efficiently-join-multiple-dataframe-objects-by-index-at-once-by-passing-a-list.}{%
\section{Explicitly using join clause: Join columns of another DataFrame
either on index or on a key column. Efficiently join multiple DataFrame
objects by index at once by passing a
list.}\label{explicitly-using-join-clause-join-columns-of-another-dataframe-either-on-index-or-on-a-key-column.-efficiently-join-multiple-dataframe-objects-by-index-at-once-by-passing-a-list.}}

    \begin{tcolorbox}[breakable, size=fbox, boxrule=1pt, pad at break*=1mm,colback=cellbackground, colframe=cellborder]
\prompt{In}{incolor}{76}{\boxspacing}
\begin{Verbatim}[commandchars=\\\{\}]
\PY{n}{df1}
\end{Verbatim}
\end{tcolorbox}

            \begin{tcolorbox}[breakable, size=fbox, boxrule=.5pt, pad at break*=1mm, opacityfill=0]
\prompt{Out}{outcolor}{76}{\boxspacing}
\begin{Verbatim}[commandchars=\\\{\}]
  key  data1  new
0   b      0    0
1   b      1    1
2   a      2    2
3   c      3    3
4   a      4    4
5   a      5    5
6   b      6    6
\end{Verbatim}
\end{tcolorbox}
        
    \begin{tcolorbox}[breakable, size=fbox, boxrule=1pt, pad at break*=1mm,colback=cellbackground, colframe=cellborder]
\prompt{In}{incolor}{68}{\boxspacing}
\begin{Verbatim}[commandchars=\\\{\}]
\PY{c+c1}{\PYZsh{}df2=df2.append(df2.iloc[2])}
\PY{n}{df2}
\end{Verbatim}
\end{tcolorbox}

            \begin{tcolorbox}[breakable, size=fbox, boxrule=.5pt, pad at break*=1mm, opacityfill=0]
\prompt{Out}{outcolor}{68}{\boxspacing}
\begin{Verbatim}[commandchars=\\\{\}]
  key  data1  new
0   a      0    1
1   b      1    2
2   d      2   10
9   a      0    1
\end{Verbatim}
\end{tcolorbox}
        
    \begin{tcolorbox}[breakable, size=fbox, boxrule=1pt, pad at break*=1mm,colback=cellbackground, colframe=cellborder]
\prompt{In}{incolor}{15}{\boxspacing}
\begin{Verbatim}[commandchars=\\\{\}]
\PY{n}{df2}\PY{o}{.}\PY{n}{append}\PY{p}{(}\PY{n}{df2}\PY{o}{.}\PY{n}{iloc}\PY{p}{[}\PY{l+m+mi}{2}\PY{p}{]}\PY{p}{)}
\PY{n}{df2}\PY{o}{.}\PY{n}{iloc}\PY{p}{[}\PY{l+m+mi}{2}\PY{p}{,}\PY{p}{:}\PY{p}{]}\PY{o}{=}\PY{p}{[}\PY{l+s+s1}{\PYZsq{}}\PY{l+s+s1}{d}\PY{l+s+s1}{\PYZsq{}}\PY{p}{,}\PY{l+m+mi}{2}\PY{p}{,}\PY{l+m+mi}{3}\PY{p}{]}
\end{Verbatim}
\end{tcolorbox}

    \begin{tcolorbox}[breakable, size=fbox, boxrule=1pt, pad at break*=1mm,colback=cellbackground, colframe=cellborder]
\prompt{In}{incolor}{74}{\boxspacing}
\begin{Verbatim}[commandchars=\\\{\}]
\PY{n}{df2}
\end{Verbatim}
\end{tcolorbox}

            \begin{tcolorbox}[breakable, size=fbox, boxrule=.5pt, pad at break*=1mm, opacityfill=0]
\prompt{Out}{outcolor}{74}{\boxspacing}
\begin{Verbatim}[commandchars=\\\{\}]
  key  data1  new
0   a      0    1
1   b      1    2
2   d      2   10
9   a      0    1
\end{Verbatim}
\end{tcolorbox}
        
    \begin{tcolorbox}[breakable, size=fbox, boxrule=1pt, pad at break*=1mm,colback=cellbackground, colframe=cellborder]
\prompt{In}{incolor}{70}{\boxspacing}
\begin{Verbatim}[commandchars=\\\{\}]
\PY{n}{pd}\PY{o}{.}\PY{n}{merge}\PY{p}{(}\PY{n}{df1}\PY{p}{,}\PY{n}{df2}\PY{p}{)}
\end{Verbatim}
\end{tcolorbox}

            \begin{tcolorbox}[breakable, size=fbox, boxrule=.5pt, pad at break*=1mm, opacityfill=0]
\prompt{Out}{outcolor}{70}{\boxspacing}
\begin{Verbatim}[commandchars=\\\{\}]
Empty DataFrame
Columns: [key, data1, new]
Index: []
\end{Verbatim}
\end{tcolorbox}
        
    \hypertarget{by-default-join-is-done-matching-row-index-of-left-side-table-with-that-of-rhsfor-preserving-its-row-index}{%
\subsection{by default join is done matching row-index of left side
table with that of RHSfor preserving its row
index}\label{by-default-join-is-done-matching-row-index-of-left-side-table-with-that-of-rhsfor-preserving-its-row-index}}

\begin{itemize}
\tightlist
\item
  lsuffix/rsuffix is to rename common columns in LHS/RHS table by
  specified word
\end{itemize}

    \begin{tcolorbox}[breakable, size=fbox, boxrule=1pt, pad at break*=1mm,colback=cellbackground, colframe=cellborder]
\prompt{In}{incolor}{75}{\boxspacing}
\begin{Verbatim}[commandchars=\\\{\}]
\PY{n}{df1}\PY{o}{.}\PY{n}{join}\PY{p}{(}\PY{n}{df2}\PY{p}{,}\PY{n}{lsuffix}\PY{o}{=}\PY{l+s+s1}{\PYZsq{}}\PY{l+s+s1}{left}\PY{l+s+s1}{\PYZsq{}}\PY{p}{)}
\end{Verbatim}
\end{tcolorbox}

            \begin{tcolorbox}[breakable, size=fbox, boxrule=.5pt, pad at break*=1mm, opacityfill=0]
\prompt{Out}{outcolor}{75}{\boxspacing}
\begin{Verbatim}[commandchars=\\\{\}]
  keyleft  data1left  newleft  key  data1   new
0       b          0        0    a    0.0   1.0
1       b          1        1    b    1.0   2.0
2       a          2        2    d    2.0  10.0
3       c          3        3  NaN    NaN   NaN
4       a          4        4  NaN    NaN   NaN
5       a          5        5  NaN    NaN   NaN
6       b          6        6  NaN    NaN   NaN
\end{Verbatim}
\end{tcolorbox}
        
    \begin{tcolorbox}[breakable, size=fbox, boxrule=1pt, pad at break*=1mm,colback=cellbackground, colframe=cellborder]
\prompt{In}{incolor}{80}{\boxspacing}
\begin{Verbatim}[commandchars=\\\{\}]
\PY{n}{df2}\PY{o}{.}\PY{n}{join}\PY{p}{(}\PY{n}{df1}\PY{p}{,}\PY{n}{rsuffix}\PY{o}{=}\PY{l+s+s1}{\PYZsq{}}\PY{l+s+s1}{right}\PY{l+s+s1}{\PYZsq{}}\PY{p}{,}\PY{n}{how}\PY{o}{=}\PY{l+s+s1}{\PYZsq{}}\PY{l+s+s1}{outer}\PY{l+s+s1}{\PYZsq{}}\PY{p}{)}
\end{Verbatim}
\end{tcolorbox}

            \begin{tcolorbox}[breakable, size=fbox, boxrule=.5pt, pad at break*=1mm, opacityfill=0]
\prompt{Out}{outcolor}{80}{\boxspacing}
\begin{Verbatim}[commandchars=\\\{\}]
   key  data1   new keyright  data1right  newright
0    a    0.0   1.0        b         0.0       0.0
1    b    1.0   2.0        b         1.0       1.0
2    d    2.0  10.0        a         2.0       2.0
3  NaN    NaN   NaN        c         3.0       3.0
4  NaN    NaN   NaN        a         4.0       4.0
5  NaN    NaN   NaN        a         5.0       5.0
6  NaN    NaN   NaN        b         6.0       6.0
9    a    0.0   1.0      NaN         NaN       NaN
\end{Verbatim}
\end{tcolorbox}
        
    \begin{tcolorbox}[breakable, size=fbox, boxrule=1pt, pad at break*=1mm,colback=cellbackground, colframe=cellborder]
\prompt{In}{incolor}{24}{\boxspacing}
\begin{Verbatim}[commandchars=\\\{\}]
\PY{n}{df2}\PY{o}{=}\PY{n}{df2}\PY{o}{.}\PY{n}{append}\PY{p}{(}\PY{n}{df2}\PY{o}{.}\PY{n}{iloc}\PY{p}{[}\PY{l+m+mi}{0}\PY{p}{,}\PY{p}{:}\PY{p}{]}\PY{p}{)}
\end{Verbatim}
\end{tcolorbox}

    \begin{tcolorbox}[breakable, size=fbox, boxrule=1pt, pad at break*=1mm,colback=cellbackground, colframe=cellborder]
\prompt{In}{incolor}{26}{\boxspacing}
\begin{Verbatim}[commandchars=\\\{\}]
\PY{n}{df2}\PY{o}{.}\PY{n}{index}\PY{o}{=}\PY{p}{[}\PY{l+m+mi}{0}\PY{p}{,}\PY{l+m+mi}{1}\PY{p}{,}\PY{l+m+mi}{2}\PY{p}{,}\PY{l+m+mi}{9}\PY{p}{]}
\end{Verbatim}
\end{tcolorbox}

    \begin{tcolorbox}[breakable, size=fbox, boxrule=1pt, pad at break*=1mm,colback=cellbackground, colframe=cellborder]
\prompt{In}{incolor}{86}{\boxspacing}
\begin{Verbatim}[commandchars=\\\{\}]
\PY{n}{df2}
\end{Verbatim}
\end{tcolorbox}

            \begin{tcolorbox}[breakable, size=fbox, boxrule=.5pt, pad at break*=1mm, opacityfill=0]
\prompt{Out}{outcolor}{86}{\boxspacing}
\begin{Verbatim}[commandchars=\\\{\}]
  key  data1  new1
0   a      0     1
1   b      1     2
2   d      2    10
9   a      0     1
\end{Verbatim}
\end{tcolorbox}
        
    \begin{tcolorbox}[breakable, size=fbox, boxrule=1pt, pad at break*=1mm,colback=cellbackground, colframe=cellborder]
\prompt{In}{incolor}{83}{\boxspacing}
\begin{Verbatim}[commandchars=\\\{\}]
\PY{n}{df1}
\end{Verbatim}
\end{tcolorbox}

            \begin{tcolorbox}[breakable, size=fbox, boxrule=.5pt, pad at break*=1mm, opacityfill=0]
\prompt{Out}{outcolor}{83}{\boxspacing}
\begin{Verbatim}[commandchars=\\\{\}]
  key  data1  new
0   b      0    0
1   b      1    1
2   a      2    2
3   c      3    3
4   a      4    4
5   a      5    5
6   b      6    6
\end{Verbatim}
\end{tcolorbox}
        
    \hypertarget{renaming-column-new-to}{%
\subsection{renaming column new to}\label{renaming-column-new-to}}

    \begin{tcolorbox}[breakable, size=fbox, boxrule=1pt, pad at break*=1mm,colback=cellbackground, colframe=cellborder]
\prompt{In}{incolor}{81}{\boxspacing}
\begin{Verbatim}[commandchars=\\\{\}]
\PY{n}{df2}\PY{o}{.}\PY{n}{rename}\PY{p}{(}\PY{n}{columns}\PY{o}{=}\PY{p}{\PYZob{}}\PY{l+s+s1}{\PYZsq{}}\PY{l+s+s1}{new}\PY{l+s+s1}{\PYZsq{}}\PY{p}{:}\PY{l+s+s1}{\PYZsq{}}\PY{l+s+s1}{new1}\PY{l+s+s1}{\PYZsq{}}\PY{p}{\PYZcb{}}\PY{p}{,}\PY{n}{inplace}\PY{o}{=}\PY{k+kc}{True}\PY{p}{)}
\PY{n}{df2}\PY{o}{.}\PY{n}{iloc}\PY{p}{[}\PY{l+m+mi}{2}\PY{p}{,}\PY{l+m+mi}{2}\PY{p}{]}\PY{o}{=}\PY{l+m+mi}{10}
\end{Verbatim}
\end{tcolorbox}

    \hypertarget{column-lhs-and-row-index-of-rhs}{%
\subsection{column LHS and row index of
RHS}\label{column-lhs-and-row-index-of-rhs}}

    \begin{tcolorbox}[breakable, size=fbox, boxrule=1pt, pad at break*=1mm,colback=cellbackground, colframe=cellborder]
\prompt{In}{incolor}{84}{\boxspacing}
\begin{Verbatim}[commandchars=\\\{\}]
\PY{n}{df1}\PY{o}{.}\PY{n}{join}\PY{p}{(}\PY{n}{df2}\PY{p}{,} \PY{n}{how}\PY{o}{=}\PY{l+s+s1}{\PYZsq{}}\PY{l+s+s1}{inner}\PY{l+s+s1}{\PYZsq{}}\PY{p}{,}\PY{n}{lsuffix}\PY{o}{=}\PY{l+s+s1}{\PYZsq{}}\PY{l+s+s1}{left}\PY{l+s+s1}{\PYZsq{}}\PY{p}{,}\PY{n}{on}\PY{o}{=}\PY{p}{[}\PY{l+s+s1}{\PYZsq{}}\PY{l+s+s1}{new}\PY{l+s+s1}{\PYZsq{}}\PY{p}{]}\PY{p}{)}
\end{Verbatim}
\end{tcolorbox}

            \begin{tcolorbox}[breakable, size=fbox, boxrule=.5pt, pad at break*=1mm, opacityfill=0]
\prompt{Out}{outcolor}{84}{\boxspacing}
\begin{Verbatim}[commandchars=\\\{\}]
  keyleft  data1left  new key  data1  new1
0       b          0    0   a      0     1
1       b          1    1   b      1     2
2       a          2    2   d      2    10
\end{Verbatim}
\end{tcolorbox}
        
    \hypertarget{join-one-df-with-more-than-one-dfs}{%
\subsection{Join one DF with more than one
DFs}\label{join-one-df-with-more-than-one-dfs}}

    \hypertarget{creating-a-new-df}{%
\subsubsection{creating a new DF}\label{creating-a-new-df}}

    \begin{tcolorbox}[breakable, size=fbox, boxrule=1pt, pad at break*=1mm,colback=cellbackground, colframe=cellborder]
\prompt{In}{incolor}{87}{\boxspacing}
\begin{Verbatim}[commandchars=\\\{\}]
\PY{n}{df3}\PY{o}{=}\PY{n}{df2}\PY{o}{.}\PY{n}{copy}\PY{p}{(}\PY{p}{)}
\PY{n}{df3}\PY{p}{[}\PY{l+s+s1}{\PYZsq{}}\PY{l+s+s1}{data1}\PY{l+s+s1}{\PYZsq{}}\PY{p}{]}\PY{o}{*}\PY{o}{=}\PY{l+m+mi}{4}
\end{Verbatim}
\end{tcolorbox}

    \hypertarget{renaming-its-columns}{%
\subsubsection{renaming its columns}\label{renaming-its-columns}}

    \begin{tcolorbox}[breakable, size=fbox, boxrule=1pt, pad at break*=1mm,colback=cellbackground, colframe=cellborder]
\prompt{In}{incolor}{88}{\boxspacing}
\begin{Verbatim}[commandchars=\\\{\}]
\PY{n}{df3}\PY{o}{.}\PY{n}{columns}\PY{o}{=}\PY{p}{[}\PY{l+s+s1}{\PYZsq{}}\PY{l+s+s1}{a}\PY{l+s+s1}{\PYZsq{}}\PY{p}{,}\PY{l+s+s1}{\PYZsq{}}\PY{l+s+s1}{b}\PY{l+s+s1}{\PYZsq{}}\PY{p}{,}\PY{l+s+s1}{\PYZsq{}}\PY{l+s+s1}{c}\PY{l+s+s1}{\PYZsq{}}\PY{p}{]}
\end{Verbatim}
\end{tcolorbox}

    \begin{tcolorbox}[breakable, size=fbox, boxrule=1pt, pad at break*=1mm,colback=cellbackground, colframe=cellborder]
\prompt{In}{incolor}{89}{\boxspacing}
\begin{Verbatim}[commandchars=\\\{\}]
\PY{n}{df3}\PY{p}{[}\PY{l+s+s1}{\PYZsq{}}\PY{l+s+s1}{d}\PY{l+s+s1}{\PYZsq{}}\PY{p}{]}\PY{o}{=}\PY{n}{df3}\PY{p}{[}\PY{l+s+s1}{\PYZsq{}}\PY{l+s+s1}{b}\PY{l+s+s1}{\PYZsq{}}\PY{p}{]}\PY{o}{\PYZhy{}}\PY{n}{df3}\PY{p}{[}\PY{l+s+s1}{\PYZsq{}}\PY{l+s+s1}{c}\PY{l+s+s1}{\PYZsq{}}\PY{p}{]}
\end{Verbatim}
\end{tcolorbox}

    \begin{tcolorbox}[breakable, size=fbox, boxrule=1pt, pad at break*=1mm,colback=cellbackground, colframe=cellborder]
\prompt{In}{incolor}{90}{\boxspacing}
\begin{Verbatim}[commandchars=\\\{\}]
\PY{n}{df3}
\end{Verbatim}
\end{tcolorbox}

            \begin{tcolorbox}[breakable, size=fbox, boxrule=.5pt, pad at break*=1mm, opacityfill=0]
\prompt{Out}{outcolor}{90}{\boxspacing}
\begin{Verbatim}[commandchars=\\\{\}]
   a  b   c  d
0  a  0   1 -1
1  b  4   2  2
2  d  8  10 -2
9  a  0   1 -1
\end{Verbatim}
\end{tcolorbox}
        
    \hypertarget{renaming-columns-of-df2-to-remove-same-column-names-as-in-other-df-df1}{%
\subsubsection{renaming columns of DF2 to remove same column names as in
other DF
df1}\label{renaming-columns-of-df2-to-remove-same-column-names-as-in-other-df-df1}}

    \begin{tcolorbox}[breakable, size=fbox, boxrule=1pt, pad at break*=1mm,colback=cellbackground, colframe=cellborder]
\prompt{In}{incolor}{96}{\boxspacing}
\begin{Verbatim}[commandchars=\\\{\}]
\PY{n}{df2}\PY{o}{.}\PY{n}{columns}\PY{o}{=}\PY{p}{[}\PY{l+s+s1}{\PYZsq{}}\PY{l+s+s1}{x}\PY{l+s+s1}{\PYZsq{}}\PY{p}{,}\PY{l+s+s1}{\PYZsq{}}\PY{l+s+s1}{y}\PY{l+s+s1}{\PYZsq{}}\PY{p}{,}\PY{l+s+s1}{\PYZsq{}}\PY{l+s+s1}{z}\PY{l+s+s1}{\PYZsq{}}\PY{p}{]}
\PY{n}{df2}
\end{Verbatim}
\end{tcolorbox}

            \begin{tcolorbox}[breakable, size=fbox, boxrule=.5pt, pad at break*=1mm, opacityfill=0]
\prompt{Out}{outcolor}{96}{\boxspacing}
\begin{Verbatim}[commandchars=\\\{\}]
   x  y   z
0  a  0   1
1  b  1   2
2  d  2  10
9  a  0   1
\end{Verbatim}
\end{tcolorbox}
        
    \begin{tcolorbox}[breakable, size=fbox, boxrule=1pt, pad at break*=1mm,colback=cellbackground, colframe=cellborder]
\prompt{In}{incolor}{95}{\boxspacing}
\begin{Verbatim}[commandchars=\\\{\}]
\PY{n}{df1}
\end{Verbatim}
\end{tcolorbox}

            \begin{tcolorbox}[breakable, size=fbox, boxrule=.5pt, pad at break*=1mm, opacityfill=0]
\prompt{Out}{outcolor}{95}{\boxspacing}
\begin{Verbatim}[commandchars=\\\{\}]
  key  data1  new
0   b      0    0
1   b      1    1
2   a      2    2
3   c      3    3
4   a      4    4
5   a      5    5
6   b      6    6
\end{Verbatim}
\end{tcolorbox}
        
    \begin{tcolorbox}[breakable, size=fbox, boxrule=1pt, pad at break*=1mm,colback=cellbackground, colframe=cellborder]
\prompt{In}{incolor}{94}{\boxspacing}
\begin{Verbatim}[commandchars=\\\{\}]
\PY{n}{df3}
\end{Verbatim}
\end{tcolorbox}

            \begin{tcolorbox}[breakable, size=fbox, boxrule=.5pt, pad at break*=1mm, opacityfill=0]
\prompt{Out}{outcolor}{94}{\boxspacing}
\begin{Verbatim}[commandchars=\\\{\}]
   a  b   c  d
0  a  0   1 -1
1  b  4   2  2
2  d  8  10 -2
9  a  0   1 -1
\end{Verbatim}
\end{tcolorbox}
        
    \begin{tcolorbox}[breakable, size=fbox, boxrule=1pt, pad at break*=1mm,colback=cellbackground, colframe=cellborder]
\prompt{In}{incolor}{48}{\boxspacing}
\begin{Verbatim}[commandchars=\\\{\}]
\PY{n}{df3}\PY{o}{=}\PY{n}{df3}\PY{o}{.}\PY{n}{append}\PY{p}{(}\PY{n}{df3}\PY{o}{.}\PY{n}{iloc}\PY{p}{[}\PY{l+m+mi}{3}\PY{p}{,}\PY{p}{:}\PY{p}{]}\PY{p}{)}
\end{Verbatim}
\end{tcolorbox}

    \hypertarget{now-joining-three-dfs-note-on-the-basis-of-common-row-index}{%
\subsubsection{now joining three DFs : note on the basis of common row
index}\label{now-joining-three-dfs-note-on-the-basis-of-common-row-index}}

    \begin{tcolorbox}[breakable, size=fbox, boxrule=1pt, pad at break*=1mm,colback=cellbackground, colframe=cellborder]
\prompt{In}{incolor}{93}{\boxspacing}
\begin{Verbatim}[commandchars=\\\{\}]
\PY{n}{df1}\PY{o}{.}\PY{n}{join}\PY{p}{(}\PY{p}{[}\PY{n}{df2}\PY{p}{,}\PY{n}{df3}\PY{p}{]}\PY{p}{)}
\end{Verbatim}
\end{tcolorbox}

            \begin{tcolorbox}[breakable, size=fbox, boxrule=.5pt, pad at break*=1mm, opacityfill=0]
\prompt{Out}{outcolor}{93}{\boxspacing}
\begin{Verbatim}[commandchars=\\\{\}]
  key  data1  new    x    y     z    a    b     c    d
0   b    0.0  0.0    a  0.0   1.0    a  0.0   1.0 -1.0
1   b    1.0  1.0    b  1.0   2.0    b  4.0   2.0  2.0
2   a    2.0  2.0    d  2.0  10.0    d  8.0  10.0 -2.0
3   c    3.0  3.0  NaN  NaN   NaN  NaN  NaN   NaN  NaN
4   a    4.0  4.0  NaN  NaN   NaN  NaN  NaN   NaN  NaN
5   a    5.0  5.0  NaN  NaN   NaN  NaN  NaN   NaN  NaN
6   b    6.0  6.0  NaN  NaN   NaN  NaN  NaN   NaN  NaN
\end{Verbatim}
\end{tcolorbox}
        
    \begin{tcolorbox}[breakable, size=fbox, boxrule=1pt, pad at break*=1mm,colback=cellbackground, colframe=cellborder]
\prompt{In}{incolor}{97}{\boxspacing}
\begin{Verbatim}[commandchars=\\\{\}]
\PY{n}{df2}\PY{o}{.}\PY{n}{join}\PY{p}{(}\PY{p}{[}\PY{n}{df1}\PY{p}{,}\PY{n}{df3}\PY{p}{]}\PY{p}{)}
\end{Verbatim}
\end{tcolorbox}

            \begin{tcolorbox}[breakable, size=fbox, boxrule=.5pt, pad at break*=1mm, opacityfill=0]
\prompt{Out}{outcolor}{97}{\boxspacing}
\begin{Verbatim}[commandchars=\\\{\}]
   x    y     z  key  data1  new  a    b     c    d
0  a  0.0   1.0    b    0.0  0.0  a  0.0   1.0 -1.0
1  b  1.0   2.0    b    1.0  1.0  b  4.0   2.0  2.0
2  d  2.0  10.0    a    2.0  2.0  d  8.0  10.0 -2.0
9  a  0.0   1.0  NaN    NaN  NaN  a  0.0   1.0 -1.0
\end{Verbatim}
\end{tcolorbox}
        
    \begin{tcolorbox}[breakable, size=fbox, boxrule=1pt, pad at break*=1mm,colback=cellbackground, colframe=cellborder]
\prompt{In}{incolor}{51}{\boxspacing}
\begin{Verbatim}[commandchars=\\\{\}]
\PY{n}{df2}
\end{Verbatim}
\end{tcolorbox}

            \begin{tcolorbox}[breakable, size=fbox, boxrule=.5pt, pad at break*=1mm, opacityfill=0]
\prompt{Out}{outcolor}{51}{\boxspacing}
\begin{Verbatim}[commandchars=\\\{\}]
   x  y   z
0  a  0   1
1  b  1   2
2  d  2  10
4  b  0   0
9  a  0   1
\end{Verbatim}
\end{tcolorbox}
        
    \hypertarget{join-and-merge}{%
\section{join() and merge()}\label{join-and-merge}}

Commonality: 1. used to combines two dataframes

Difference:

\begin{enumerate}
\def\labelenumi{\arabic{enumi}.}
\tightlist
\item
  join method combines two dataframes on the basis index values wheras
  the versatile merge method allows to specify columns beside the index
  to join on for both dataframesList item
\item
  join can be used to merge more than two DFs at a time, but not
  feasible with merge
\end{enumerate}

    \hypertarget{make-sure-to-use-proper-stmt-to-copy-two-dfs}{%
\section{Make sure to use proper stmt to copy two
DFs}\label{make-sure-to-use-proper-stmt-to-copy-two-dfs}}

    \begin{tcolorbox}[breakable, size=fbox, boxrule=1pt, pad at break*=1mm,colback=cellbackground, colframe=cellborder]
\prompt{In}{incolor}{98}{\boxspacing}
\begin{Verbatim}[commandchars=\\\{\}]
\PY{n}{DF2}
\end{Verbatim}
\end{tcolorbox}

            \begin{tcolorbox}[breakable, size=fbox, boxrule=.5pt, pad at break*=1mm, opacityfill=0]
\prompt{Out}{outcolor}{98}{\boxspacing}
\begin{Verbatim}[commandchars=\\\{\}]
                 Corepaper  SEC  DSE
Course Semester
PSCS   I                 3  NIL  NIL
       II                3  NIL  NIL
       III               4    1  NIL
       IV                4    1  NIL
       V                 4    1    1
       VI                4    1    1
CSHons I                 4    2  NIL
       II                4    2  NIL
       III               4    2  NIL
       IV                4    2  NIL
       V                 4  NIL    2
       VI                4  NIL    2
Bcom   I                 4    2  NIL
       II                4    2  NIL
       III               4    3    1
       IV                4    3  NIL
lifesc I                 5  NIL    1
       II                5    0    1
\end{Verbatim}
\end{tcolorbox}
        
    \hypertarget{what-is-done-using-following-statment}{%
\subsubsection{what is done using following
statment?}\label{what-is-done-using-following-statment}}

    \begin{tcolorbox}[breakable, size=fbox, boxrule=1pt, pad at break*=1mm,colback=cellbackground, colframe=cellborder]
\prompt{In}{incolor}{60}{\boxspacing}
\begin{Verbatim}[commandchars=\\\{\}]
\PY{n}{DF3}\PY{o}{=}\PY{n}{DF2}
\end{Verbatim}
\end{tcolorbox}

    \begin{tcolorbox}[breakable, size=fbox, boxrule=1pt, pad at break*=1mm,colback=cellbackground, colframe=cellborder]
\prompt{In}{incolor}{61}{\boxspacing}
\begin{Verbatim}[commandchars=\\\{\}]
\PY{n}{DF4}\PY{o}{=}\PY{n}{DF2}\PY{o}{.}\PY{n}{copy}\PY{p}{(}\PY{n}{deep}\PY{o}{=}\PY{k+kc}{True}\PY{p}{)}
\end{Verbatim}
\end{tcolorbox}

    \begin{tcolorbox}[breakable, size=fbox, boxrule=1pt, pad at break*=1mm,colback=cellbackground, colframe=cellborder]
\prompt{In}{incolor}{62}{\boxspacing}
\begin{Verbatim}[commandchars=\\\{\}]
\PY{n}{DF3}\PY{o}{.}\PY{n}{rename}\PY{p}{(}\PY{n}{index}\PY{o}{=}\PY{p}{\PYZob{}}\PY{l+s+s1}{\PYZsq{}}\PY{l+s+s1}{lifesc}\PY{l+s+s1}{\PYZsq{}}\PY{p}{:}\PY{l+s+s1}{\PYZsq{}}\PY{l+s+s1}{ZooHons}\PY{l+s+s1}{\PYZsq{}}\PY{p}{\PYZcb{}}\PY{p}{,}\PY{n}{inplace}\PY{o}{=}\PY{l+s+s1}{\PYZsq{}}\PY{l+s+s1}{True}\PY{l+s+s1}{\PYZsq{}}\PY{p}{)}
\end{Verbatim}
\end{tcolorbox}

    \begin{tcolorbox}[breakable, size=fbox, boxrule=1pt, pad at break*=1mm,colback=cellbackground, colframe=cellborder]
\prompt{In}{incolor}{63}{\boxspacing}
\begin{Verbatim}[commandchars=\\\{\}]
\PY{n}{DF2}
\end{Verbatim}
\end{tcolorbox}

            \begin{tcolorbox}[breakable, size=fbox, boxrule=.5pt, pad at break*=1mm, opacityfill=0]
\prompt{Out}{outcolor}{63}{\boxspacing}
\begin{Verbatim}[commandchars=\\\{\}]
                  Corepaper  SEC  DSE
Course  Semester
PSCS    I                 3  NIL  NIL
        II                3  NIL  NIL
        III               4    1  NIL
        IV                4    1  NIL
        V                 4    1    1
        VI                4    1    1
CSHons  I                 4    2  NIL
        II                4    2  NIL
        III               4    2  NIL
        IV                4    2  NIL
        V                 4  NIL    2
        VI                4  NIL    2
Bcom    I                 4    2  NIL
        II                4    2  NIL
        III               4    3    1
        IV                4    3  NIL
ZooHons I                 5  NIL    1
        II                5    0    1
\end{Verbatim}
\end{tcolorbox}
        
    \begin{tcolorbox}[breakable, size=fbox, boxrule=1pt, pad at break*=1mm,colback=cellbackground, colframe=cellborder]
\prompt{In}{incolor}{64}{\boxspacing}
\begin{Verbatim}[commandchars=\\\{\}]
\PY{n}{DF3}\PY{o}{.}\PY{n}{loc}\PY{p}{[}\PY{p}{[}\PY{l+s+s1}{\PYZsq{}}\PY{l+s+s1}{Bcom}\PY{l+s+s1}{\PYZsq{}}\PY{p}{]}\PY{p}{]}
\end{Verbatim}
\end{tcolorbox}

            \begin{tcolorbox}[breakable, size=fbox, boxrule=.5pt, pad at break*=1mm, opacityfill=0]
\prompt{Out}{outcolor}{64}{\boxspacing}
\begin{Verbatim}[commandchars=\\\{\}]
                 Corepaper SEC  DSE
Course Semester
Bcom   I                 4   2  NIL
       II                4   2  NIL
       III               4   3    1
       IV                4   3  NIL
\end{Verbatim}
\end{tcolorbox}
        
    \begin{tcolorbox}[breakable, size=fbox, boxrule=1pt, pad at break*=1mm,colback=cellbackground, colframe=cellborder]
\prompt{In}{incolor}{ }{\boxspacing}
\begin{Verbatim}[commandchars=\\\{\}]
\PY{n}{DF2}
\end{Verbatim}
\end{tcolorbox}

            \begin{tcolorbox}[breakable, size=fbox, boxrule=.5pt, pad at break*=1mm, opacityfill=0]
\prompt{Out}{outcolor}{ }{\boxspacing}
\begin{Verbatim}[commandchars=\\\{\}]
                  corepaper  SEC  DSE newcol
course  Semester
PSCS    I                 3  NIL  NIL     VI
        II                3  NIL  NIL     VI
        III               4    1  NIL      I
        IV                4    1  NIL      I
        V                 4    1    1      I
        VI                4    1    1      I
CSHons  I                 4    2  NIL     II
        II                4    2  NIL     II
        III               4    2  NIL     II
        IV                4    2  NIL     II
        V                 4  NIL    2     VI
        VI                4  NIL    2     VI
Bcom    I                 4    2  NIL     II
        II                4    2  NIL     II
        III               4    3    1    III
        IV                4    3    4    III
ZooHons I                 5  NIL    1     VI
        II                5    0    1    NaN
\end{Verbatim}
\end{tcolorbox}
        
    \hypertarget{simple-assignment-refers-to-original-memory-address-whereas-copy-makes-a-new-copy-so-that-changes-in-the-copied-object-are-not-reflected-to-original-object}{%
\subsubsection{simple =(assignment) refers to original memory address
whereas copy makes a new copy so that changes in the copied object are
not reflected to original
object}\label{simple-assignment-refers-to-original-memory-address-whereas-copy-makes-a-new-copy-so-that-changes-in-the-copied-object-are-not-reflected-to-original-object}}

    \hypertarget{concatenate-concatenating-along-axis-for-bindingstacking}{%
\section{concatenate(): Concatenating along axis for
binding/stacking}\label{concatenate-concatenating-along-axis-for-bindingstacking}}

\begin{itemize}
\tightlist
\item
  if axis=0 then append in other DF
\item
  if axis=1 then append coulmns for matched index
\end{itemize}

    \begin{tcolorbox}[breakable, size=fbox, boxrule=1pt, pad at break*=1mm,colback=cellbackground, colframe=cellborder]
\prompt{In}{incolor}{99}{\boxspacing}
\begin{Verbatim}[commandchars=\\\{\}]
\PY{n}{df1}
\end{Verbatim}
\end{tcolorbox}

            \begin{tcolorbox}[breakable, size=fbox, boxrule=.5pt, pad at break*=1mm, opacityfill=0]
\prompt{Out}{outcolor}{99}{\boxspacing}
\begin{Verbatim}[commandchars=\\\{\}]
  key  data1  new
0   b      0    0
1   b      1    1
2   a      2    2
3   c      3    3
4   a      4    4
5   a      5    5
6   b      6    6
\end{Verbatim}
\end{tcolorbox}
        
    \begin{tcolorbox}[breakable, size=fbox, boxrule=1pt, pad at break*=1mm,colback=cellbackground, colframe=cellborder]
\prompt{In}{incolor}{100}{\boxspacing}
\begin{Verbatim}[commandchars=\\\{\}]
\PY{n}{df2}
\end{Verbatim}
\end{tcolorbox}

            \begin{tcolorbox}[breakable, size=fbox, boxrule=.5pt, pad at break*=1mm, opacityfill=0]
\prompt{Out}{outcolor}{100}{\boxspacing}
\begin{Verbatim}[commandchars=\\\{\}]
   x  y   z
0  a  0   1
1  b  1   2
2  d  2  10
9  a  0   1
\end{Verbatim}
\end{tcolorbox}
        
    \begin{tcolorbox}[breakable, size=fbox, boxrule=1pt, pad at break*=1mm,colback=cellbackground, colframe=cellborder]
\prompt{In}{incolor}{101}{\boxspacing}
\begin{Verbatim}[commandchars=\\\{\}]
\PY{n}{df2}\PY{o}{.}\PY{n}{columns}\PY{o}{=}\PY{n}{df1}\PY{o}{.}\PY{n}{columns}
\end{Verbatim}
\end{tcolorbox}

    \begin{tcolorbox}[breakable, size=fbox, boxrule=1pt, pad at break*=1mm,colback=cellbackground, colframe=cellborder]
\prompt{In}{incolor}{102}{\boxspacing}
\begin{Verbatim}[commandchars=\\\{\}]
\PY{n}{pd}\PY{o}{.}\PY{n}{concat}\PY{p}{(}\PY{p}{[}\PY{n}{df1}\PY{p}{,}\PY{n}{df2}\PY{p}{]}\PY{p}{)}
\end{Verbatim}
\end{tcolorbox}

            \begin{tcolorbox}[breakable, size=fbox, boxrule=.5pt, pad at break*=1mm, opacityfill=0]
\prompt{Out}{outcolor}{102}{\boxspacing}
\begin{Verbatim}[commandchars=\\\{\}]
  key  data1  new
0   b      0    0
1   b      1    1
2   a      2    2
3   c      3    3
4   a      4    4
5   a      5    5
6   b      6    6
0   a      0    1
1   b      1    2
2   d      2   10
9   a      0    1
\end{Verbatim}
\end{tcolorbox}
        
    \begin{tcolorbox}[breakable, size=fbox, boxrule=1pt, pad at break*=1mm,colback=cellbackground, colframe=cellborder]
\prompt{In}{incolor}{103}{\boxspacing}
\begin{Verbatim}[commandchars=\\\{\}]
\PY{n}{pd}\PY{o}{.}\PY{n}{concat}\PY{p}{(}\PY{p}{[}\PY{n}{df1}\PY{p}{,}\PY{n}{df1}\PY{p}{]}\PY{p}{,}\PY{n}{ignore\PYZus{}index}\PY{o}{=}\PY{k+kc}{True}\PY{p}{)}
\end{Verbatim}
\end{tcolorbox}

            \begin{tcolorbox}[breakable, size=fbox, boxrule=.5pt, pad at break*=1mm, opacityfill=0]
\prompt{Out}{outcolor}{103}{\boxspacing}
\begin{Verbatim}[commandchars=\\\{\}]
   key  data1  new
0    b      0    0
1    b      1    1
2    a      2    2
3    c      3    3
4    a      4    4
5    a      5    5
6    b      6    6
7    b      0    0
8    b      1    1
9    a      2    2
10   c      3    3
11   a      4    4
12   a      5    5
13   b      6    6
\end{Verbatim}
\end{tcolorbox}
        
    \begin{tcolorbox}[breakable, size=fbox, boxrule=1pt, pad at break*=1mm,colback=cellbackground, colframe=cellborder]
\prompt{In}{incolor}{70}{\boxspacing}
\begin{Verbatim}[commandchars=\\\{\}]
\PY{n}{df1}
\end{Verbatim}
\end{tcolorbox}

            \begin{tcolorbox}[breakable, size=fbox, boxrule=.5pt, pad at break*=1mm, opacityfill=0]
\prompt{Out}{outcolor}{70}{\boxspacing}
\begin{Verbatim}[commandchars=\\\{\}]
  key  data1  new
0   b      0    0
1   b      1    1
2   a      2    2
3   c      3    3
4   a      4    4
5   a      5    5
6   b      6    6
\end{Verbatim}
\end{tcolorbox}
        
    \begin{tcolorbox}[breakable, size=fbox, boxrule=1pt, pad at break*=1mm,colback=cellbackground, colframe=cellborder]
\prompt{In}{incolor}{73}{\boxspacing}
\begin{Verbatim}[commandchars=\\\{\}]
\PY{n}{df2}
\end{Verbatim}
\end{tcolorbox}

            \begin{tcolorbox}[breakable, size=fbox, boxrule=.5pt, pad at break*=1mm, opacityfill=0]
\prompt{Out}{outcolor}{73}{\boxspacing}
\begin{Verbatim}[commandchars=\\\{\}]
  key  data1  new
0   a      0    1
1   b      1    2
2   d      2   10
4   b      0    0
9   a      0    1
\end{Verbatim}
\end{tcolorbox}
        
    \begin{tcolorbox}[breakable, size=fbox, boxrule=1pt, pad at break*=1mm,colback=cellbackground, colframe=cellborder]
\prompt{In}{incolor}{ }{\boxspacing}
\begin{Verbatim}[commandchars=\\\{\}]
\PY{n}{df1}\PY{o}{.}\PY{n}{columns}\PY{o}{.}\PY{n}{values}
\end{Verbatim}
\end{tcolorbox}

            \begin{tcolorbox}[breakable, size=fbox, boxrule=.5pt, pad at break*=1mm, opacityfill=0]
\prompt{Out}{outcolor}{ }{\boxspacing}
\begin{Verbatim}[commandchars=\\\{\}]
array(['key', 'data1', 'new'], dtype=object)
\end{Verbatim}
\end{tcolorbox}
        
    \hypertarget{columns-are-added-on-the-basis-of-matching-index-as-nan-is-added-axis1}{%
\subsubsection{columns are added on the basis of matching index as Nan
is added
(axis=1)}\label{columns-are-added-on-the-basis-of-matching-index-as-nan-is-added-axis1}}

    \begin{tcolorbox}[breakable, size=fbox, boxrule=1pt, pad at break*=1mm,colback=cellbackground, colframe=cellborder]
\prompt{In}{incolor}{104}{\boxspacing}
\begin{Verbatim}[commandchars=\\\{\}]
\PY{n}{pd}\PY{o}{.}\PY{n}{concat}\PY{p}{(}\PY{p}{[}\PY{n}{df1}\PY{p}{,}\PY{n}{df2}\PY{p}{]}\PY{p}{,}\PY{n}{axis}\PY{o}{=}\PY{l+m+mi}{1}\PY{p}{)}
\end{Verbatim}
\end{tcolorbox}

            \begin{tcolorbox}[breakable, size=fbox, boxrule=.5pt, pad at break*=1mm, opacityfill=0]
\prompt{Out}{outcolor}{104}{\boxspacing}
\begin{Verbatim}[commandchars=\\\{\}]
   key  data1  new  key  data1   new
0    b    0.0  0.0    a    0.0   1.0
1    b    1.0  1.0    b    1.0   2.0
2    a    2.0  2.0    d    2.0  10.0
3    c    3.0  3.0  NaN    NaN   NaN
4    a    4.0  4.0  NaN    NaN   NaN
5    a    5.0  5.0  NaN    NaN   NaN
6    b    6.0  6.0  NaN    NaN   NaN
9  NaN    NaN  NaN    a    0.0   1.0
\end{Verbatim}
\end{tcolorbox}
        
    \hypertarget{distinuishing-between-columns-of-two-dfs-and-putting-a-level}{%
\subsubsection{distinuishing between columns of two DFs and putting a
level}\label{distinuishing-between-columns-of-two-dfs-and-putting-a-level}}

    \begin{tcolorbox}[breakable, size=fbox, boxrule=1pt, pad at break*=1mm,colback=cellbackground, colframe=cellborder]
\prompt{In}{incolor}{105}{\boxspacing}
\begin{Verbatim}[commandchars=\\\{\}]
\PY{n}{dftemp}\PY{o}{=}\PY{n}{pd}\PY{o}{.}\PY{n}{concat}\PY{p}{(}\PY{p}{[}\PY{n}{df1}\PY{p}{,} \PY{n}{df2}\PY{p}{]}\PY{p}{,} \PY{n}{axis}\PY{o}{=}\PY{l+m+mi}{1}\PY{p}{,} \PY{n}{keys}\PY{o}{=}\PY{p}{[}\PY{l+s+s1}{\PYZsq{}}\PY{l+s+s1}{level1}\PY{l+s+s1}{\PYZsq{}}\PY{p}{,} \PY{l+s+s1}{\PYZsq{}}\PY{l+s+s1}{level2}\PY{l+s+s1}{\PYZsq{}}\PY{p}{]}\PY{p}{)}
\end{Verbatim}
\end{tcolorbox}

    \begin{tcolorbox}[breakable, size=fbox, boxrule=1pt, pad at break*=1mm,colback=cellbackground, colframe=cellborder]
\prompt{In}{incolor}{107}{\boxspacing}
\begin{Verbatim}[commandchars=\\\{\}]
\PY{n}{dftemp}\PY{o}{.}\PY{n}{columns}\PY{o}{.}\PY{n}{levels}
\end{Verbatim}
\end{tcolorbox}

            \begin{tcolorbox}[breakable, size=fbox, boxrule=.5pt, pad at break*=1mm, opacityfill=0]
\prompt{Out}{outcolor}{107}{\boxspacing}
\begin{Verbatim}[commandchars=\\\{\}]
FrozenList([['level1', 'level2'], ['key', 'data1', 'new']])
\end{Verbatim}
\end{tcolorbox}
        
    \begin{tcolorbox}[breakable, size=fbox, boxrule=1pt, pad at break*=1mm,colback=cellbackground, colframe=cellborder]
\prompt{In}{incolor}{106}{\boxspacing}
\begin{Verbatim}[commandchars=\\\{\}]
\PY{n}{dftemp}
\end{Verbatim}
\end{tcolorbox}

            \begin{tcolorbox}[breakable, size=fbox, boxrule=.5pt, pad at break*=1mm, opacityfill=0]
\prompt{Out}{outcolor}{106}{\boxspacing}
\begin{Verbatim}[commandchars=\\\{\}]
  level1            level2
     key data1  new    key data1   new
0      b   0.0  0.0      a   0.0   1.0
1      b   1.0  1.0      b   1.0   2.0
2      a   2.0  2.0      d   2.0  10.0
3      c   3.0  3.0    NaN   NaN   NaN
4      a   4.0  4.0    NaN   NaN   NaN
5      a   5.0  5.0    NaN   NaN   NaN
6      b   6.0  6.0    NaN   NaN   NaN
9    NaN   NaN  NaN      a   0.0   1.0
\end{Verbatim}
\end{tcolorbox}
        
    \hypertarget{concatenate-two-dfs-side-by-side-by-simply-as-for-strings-using-dictionary}{%
\subsubsection{concatenate two DFs side by side by simply as for strings
using
dictionary}\label{concatenate-two-dfs-side-by-side-by-simply-as-for-strings-using-dictionary}}

    \begin{tcolorbox}[breakable, size=fbox, boxrule=1pt, pad at break*=1mm,colback=cellbackground, colframe=cellborder]
\prompt{In}{incolor}{108}{\boxspacing}
\begin{Verbatim}[commandchars=\\\{\}]
\PY{n}{dftemp1}\PY{o}{=}\PY{n}{pd}\PY{o}{.}\PY{n}{concat}\PY{p}{(}\PY{p}{\PYZob{}}\PY{l+s+s1}{\PYZsq{}}\PY{l+s+s1}{level1}\PY{l+s+s1}{\PYZsq{}}\PY{p}{:} \PY{n}{df1}\PY{p}{,} \PY{l+s+s1}{\PYZsq{}}\PY{l+s+s1}{level2}\PY{l+s+s1}{\PYZsq{}}\PY{p}{:} \PY{n}{df2}\PY{p}{\PYZcb{}}\PY{p}{,} \PY{n}{axis}\PY{o}{=}\PY{l+m+mi}{1}\PY{p}{)}
\end{Verbatim}
\end{tcolorbox}

    \begin{tcolorbox}[breakable, size=fbox, boxrule=1pt, pad at break*=1mm,colback=cellbackground, colframe=cellborder]
\prompt{In}{incolor}{86}{\boxspacing}
\begin{Verbatim}[commandchars=\\\{\}]
\PY{n}{dftemp}\PY{o}{.}\PY{n}{dtypes}
\end{Verbatim}
\end{tcolorbox}

            \begin{tcolorbox}[breakable, size=fbox, boxrule=.5pt, pad at break*=1mm, opacityfill=0]
\prompt{Out}{outcolor}{86}{\boxspacing}
\begin{Verbatim}[commandchars=\\\{\}]
level1  key       object
        data1    float64
        new      float64
level2  key       object
        data1    float64
        new      float64
dtype: object
\end{Verbatim}
\end{tcolorbox}
        
    \begin{tcolorbox}[breakable, size=fbox, boxrule=1pt, pad at break*=1mm,colback=cellbackground, colframe=cellborder]
\prompt{In}{incolor}{109}{\boxspacing}
\begin{Verbatim}[commandchars=\\\{\}]
\PY{n}{dftemp1}
\end{Verbatim}
\end{tcolorbox}

            \begin{tcolorbox}[breakable, size=fbox, boxrule=.5pt, pad at break*=1mm, opacityfill=0]
\prompt{Out}{outcolor}{109}{\boxspacing}
\begin{Verbatim}[commandchars=\\\{\}]
  level1            level2
     key data1  new    key data1   new
0      b   0.0  0.0      a   0.0   1.0
1      b   1.0  1.0      b   1.0   2.0
2      a   2.0  2.0      d   2.0  10.0
3      c   3.0  3.0    NaN   NaN   NaN
4      a   4.0  4.0    NaN   NaN   NaN
5      a   5.0  5.0    NaN   NaN   NaN
6      b   6.0  6.0    NaN   NaN   NaN
9    NaN   NaN  NaN      a   0.0   1.0
\end{Verbatim}
\end{tcolorbox}
        
    \hypertarget{why-comparing-nan-values-result-in-false}{%
\subsubsection{why comparing Nan values result in
False?}\label{why-comparing-nan-values-result-in-false}}

    \begin{tcolorbox}[breakable, size=fbox, boxrule=1pt, pad at break*=1mm,colback=cellbackground, colframe=cellborder]
\prompt{In}{incolor}{110}{\boxspacing}
\begin{Verbatim}[commandchars=\\\{\}]
\PY{n}{dftemp}\PY{o}{==}\PY{n}{dftemp1}
\end{Verbatim}
\end{tcolorbox}

            \begin{tcolorbox}[breakable, size=fbox, boxrule=.5pt, pad at break*=1mm, opacityfill=0]
\prompt{Out}{outcolor}{110}{\boxspacing}
\begin{Verbatim}[commandchars=\\\{\}]
  level1               level2
     key  data1    new    key  data1    new
0   True   True   True   True   True   True
1   True   True   True   True   True   True
2   True   True   True   True   True   True
3   True   True   True  False  False  False
4   True   True   True  False  False  False
5   True   True   True  False  False  False
6   True   True   True  False  False  False
9  False  False  False   True   True   True
\end{Verbatim}
\end{tcolorbox}
        
    \hypertarget{naming-to-coulmn-levels}{%
\subsection{naming to coulmn levels}\label{naming-to-coulmn-levels}}

    \begin{tcolorbox}[breakable, size=fbox, boxrule=1pt, pad at break*=1mm,colback=cellbackground, colframe=cellborder]
\prompt{In}{incolor}{111}{\boxspacing}
\begin{Verbatim}[commandchars=\\\{\}]
\PY{n}{pd}\PY{o}{.}\PY{n}{concat}\PY{p}{(}\PY{p}{[}\PY{n}{df1}\PY{p}{,} \PY{n}{df2}\PY{p}{]}\PY{p}{,} \PY{n}{axis}\PY{o}{=}\PY{l+m+mi}{1}\PY{p}{,} \PY{n}{keys}\PY{o}{=}\PY{p}{[}\PY{l+s+s1}{\PYZsq{}}\PY{l+s+s1}{level1}\PY{l+s+s1}{\PYZsq{}}\PY{p}{,} \PY{l+s+s1}{\PYZsq{}}\PY{l+s+s1}{level2}\PY{l+s+s1}{\PYZsq{}}\PY{p}{]}\PY{p}{,}
          \PY{n}{names}\PY{o}{=}\PY{p}{[}\PY{l+s+s1}{\PYZsq{}}\PY{l+s+s1}{upper}\PY{l+s+s1}{\PYZsq{}}\PY{p}{,} \PY{l+s+s1}{\PYZsq{}}\PY{l+s+s1}{lower}\PY{l+s+s1}{\PYZsq{}}\PY{p}{]}\PY{p}{)}
\end{Verbatim}
\end{tcolorbox}

            \begin{tcolorbox}[breakable, size=fbox, boxrule=.5pt, pad at break*=1mm, opacityfill=0]
\prompt{Out}{outcolor}{111}{\boxspacing}
\begin{Verbatim}[commandchars=\\\{\}]
upper level1            level2
lower    key data1  new    key data1   new
0          b   0.0  0.0      a   0.0   1.0
1          b   1.0  1.0      b   1.0   2.0
2          a   2.0  2.0      d   2.0  10.0
3          c   3.0  3.0    NaN   NaN   NaN
4          a   4.0  4.0    NaN   NaN   NaN
5          a   5.0  5.0    NaN   NaN   NaN
6          b   6.0  6.0    NaN   NaN   NaN
9        NaN   NaN  NaN      a   0.0   1.0
\end{Verbatim}
\end{tcolorbox}
        
    \begin{tcolorbox}[breakable, size=fbox, boxrule=1pt, pad at break*=1mm,colback=cellbackground, colframe=cellborder]
\prompt{In}{incolor}{88}{\boxspacing}
\begin{Verbatim}[commandchars=\\\{\}]
\PY{n}{df1}
\end{Verbatim}
\end{tcolorbox}

            \begin{tcolorbox}[breakable, size=fbox, boxrule=.5pt, pad at break*=1mm, opacityfill=0]
\prompt{Out}{outcolor}{88}{\boxspacing}
\begin{Verbatim}[commandchars=\\\{\}]
  key  data1  new
0   b      0    0
1   b      1    1
2   a      2    2
3   c      3    3
4   a      4    4
5   a      5    5
6   b      6    6
\end{Verbatim}
\end{tcolorbox}
        
    \begin{tcolorbox}[breakable, size=fbox, boxrule=1pt, pad at break*=1mm,colback=cellbackground, colframe=cellborder]
\prompt{In}{incolor}{89}{\boxspacing}
\begin{Verbatim}[commandchars=\\\{\}]
\PY{n}{df2}
\end{Verbatim}
\end{tcolorbox}

            \begin{tcolorbox}[breakable, size=fbox, boxrule=.5pt, pad at break*=1mm, opacityfill=0]
\prompt{Out}{outcolor}{89}{\boxspacing}
\begin{Verbatim}[commandchars=\\\{\}]
  key  data1  new
0   a      0    1
1   b      1    2
2   d      2   10
4   b      0    0
9   a      0    1
\end{Verbatim}
\end{tcolorbox}
        
    \hypertarget{queries-row-index-of-the-dfs-are-rollnumer}{%
\section{queries: row index of the DFs are
rollnumer}\label{queries-row-index-of-the-dfs-are-rollnumer}}

\begin{itemize}
\item
  Find students names appearing in both tests (intersection)
\item
  Find all the students appearing in either of the tests (union)
\end{itemize}

    \hypertarget{find-the-output}{%
\subsection{Find the output?}\label{find-the-output}}

    \begin{tcolorbox}[breakable, size=fbox, boxrule=1pt, pad at break*=1mm,colback=cellbackground, colframe=cellborder]
\prompt{In}{incolor}{112}{\boxspacing}
\begin{Verbatim}[commandchars=\\\{\}]
\PY{n}{S}\PY{o}{=}\PY{n}{pd}\PY{o}{.}\PY{n}{merge}\PY{p}{(}\PY{n}{df1}\PY{p}{,}\PY{n}{df2}\PY{p}{,}\PY{n}{how}\PY{o}{=}\PY{l+s+s1}{\PYZsq{}}\PY{l+s+s1}{outer}\PY{l+s+s1}{\PYZsq{}}\PY{p}{)}\PY{p}{[}\PY{l+s+s1}{\PYZsq{}}\PY{l+s+s1}{key}\PY{l+s+s1}{\PYZsq{}}\PY{p}{]}
\end{Verbatim}
\end{tcolorbox}

    \begin{tcolorbox}[breakable, size=fbox, boxrule=1pt, pad at break*=1mm,colback=cellbackground, colframe=cellborder]
\prompt{In}{incolor}{91}{\boxspacing}
\begin{Verbatim}[commandchars=\\\{\}]
\PY{n}{S}\PY{o}{.}\PY{n}{unique}\PY{p}{(}\PY{p}{)}
\end{Verbatim}
\end{tcolorbox}

            \begin{tcolorbox}[breakable, size=fbox, boxrule=.5pt, pad at break*=1mm, opacityfill=0]
\prompt{Out}{outcolor}{91}{\boxspacing}
\begin{Verbatim}[commandchars=\\\{\}]
array(['b', 'a', 'c', 'd'], dtype=object)
\end{Verbatim}
\end{tcolorbox}
        
    \hypertarget{stack-unstack-of-multilevel-indexing}{%
\section{stack() unstack() of Multilevel
indexing}\label{stack-unstack-of-multilevel-indexing}}

    \begin{tcolorbox}[breakable, size=fbox, boxrule=1pt, pad at break*=1mm,colback=cellbackground, colframe=cellborder]
\prompt{In}{incolor}{113}{\boxspacing}
\begin{Verbatim}[commandchars=\\\{\}]
\PY{n}{DF1}\PY{o}{=}\PY{n}{pd}\PY{o}{.}\PY{n}{read\PYZus{}excel}\PY{p}{(}\PY{n}{f1}\PY{p}{,}\PY{n}{sheet\PYZus{}name}\PY{o}{=}\PY{l+m+mi}{1}\PY{p}{,}\PY{n}{na\PYZus{}values}\PY{o}{=}\PY{p}{[}\PY{l+s+s1}{\PYZsq{}}\PY{l+s+s1}{NIL}\PY{l+s+s1}{\PYZsq{}}\PY{p}{]}\PY{p}{)}
\end{Verbatim}
\end{tcolorbox}

    \begin{tcolorbox}[breakable, size=fbox, boxrule=1pt, pad at break*=1mm,colback=cellbackground, colframe=cellborder]
\prompt{In}{incolor}{114}{\boxspacing}
\begin{Verbatim}[commandchars=\\\{\}]
\PY{n}{DF1}\PY{o}{.}\PY{n}{columns}\PY{o}{=}\PY{p}{[}\PY{l+s+s1}{\PYZsq{}}\PY{l+s+s1}{course}\PY{l+s+s1}{\PYZsq{}}\PY{p}{,}\PY{l+s+s1}{\PYZsq{}}\PY{l+s+s1}{subject type}\PY{l+s+s1}{\PYZsq{}}\PY{p}{,}\PY{l+s+s1}{\PYZsq{}}\PY{l+s+s1}{Year I}\PY{l+s+s1}{\PYZsq{}}\PY{p}{,}\PY{l+s+s1}{\PYZsq{}}\PY{l+s+s1}{Year II}\PY{l+s+s1}{\PYZsq{}}\PY{p}{,}\PY{l+s+s1}{\PYZsq{}}\PY{l+s+s1}{Year III}\PY{l+s+s1}{\PYZsq{}}\PY{p}{]}
\end{Verbatim}
\end{tcolorbox}

    \begin{tcolorbox}[breakable, size=fbox, boxrule=1pt, pad at break*=1mm,colback=cellbackground, colframe=cellborder]
\prompt{In}{incolor}{115}{\boxspacing}
\begin{Verbatim}[commandchars=\\\{\}]
\PY{n}{DF1}
\end{Verbatim}
\end{tcolorbox}

            \begin{tcolorbox}[breakable, size=fbox, boxrule=.5pt, pad at break*=1mm, opacityfill=0]
\prompt{Out}{outcolor}{115}{\boxspacing}
\begin{Verbatim}[commandchars=\\\{\}]
    course subject type  Year I  Year II  Year III
0     PSCS         Core     6.0      6.0       8.0
1     PSCS          Sec     0.0      2.0       2.0
2     PSCS           DS     0.0      0.0       2.0
3     PSCS         AECC     2.0      0.0       0.0
4     PSCS           GE     NaN      NaN       NaN
5   CSHons         Core     6.0      6.0       4.0
6   CSHons          Sec     0.0      2.0       2.0
7   CSHons           DS     0.0      0.0       4.0
8   CSHons         AECC     2.0      0.0       0.0
9   CSHons           GE     2.0      2.0       0.0
10    Bcom         Core     6.0      6.0       4.0
11    Bcom          Sec     0.0      2.0       2.0
12    Bcom           DS     0.0      0.0       4.0
13    Bcom         AECC     2.0      0.0       0.0
14    Bcom           GE     2.0      2.0       0.0
\end{Verbatim}
\end{tcolorbox}
        
    \hypertarget{stack-coverts-column-levels-to-row-level-in-case-only-one-level-at-column-then-output-is-a-series.-in-case-mlevel-inner-most-colum-is-changed-to-row-index}{%
\subsection{Stack() coverts column levels to row level, in case only one
level at column then output is a series. In case MLevel, inner most
colum is changed to row
index}\label{stack-coverts-column-levels-to-row-level-in-case-only-one-level-at-column-then-output-is-a-series.-in-case-mlevel-inner-most-colum-is-changed-to-row-index}}

    \begin{tcolorbox}[breakable, size=fbox, boxrule=1pt, pad at break*=1mm,colback=cellbackground, colframe=cellborder]
\prompt{In}{incolor}{116}{\boxspacing}
\begin{Verbatim}[commandchars=\\\{\}]
\PY{n}{DF2}\PY{o}{=}\PY{n}{DF1}\PY{o}{.}\PY{n}{stack}\PY{p}{(}\PY{p}{)}
\end{Verbatim}
\end{tcolorbox}

    \begin{tcolorbox}[breakable, size=fbox, boxrule=1pt, pad at break*=1mm,colback=cellbackground, colframe=cellborder]
\prompt{In}{incolor}{117}{\boxspacing}
\begin{Verbatim}[commandchars=\\\{\}]
\PY{n}{DF2}
\end{Verbatim}
\end{tcolorbox}

            \begin{tcolorbox}[breakable, size=fbox, boxrule=.5pt, pad at break*=1mm, opacityfill=0]
\prompt{Out}{outcolor}{117}{\boxspacing}
\begin{Verbatim}[commandchars=\\\{\}]
0   course          PSCS
    subject type    Core
    Year I           6.0
    Year II          6.0
    Year III         8.0
                    {\ldots}
14  course          Bcom
    subject type      GE
    Year I           2.0
    Year II          2.0
    Year III         0.0
Length: 72, dtype: object
\end{Verbatim}
\end{tcolorbox}
        
    \begin{tcolorbox}[breakable, size=fbox, boxrule=1pt, pad at break*=1mm,colback=cellbackground, colframe=cellborder]
\prompt{In}{incolor}{99}{\boxspacing}
\begin{Verbatim}[commandchars=\\\{\}]
\PY{n}{DF2}\PY{o}{.}\PY{n}{index}
\end{Verbatim}
\end{tcolorbox}

            \begin{tcolorbox}[breakable, size=fbox, boxrule=.5pt, pad at break*=1mm, opacityfill=0]
\prompt{Out}{outcolor}{99}{\boxspacing}
\begin{Verbatim}[commandchars=\\\{\}]
MultiIndex([( 0,       'course'),
            ( 0, 'subject type'),
            ( 0,       'Year I'),
            ( 0,      'Year II'),
            ( 0,     'Year III'),
            ( 1,       'course'),
            ( 1, 'subject type'),
            ( 1,       'Year I'),
            ( 1,      'Year II'),
            ( 1,     'Year III'),
            ( 2,       'course'),
            ( 2, 'subject type'),
            ( 2,       'Year I'),
            ( 2,      'Year II'),
            ( 2,     'Year III'),
            ( 3,       'course'),
            ( 3, 'subject type'),
            ( 3,       'Year I'),
            ( 3,      'Year II'),
            ( 3,     'Year III'),
            ( 4,       'course'),
            ( 4, 'subject type'),
            ( 5,       'course'),
            ( 5, 'subject type'),
            ( 5,       'Year I'),
            ( 5,      'Year II'),
            ( 5,     'Year III'),
            ( 6,       'course'),
            ( 6, 'subject type'),
            ( 6,       'Year I'),
            ( 6,      'Year II'),
            ( 6,     'Year III'),
            ( 7,       'course'),
            ( 7, 'subject type'),
            ( 7,       'Year I'),
            ( 7,      'Year II'),
            ( 7,     'Year III'),
            ( 8,       'course'),
            ( 8, 'subject type'),
            ( 8,       'Year I'),
            ( 8,      'Year II'),
            ( 8,     'Year III'),
            ( 9,       'course'),
            ( 9, 'subject type'),
            ( 9,       'Year I'),
            ( 9,      'Year II'),
            ( 9,     'Year III'),
            (10,       'course'),
            (10, 'subject type'),
            (10,       'Year I'),
            (10,      'Year II'),
            (10,     'Year III'),
            (11,       'course'),
            (11, 'subject type'),
            (11,       'Year I'),
            (11,      'Year II'),
            (11,     'Year III'),
            (12,       'course'),
            (12, 'subject type'),
            (12,       'Year I'),
            (12,      'Year II'),
            (12,     'Year III'),
            (13,       'course'),
            (13, 'subject type'),
            (13,       'Year I'),
            (13,      'Year II'),
            (13,     'Year III'),
            (14,       'course'),
            (14, 'subject type'),
            (14,       'Year I'),
            (14,      'Year II'),
            (14,     'Year III')],
           )
\end{Verbatim}
\end{tcolorbox}
        
    \begin{tcolorbox}[breakable, size=fbox, boxrule=1pt, pad at break*=1mm,colback=cellbackground, colframe=cellborder]
\prompt{In}{incolor}{100}{\boxspacing}
\begin{Verbatim}[commandchars=\\\{\}]
\PY{n}{DF2}
\end{Verbatim}
\end{tcolorbox}

            \begin{tcolorbox}[breakable, size=fbox, boxrule=.5pt, pad at break*=1mm, opacityfill=0]
\prompt{Out}{outcolor}{100}{\boxspacing}
\begin{Verbatim}[commandchars=\\\{\}]
0   course          PSCS
    subject type    Core
    Year I           6.0
    Year II          6.0
    Year III         8.0
                    {\ldots}
14  course          Bcom
    subject type      GE
    Year I           2.0
    Year II          2.0
    Year III         0.0
Length: 72, dtype: object
\end{Verbatim}
\end{tcolorbox}
        
    \begin{tcolorbox}[breakable, size=fbox, boxrule=1pt, pad at break*=1mm,colback=cellbackground, colframe=cellborder]
\prompt{In}{incolor}{119}{\boxspacing}
\begin{Verbatim}[commandchars=\\\{\}]
\PY{n}{DF2}\PY{p}{[}\PY{p}{(}\PY{l+m+mi}{14}\PY{p}{,}\PY{l+s+s1}{\PYZsq{}}\PY{l+s+s1}{course}\PY{l+s+s1}{\PYZsq{}}\PY{p}{)}\PY{p}{]}
\end{Verbatim}
\end{tcolorbox}

            \begin{tcolorbox}[breakable, size=fbox, boxrule=.5pt, pad at break*=1mm, opacityfill=0]
\prompt{Out}{outcolor}{119}{\boxspacing}
\begin{Verbatim}[commandchars=\\\{\}]
'Bcom'
\end{Verbatim}
\end{tcolorbox}
        
    \begin{tcolorbox}[breakable, size=fbox, boxrule=1pt, pad at break*=1mm,colback=cellbackground, colframe=cellborder]
\prompt{In}{incolor}{120}{\boxspacing}
\begin{Verbatim}[commandchars=\\\{\}]
\PY{n}{dftemp}
\end{Verbatim}
\end{tcolorbox}

            \begin{tcolorbox}[breakable, size=fbox, boxrule=.5pt, pad at break*=1mm, opacityfill=0]
\prompt{Out}{outcolor}{120}{\boxspacing}
\begin{Verbatim}[commandchars=\\\{\}]
  level1            level2
     key data1  new    key data1   new
0      b   0.0  0.0      a   0.0   1.0
1      b   1.0  1.0      b   1.0   2.0
2      a   2.0  2.0      d   2.0  10.0
3      c   3.0  3.0    NaN   NaN   NaN
4      a   4.0  4.0    NaN   NaN   NaN
5      a   5.0  5.0    NaN   NaN   NaN
6      b   6.0  6.0    NaN   NaN   NaN
9    NaN   NaN  NaN      a   0.0   1.0
\end{Verbatim}
\end{tcolorbox}
        
    \begin{tcolorbox}[breakable, size=fbox, boxrule=1pt, pad at break*=1mm,colback=cellbackground, colframe=cellborder]
\prompt{In}{incolor}{121}{\boxspacing}
\begin{Verbatim}[commandchars=\\\{\}]
\PY{n}{dftemp}\PY{o}{.}\PY{n}{stack}\PY{p}{(}\PY{p}{)}
\end{Verbatim}
\end{tcolorbox}

            \begin{tcolorbox}[breakable, size=fbox, boxrule=.5pt, pad at break*=1mm, opacityfill=0]
\prompt{Out}{outcolor}{121}{\boxspacing}
\begin{Verbatim}[commandchars=\\\{\}]
        level1 level2
0 key        b      a
  data1    0.0    0.0
  new      0.0    1.0
1 key        b      b
  data1    1.0    1.0
  new      1.0    2.0
2 key        a      d
  data1    2.0    2.0
  new      2.0   10.0
3 key        c    NaN
  data1    3.0    NaN
  new      3.0    NaN
4 key        a    NaN
  data1    4.0    NaN
  new      4.0    NaN
5 key        a    NaN
  data1    5.0    NaN
  new      5.0    NaN
6 key        b    NaN
  data1    6.0    NaN
  new      6.0    NaN
9 key      NaN      a
  data1    NaN    0.0
  new      NaN    1.0
\end{Verbatim}
\end{tcolorbox}
        
    \hypertarget{setting-first-two-columns-as-index}{%
\section{setting first two columns as
index}\label{setting-first-two-columns-as-index}}

    \begin{tcolorbox}[breakable, size=fbox, boxrule=1pt, pad at break*=1mm,colback=cellbackground, colframe=cellborder]
\prompt{In}{incolor}{103}{\boxspacing}
\begin{Verbatim}[commandchars=\\\{\}]
\PY{n}{DF1}\PY{o}{.}\PY{n}{columns}
\end{Verbatim}
\end{tcolorbox}

            \begin{tcolorbox}[breakable, size=fbox, boxrule=.5pt, pad at break*=1mm, opacityfill=0]
\prompt{Out}{outcolor}{103}{\boxspacing}
\begin{Verbatim}[commandchars=\\\{\}]
Index(['course', 'subject type', 'Year I', 'Year II', 'Year III'],
dtype='object')
\end{Verbatim}
\end{tcolorbox}
        
    \begin{tcolorbox}[breakable, size=fbox, boxrule=1pt, pad at break*=1mm,colback=cellbackground, colframe=cellborder]
\prompt{In}{incolor}{122}{\boxspacing}
\begin{Verbatim}[commandchars=\\\{\}]
\PY{n}{DF1}
\end{Verbatim}
\end{tcolorbox}

            \begin{tcolorbox}[breakable, size=fbox, boxrule=.5pt, pad at break*=1mm, opacityfill=0]
\prompt{Out}{outcolor}{122}{\boxspacing}
\begin{Verbatim}[commandchars=\\\{\}]
    course subject type  Year I  Year II  Year III
0     PSCS         Core     6.0      6.0       8.0
1     PSCS          Sec     0.0      2.0       2.0
2     PSCS           DS     0.0      0.0       2.0
3     PSCS         AECC     2.0      0.0       0.0
4     PSCS           GE     NaN      NaN       NaN
5   CSHons         Core     6.0      6.0       4.0
6   CSHons          Sec     0.0      2.0       2.0
7   CSHons           DS     0.0      0.0       4.0
8   CSHons         AECC     2.0      0.0       0.0
9   CSHons           GE     2.0      2.0       0.0
10    Bcom         Core     6.0      6.0       4.0
11    Bcom          Sec     0.0      2.0       2.0
12    Bcom           DS     0.0      0.0       4.0
13    Bcom         AECC     2.0      0.0       0.0
14    Bcom           GE     2.0      2.0       0.0
\end{Verbatim}
\end{tcolorbox}
        
    \hypertarget{renaming-specific-columns}{%
\subsubsection{renaming specific
columns}\label{renaming-specific-columns}}

    \begin{tcolorbox}[breakable, size=fbox, boxrule=1pt, pad at break*=1mm,colback=cellbackground, colframe=cellborder]
\prompt{In}{incolor}{97}{\boxspacing}
\begin{Verbatim}[commandchars=\\\{\}]
\PY{n}{DF1}\PY{o}{.}\PY{n}{rename}\PY{p}{(}\PY{n}{columns}\PY{o}{=}\PY{p}{\PYZob{}}\PY{l+s+s1}{\PYZsq{}}\PY{l+s+s1}{Unnamed: 0}\PY{l+s+s1}{\PYZsq{}}\PY{p}{:}\PY{l+s+s1}{\PYZsq{}}\PY{l+s+s1}{Course}\PY{l+s+s1}{\PYZsq{}}\PY{p}{,}\PY{l+s+s1}{\PYZsq{}}\PY{l+s+s1}{Unnamed: 1}\PY{l+s+s1}{\PYZsq{}}\PY{p}{:}\PY{l+s+s1}{\PYZsq{}}\PY{l+s+s1}{Types}\PY{l+s+s1}{\PYZsq{}}\PY{p}{\PYZcb{}}\PY{p}{,}\PY{n}{inplace}\PY{o}{=}\PY{k+kc}{True}\PY{p}{)}
\end{Verbatim}
\end{tcolorbox}

    \begin{tcolorbox}[breakable, size=fbox, boxrule=1pt, pad at break*=1mm,colback=cellbackground, colframe=cellborder]
\prompt{In}{incolor}{123}{\boxspacing}
\begin{Verbatim}[commandchars=\\\{\}]
\PY{n}{DF2}\PY{o}{=}\PY{n}{DF1}\PY{o}{.}\PY{n}{set\PYZus{}index}\PY{p}{(}\PY{n}{keys}\PY{o}{=}\PY{p}{[}\PY{n}{DF1}\PY{o}{.}\PY{n}{columns}\PY{p}{[}\PY{l+m+mi}{0}\PY{p}{]}\PY{p}{,}\PY{n}{DF1}\PY{o}{.}\PY{n}{columns}\PY{p}{[}\PY{l+m+mi}{1}\PY{p}{]}\PY{p}{]}\PY{p}{)}
\end{Verbatim}
\end{tcolorbox}

    \begin{tcolorbox}[breakable, size=fbox, boxrule=1pt, pad at break*=1mm,colback=cellbackground, colframe=cellborder]
\prompt{In}{incolor}{124}{\boxspacing}
\begin{Verbatim}[commandchars=\\\{\}]
\PY{n}{DF2}
\end{Verbatim}
\end{tcolorbox}

            \begin{tcolorbox}[breakable, size=fbox, boxrule=.5pt, pad at break*=1mm, opacityfill=0]
\prompt{Out}{outcolor}{124}{\boxspacing}
\begin{Verbatim}[commandchars=\\\{\}]
                     Year I  Year II  Year III
course subject type
PSCS   Core             6.0      6.0       8.0
       Sec              0.0      2.0       2.0
       DS               0.0      0.0       2.0
       AECC             2.0      0.0       0.0
       GE               NaN      NaN       NaN
CSHons Core             6.0      6.0       4.0
       Sec              0.0      2.0       2.0
       DS               0.0      0.0       4.0
       AECC             2.0      0.0       0.0
       GE               2.0      2.0       0.0
Bcom   Core             6.0      6.0       4.0
       Sec              0.0      2.0       2.0
       DS               0.0      0.0       4.0
       AECC             2.0      0.0       0.0
       GE               2.0      2.0       0.0
\end{Verbatim}
\end{tcolorbox}
        
    \hypertarget{stack-changing-columns-as-rows-in-the-df}{%
\subsection{stack: changing columns as rows in the
DF}\label{stack-changing-columns-as-rows-in-the-df}}

\begin{itemize}
\tightlist
\item
  As there may be multiple indices, stacking means converting (also
  called rotating or pivoting) the innermost column index into the
  innermost row index.
\item
  Unstacking: exactly the inverse operation of stacking--- it will
  convert the innermost row index back into the innermost column index.
\end{itemize}

    \hypertarget{row-having-na-will-be-igonred}{%
\subsection{row having NA will be
igonred}\label{row-having-na-will-be-igonred}}

    \begin{tcolorbox}[breakable, size=fbox, boxrule=1pt, pad at break*=1mm,colback=cellbackground, colframe=cellborder]
\prompt{In}{incolor}{125}{\boxspacing}
\begin{Verbatim}[commandchars=\\\{\}]
\PY{n+nb}{type}\PY{p}{(}\PY{n}{DF2}\PY{p}{)}
\end{Verbatim}
\end{tcolorbox}

            \begin{tcolorbox}[breakable, size=fbox, boxrule=.5pt, pad at break*=1mm, opacityfill=0]
\prompt{Out}{outcolor}{125}{\boxspacing}
\begin{Verbatim}[commandchars=\\\{\}]
pandas.core.frame.DataFrame
\end{Verbatim}
\end{tcolorbox}
        
    \begin{tcolorbox}[breakable, size=fbox, boxrule=1pt, pad at break*=1mm,colback=cellbackground, colframe=cellborder]
\prompt{In}{incolor}{136}{\boxspacing}
\begin{Verbatim}[commandchars=\\\{\}]
\PY{n}{DF2}
\end{Verbatim}
\end{tcolorbox}

            \begin{tcolorbox}[breakable, size=fbox, boxrule=.5pt, pad at break*=1mm, opacityfill=0]
\prompt{Out}{outcolor}{136}{\boxspacing}
\begin{Verbatim}[commandchars=\\\{\}]
                     Year I  Year II  Year III
course subject type
PSCS   Core             6.0      6.0       8.0
       Sec              0.0      2.0       2.0
       DS               0.0      0.0       2.0
       AECC             2.0      0.0       0.0
       GE               NaN      NaN       NaN
CSHons Core             6.0      6.0       4.0
       Sec              0.0      2.0       2.0
       DS               0.0      0.0       4.0
       AECC             2.0      0.0       0.0
       GE               2.0      2.0       0.0
Bcom   Core             6.0      6.0       4.0
       Sec              0.0      2.0       2.0
       DS               0.0      0.0       4.0
       AECC             2.0      0.0       0.0
       GE               2.0      2.0       0.0
\end{Verbatim}
\end{tcolorbox}
        
    \begin{tcolorbox}[breakable, size=fbox, boxrule=1pt, pad at break*=1mm,colback=cellbackground, colframe=cellborder]
\prompt{In}{incolor}{127}{\boxspacing}
\begin{Verbatim}[commandchars=\\\{\}]
\PY{n}{dftemp}\PY{o}{.}\PY{n}{stack}\PY{p}{(}\PY{p}{)}\PY{o}{.}\PY{n}{unstack}\PY{p}{(}\PY{p}{)}
\end{Verbatim}
\end{tcolorbox}

            \begin{tcolorbox}[breakable, size=fbox, boxrule=.5pt, pad at break*=1mm, opacityfill=0]
\prompt{Out}{outcolor}{127}{\boxspacing}
\begin{Verbatim}[commandchars=\\\{\}]
  level1            level2
     key data1  new    key data1   new
0      b   0.0  0.0      a   0.0   1.0
1      b   1.0  1.0      b   1.0   2.0
2      a   2.0  2.0      d   2.0  10.0
3      c   3.0  3.0    NaN   NaN   NaN
4      a   4.0  4.0    NaN   NaN   NaN
5      a   5.0  5.0    NaN   NaN   NaN
6      b   6.0  6.0    NaN   NaN   NaN
9    NaN   NaN  NaN      a   0.0   1.0
\end{Verbatim}
\end{tcolorbox}
        
    \begin{tcolorbox}[breakable, size=fbox, boxrule=1pt, pad at break*=1mm,colback=cellbackground, colframe=cellborder]
\prompt{In}{incolor}{137}{\boxspacing}
\begin{Verbatim}[commandchars=\\\{\}]
\PY{n}{DF3}\PY{o}{=}\PY{n}{DF2}\PY{o}{.}\PY{n}{stack}\PY{p}{(}\PY{p}{)}
\end{Verbatim}
\end{tcolorbox}

    \hypertarget{df3-is-a-series-with-one-column-and-multilevel-indices}{%
\subsection{DF3 is a series with one column and multilevel
indices}\label{df3-is-a-series-with-one-column-and-multilevel-indices}}

    \begin{tcolorbox}[breakable, size=fbox, boxrule=1pt, pad at break*=1mm,colback=cellbackground, colframe=cellborder]
\prompt{In}{incolor}{112}{\boxspacing}
\begin{Verbatim}[commandchars=\\\{\}]
\PY{n+nb}{type}\PY{p}{(}\PY{n}{DF3}\PY{p}{)}
\end{Verbatim}
\end{tcolorbox}

            \begin{tcolorbox}[breakable, size=fbox, boxrule=.5pt, pad at break*=1mm, opacityfill=0]
\prompt{Out}{outcolor}{112}{\boxspacing}
\begin{Verbatim}[commandchars=\\\{\}]
pandas.core.series.Series
\end{Verbatim}
\end{tcolorbox}
        
    \begin{tcolorbox}[breakable, size=fbox, boxrule=1pt, pad at break*=1mm,colback=cellbackground, colframe=cellborder]
\prompt{In}{incolor}{135}{\boxspacing}
\begin{Verbatim}[commandchars=\\\{\}]
\PY{n}{DF3}
\end{Verbatim}
\end{tcolorbox}

            \begin{tcolorbox}[breakable, size=fbox, boxrule=.5pt, pad at break*=1mm, opacityfill=0]
\prompt{Out}{outcolor}{135}{\boxspacing}
\begin{Verbatim}[commandchars=\\\{\}]
course  subject type
PSCS    Core          Year I      6.0
                      Year II     6.0
                      Year III    8.0
        Sec           Year I      0.0
                      Year II     2.0
                      Year III    2.0
        DS            Year I      0.0
                      Year II     0.0
                      Year III    2.0
        AECC          Year I      2.0
                      Year II     0.0
                      Year III    0.0
CSHons  Core          Year I      6.0
                      Year II     6.0
                      Year III    4.0
        Sec           Year I      0.0
                      Year II     2.0
                      Year III    2.0
        DS            Year I      0.0
                      Year II     0.0
                      Year III    4.0
        AECC          Year I      2.0
                      Year II     0.0
                      Year III    0.0
        GE            Year I      2.0
                      Year II     2.0
                      Year III    0.0
Bcom    Core          Year I      6.0
                      Year II     6.0
                      Year III    4.0
        Sec           Year I      0.0
                      Year II     2.0
                      Year III    2.0
        DS            Year I      0.0
                      Year II     0.0
                      Year III    4.0
        AECC          Year I      2.0
                      Year II     0.0
                      Year III    0.0
        GE            Year I      2.0
                      Year II     2.0
                      Year III    0.0
dtype: float64
\end{Verbatim}
\end{tcolorbox}
        
    \begin{tcolorbox}[breakable, size=fbox, boxrule=1pt, pad at break*=1mm,colback=cellbackground, colframe=cellborder]
\prompt{In}{incolor}{114}{\boxspacing}
\begin{Verbatim}[commandchars=\\\{\}]
\PY{n}{DF3}\PY{o}{.}\PY{n}{index}\PY{o}{.}\PY{n}{names}
\end{Verbatim}
\end{tcolorbox}

            \begin{tcolorbox}[breakable, size=fbox, boxrule=.5pt, pad at break*=1mm, opacityfill=0]
\prompt{Out}{outcolor}{114}{\boxspacing}
\begin{Verbatim}[commandchars=\\\{\}]
FrozenList(['course', 'subject type', None])
\end{Verbatim}
\end{tcolorbox}
        
    \begin{tcolorbox}[breakable, size=fbox, boxrule=1pt, pad at break*=1mm,colback=cellbackground, colframe=cellborder]
\prompt{In}{incolor}{116}{\boxspacing}
\begin{Verbatim}[commandchars=\\\{\}]
\PY{n}{DF3}\PY{o}{.}\PY{n}{index}\PY{o}{.}\PY{n}{names}\PY{o}{=}\PY{p}{[}\PY{l+s+s1}{\PYZsq{}}\PY{l+s+s1}{Course}\PY{l+s+s1}{\PYZsq{}}\PY{p}{,} \PY{l+s+s1}{\PYZsq{}}\PY{l+s+s1}{Types}\PY{l+s+s1}{\PYZsq{}}\PY{p}{,} \PY{l+s+s1}{\PYZsq{}}\PY{l+s+s1}{Semester}\PY{l+s+s1}{\PYZsq{}}\PY{p}{]}
\end{Verbatim}
\end{tcolorbox}

    \hypertarget{unstackin-displays-row-index-values-as-sorted-by-default-innermost-row-index-level-is-changed-to-lowest-column-level.-but-any-other-row-level-may-also-be-specified}{%
\subsection{unstackin displays row index values as sorted, By default
innermost row index level is changed to lowest column level. but any
other row level may also be
specified}\label{unstackin-displays-row-index-values-as-sorted-by-default-innermost-row-index-level-is-changed-to-lowest-column-level.-but-any-other-row-level-may-also-be-specified}}

    \begin{tcolorbox}[breakable, size=fbox, boxrule=1pt, pad at break*=1mm,colback=cellbackground, colframe=cellborder]
\prompt{In}{incolor}{130}{\boxspacing}
\begin{Verbatim}[commandchars=\\\{\}]
\PY{n}{DF3}\PY{o}{.}\PY{n}{unstack}\PY{p}{(}\PY{p}{)}
\end{Verbatim}
\end{tcolorbox}

            \begin{tcolorbox}[breakable, size=fbox, boxrule=.5pt, pad at break*=1mm, opacityfill=0]
\prompt{Out}{outcolor}{130}{\boxspacing}
\begin{Verbatim}[commandchars=\\\{\}]
                     Year I  Year II  Year III
course subject type
Bcom   AECC             2.0      0.0       0.0
       Core             6.0      6.0       4.0
       DS               0.0      0.0       4.0
       GE               2.0      2.0       0.0
       Sec              0.0      2.0       2.0
CSHons AECC             2.0      0.0       0.0
       Core             6.0      6.0       4.0
       DS               0.0      0.0       4.0
       GE               2.0      2.0       0.0
       Sec              0.0      2.0       2.0
PSCS   AECC             2.0      0.0       0.0
       Core             6.0      6.0       8.0
       DS               0.0      0.0       2.0
       Sec              0.0      2.0       2.0
\end{Verbatim}
\end{tcolorbox}
        
    \begin{tcolorbox}[breakable, size=fbox, boxrule=1pt, pad at break*=1mm,colback=cellbackground, colframe=cellborder]
\prompt{In}{incolor}{118}{\boxspacing}
\begin{Verbatim}[commandchars=\\\{\}]
\PY{n+nb}{type}\PY{p}{(}\PY{n}{DF3}\PY{o}{.}\PY{n}{unstack}\PY{p}{(}\PY{p}{)}\PY{p}{)}
\end{Verbatim}
\end{tcolorbox}

            \begin{tcolorbox}[breakable, size=fbox, boxrule=.5pt, pad at break*=1mm, opacityfill=0]
\prompt{Out}{outcolor}{118}{\boxspacing}
\begin{Verbatim}[commandchars=\\\{\}]
pandas.core.frame.DataFrame
\end{Verbatim}
\end{tcolorbox}
        
    \hypertarget{preserving-of-missing-values}{%
\subsection{Preserving of missing
values}\label{preserving-of-missing-values}}

    \begin{tcolorbox}[breakable, size=fbox, boxrule=1pt, pad at break*=1mm,colback=cellbackground, colframe=cellborder]
\prompt{In}{incolor}{142}{\boxspacing}
\begin{Verbatim}[commandchars=\\\{\}]
\PY{n}{DF3}\PY{o}{=}\PY{n}{DF2}\PY{o}{.}\PY{n}{stack}\PY{p}{(}\PY{n}{dropna}\PY{o}{=}\PY{k+kc}{False}\PY{p}{)}
\end{Verbatim}
\end{tcolorbox}

    \begin{tcolorbox}[breakable, size=fbox, boxrule=1pt, pad at break*=1mm,colback=cellbackground, colframe=cellborder]
\prompt{In}{incolor}{143}{\boxspacing}
\begin{Verbatim}[commandchars=\\\{\}]
\PY{n}{DF3}
\end{Verbatim}
\end{tcolorbox}

            \begin{tcolorbox}[breakable, size=fbox, boxrule=.5pt, pad at break*=1mm, opacityfill=0]
\prompt{Out}{outcolor}{143}{\boxspacing}
\begin{Verbatim}[commandchars=\\\{\}]
course  subject type
PSCS    Core          Year I      6.0
                      Year II     6.0
                      Year III    8.0
        Sec           Year I      0.0
                      Year II     2.0
                      Year III    2.0
        DS            Year I      0.0
                      Year II     0.0
                      Year III    2.0
        AECC          Year I      2.0
                      Year II     0.0
                      Year III    0.0
        GE            Year I      NaN
                      Year II     NaN
                      Year III    NaN
CSHons  Core          Year I      6.0
                      Year II     6.0
                      Year III    4.0
        Sec           Year I      0.0
                      Year II     2.0
                      Year III    2.0
        DS            Year I      0.0
                      Year II     0.0
                      Year III    4.0
        AECC          Year I      2.0
                      Year II     0.0
                      Year III    0.0
        GE            Year I      2.0
                      Year II     2.0
                      Year III    0.0
Bcom    Core          Year I      6.0
                      Year II     6.0
                      Year III    4.0
        Sec           Year I      0.0
                      Year II     2.0
                      Year III    2.0
        DS            Year I      0.0
                      Year II     0.0
                      Year III    4.0
        AECC          Year I      2.0
                      Year II     0.0
                      Year III    0.0
        GE            Year I      2.0
                      Year II     2.0
                      Year III    0.0
dtype: float64
\end{Verbatim}
\end{tcolorbox}
        
    \begin{tcolorbox}[breakable, size=fbox, boxrule=1pt, pad at break*=1mm,colback=cellbackground, colframe=cellborder]
\prompt{In}{incolor}{133}{\boxspacing}
\begin{Verbatim}[commandchars=\\\{\}]
\PY{n}{DF3}\PY{o}{.}\PY{n}{unstack}\PY{p}{(}\PY{p}{)}
\end{Verbatim}
\end{tcolorbox}

            \begin{tcolorbox}[breakable, size=fbox, boxrule=.5pt, pad at break*=1mm, opacityfill=0]
\prompt{Out}{outcolor}{133}{\boxspacing}
\begin{Verbatim}[commandchars=\\\{\}]
                     Year I  Year II  Year III
course subject type
Bcom   AECC             2.0      0.0       0.0
       Core             6.0      6.0       4.0
       DS               0.0      0.0       4.0
       GE               2.0      2.0       0.0
       Sec              0.0      2.0       2.0
CSHons AECC             2.0      0.0       0.0
       Core             6.0      6.0       4.0
       DS               0.0      0.0       4.0
       GE               2.0      2.0       0.0
       Sec              0.0      2.0       2.0
PSCS   AECC             2.0      0.0       0.0
       Core             6.0      6.0       8.0
       DS               0.0      0.0       2.0
       GE               NaN      NaN       NaN
       Sec              0.0      2.0       2.0
\end{Verbatim}
\end{tcolorbox}
        
    \hypertarget{incase-types-of-subjects-per-year-per-course-are-to-be-displayed-together-for-comparison}{%
\subsection{incase types of subjects per year per course are to be
displayed together for
comparison}\label{incase-types-of-subjects-per-year-per-course-are-to-be-displayed-together-for-comparison}}

    \begin{tcolorbox}[breakable, size=fbox, boxrule=1pt, pad at break*=1mm,colback=cellbackground, colframe=cellborder]
\prompt{In}{incolor}{126}{\boxspacing}
\begin{Verbatim}[commandchars=\\\{\}]
\PY{n}{DF3}
\end{Verbatim}
\end{tcolorbox}

            \begin{tcolorbox}[breakable, size=fbox, boxrule=.5pt, pad at break*=1mm, opacityfill=0]
\prompt{Out}{outcolor}{126}{\boxspacing}
\begin{Verbatim}[commandchars=\\\{\}]
course  subject type
PSCS    Core          Year I      6.0
                      Year II     6.0
                      Year III    8.0
        Sec           Year I      0.0
                      Year II     2.0
                      Year III    2.0
        DS            Year I      0.0
                      Year II     0.0
                      Year III    2.0
        AECC          Year I      2.0
                      Year II     0.0
                      Year III    0.0
        GE            Year I      NaN
                      Year II     NaN
                      Year III    NaN
CSHons  Core          Year I      6.0
                      Year II     6.0
                      Year III    4.0
        Sec           Year I      0.0
                      Year II     2.0
                      Year III    2.0
        DS            Year I      0.0
                      Year II     0.0
                      Year III    4.0
        AECC          Year I      2.0
                      Year II     0.0
                      Year III    0.0
        GE            Year I      2.0
                      Year II     2.0
                      Year III    0.0
Bcom    Core          Year I      6.0
                      Year II     6.0
                      Year III    4.0
        Sec           Year I      0.0
                      Year II     2.0
                      Year III    2.0
        DS            Year I      0.0
                      Year II     0.0
                      Year III    4.0
        AECC          Year I      2.0
                      Year II     0.0
                      Year III    0.0
        GE            Year I      2.0
                      Year II     2.0
                      Year III    0.0
dtype: float64
\end{Verbatim}
\end{tcolorbox}
        
    \begin{tcolorbox}[breakable, size=fbox, boxrule=1pt, pad at break*=1mm,colback=cellbackground, colframe=cellborder]
\prompt{In}{incolor}{127}{\boxspacing}
\begin{Verbatim}[commandchars=\\\{\}]
\PY{n}{DF4}\PY{o}{=}\PY{n}{DF3}\PY{o}{.}\PY{n}{unstack}\PY{p}{(}\PY{n}{level}\PY{o}{=}\PY{l+m+mi}{0}\PY{p}{)}
\end{Verbatim}
\end{tcolorbox}

    \begin{tcolorbox}[breakable, size=fbox, boxrule=1pt, pad at break*=1mm,colback=cellbackground, colframe=cellborder]
\prompt{In}{incolor}{129}{\boxspacing}
\begin{Verbatim}[commandchars=\\\{\}]
\PY{n}{DF4}
\end{Verbatim}
\end{tcolorbox}

            \begin{tcolorbox}[breakable, size=fbox, boxrule=.5pt, pad at break*=1mm, opacityfill=0]
\prompt{Out}{outcolor}{129}{\boxspacing}
\begin{Verbatim}[commandchars=\\\{\}]
course                 Bcom  CSHons  PSCS
subject type
AECC         Year I     2.0     2.0   2.0
             Year II    0.0     0.0   0.0
             Year III   0.0     0.0   0.0
Core         Year I     6.0     6.0   6.0
             Year II    6.0     6.0   6.0
             Year III   4.0     4.0   8.0
DS           Year I     0.0     0.0   0.0
             Year II    0.0     0.0   0.0
             Year III   4.0     4.0   2.0
GE           Year I     2.0     2.0   NaN
             Year II    2.0     2.0   NaN
             Year III   0.0     0.0   NaN
Sec          Year I     0.0     0.0   0.0
             Year II    2.0     2.0   2.0
             Year III   2.0     2.0   2.0
\end{Verbatim}
\end{tcolorbox}
        
    \begin{tcolorbox}[breakable, size=fbox, boxrule=1pt, pad at break*=1mm,colback=cellbackground, colframe=cellborder]
\prompt{In}{incolor}{124}{\boxspacing}
\begin{Verbatim}[commandchars=\\\{\}]
\PY{n}{DF4}\PY{o}{.}\PY{n}{columns}
\end{Verbatim}
\end{tcolorbox}

            \begin{tcolorbox}[breakable, size=fbox, boxrule=.5pt, pad at break*=1mm, opacityfill=0]
\prompt{Out}{outcolor}{124}{\boxspacing}
\begin{Verbatim}[commandchars=\\\{\}]
Index(['Bcom', 'CSHons', 'PSCS'], dtype='object', name='course')
\end{Verbatim}
\end{tcolorbox}
        
    \begin{tcolorbox}[breakable, size=fbox, boxrule=1pt, pad at break*=1mm,colback=cellbackground, colframe=cellborder]
\prompt{In}{incolor}{125}{\boxspacing}
\begin{Verbatim}[commandchars=\\\{\}]
\PY{n}{DF4}\PY{o}{.}\PY{n}{index}
\end{Verbatim}
\end{tcolorbox}

            \begin{tcolorbox}[breakable, size=fbox, boxrule=.5pt, pad at break*=1mm, opacityfill=0]
\prompt{Out}{outcolor}{125}{\boxspacing}
\begin{Verbatim}[commandchars=\\\{\}]
MultiIndex([('AECC',   'Year I'),
            ('AECC',  'Year II'),
            ('AECC', 'Year III'),
            ('Core',   'Year I'),
            ('Core',  'Year II'),
            ('Core', 'Year III'),
            (  'DS',   'Year I'),
            (  'DS',  'Year II'),
            (  'DS', 'Year III'),
            (  'GE',   'Year I'),
            (  'GE',  'Year II'),
            (  'GE', 'Year III'),
            ( 'Sec',   'Year I'),
            ( 'Sec',  'Year II'),
            ( 'Sec', 'Year III')],
           names=['subject type', None])
\end{Verbatim}
\end{tcolorbox}
        

    % Add a bibliography block to the postdoc
    
    
    
\end{document}
